\chapter{Standard Model}

The standard model of particle physics is a theoretical model that describes
almost all sub-nuclear phenomena up to an energy scale of
$\mathcal{o}(\unit{100}{\GeV})$.

The theory was pieced togethere in the 60s and 70s and comprises two families of
elementary particles, called quarks and leptons, and incorporates the theories
of quantum electrodynamics (QED), Glashow-Weinberg-Salam theory of electroweak
dynamics and quantum chromodynamics (QCD).

The Standard Model is remarkable in its accuracy, describing every exp[erimental
test performed.

The chapter will first introduce the two families of elementary matter
particles, the leptons and quarks, and then summarize the theoretical background
to the standard model.

\todo{here is a long to do note, this might break}

%Brief history 

\section{Constituents of the Standard Model}
With only a few exceptions, all results from high energy physics experiments can
be explained within the Standard Model of particles and their interactions.

\subsection{Fermions}
Within the standard model all matter is formed from spin $\nicefrac{1}{2}$
particles called fermions.  These comprise 6 leptons and 6 quarks, both split in
to three generations.

Leptons are particles with integer charge and are summarised in \TableRef{}. The
electon, muon and tau each have unit charge, where as the corresponding
neutrinos are neutral. 

Quarks carry fractional charges of $\nicefrac{2}{3}$ and $\nicefrac{-1}{3}$ and
are summarised in \TableRef{}. 

% matter particles
%% 1/2 spin 
%% obey pauli exclusion principle
%% 6 leptons and 6 quarks in 3 generations
%%TABLE

The electron, muon and tau each carry unit negative charge. 

Each charged lepton has a neutral partner called the neutrino. 

% Quarks charged
\subsection{Interactions}

\section{Theoretical Background}

\subsection{Gauge Invariance}

\subsection{QED}

\subsection{QCD}

\subsection{Electroweak Unification}

\subsection{Higgs Mechanism}


