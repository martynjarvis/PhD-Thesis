\chapter{Standard Model}

The standard model of particle physics is a theoretical model that describes
almost all sub-nuclear phenomena up to an energy scale of
$\mathcal{O}(\unit{100}{\GeV})$.

The theory was pieced togethere in the 60s and 70s and comprises two families of
elementary particles, called quarks and leptons, and incorporates the theories
of quantum electrodynamics (QED), Glashow-Weinberg-Salam theory of electroweak
dynamics and quantum chromodynamics (QCD).

The Standard Model is remarkable in its accuracy, describing every experimental
test performed.  
The chapter will first introduce the two families of elementary matter
particles, the leptons and quarks, and then summarize the theoretical background
to the standard model.


%Brief history 

\section{Constituents of the Standard Model}
Within the Standard Model matter is described as being constructed from a small
number of spin$=\nicefrac{1}{2}$ particles called fermions that interact with
the electromagnetic, weak and strong forces. The fermions are divided in to two
families called, leptons and quarks, according to their experimentally measured
properties such as their charge. Each fermion varies in mass from the light
neutrinos (\todo{neutrino mass here}) to the heavy top quark (\todo{top mass
here}).

Each family of fermions can be further subdivided in to three generations that
increase progressivly increase in mass. The fermions of the Standard Model and
some of their properties are sumarised in \TableRef{}. \todo{Add table of SM
particles.}

Each lepton generation contains a charged lepton and a corresponding light
neutral particle called a neutrino. leptons interact via the weak force and the
electromagnetic force in the case of the charged lepton.  The first generation
of leptons is the most familiar containing the electron and the electron
neutrino.

Unlike leptons, quarks carry fractional charge, within each generation there is
a quark with a charge of $\nicefrac{2}{3}$ and another with a charge of
$\nicefrac{-1}{3}$. In  addtion to electric charge, quarks carry an additional
charge called the colour charge. The colour charge allows the quarks to interact
via the strong force in addition the electromagnetic and the weak forces.
The first generation of quarks is again the most familiar, containing the up and
down type quarks that are the constituents of the proton and neutron, which in
turn, with the electron, form atoms and all familiar matter.

The heavier generations of quarks (strange, charm, bottom and top, or \Pstrange,
\Pcharm, \Pbottom and \Ptop) are unstable and decay eventually to \Pup or
\Pdown.
The havy leptons, muon (\Pmuon) and tau (\Ptau), decay in a similar manner to
the stable electron. 

Due to the instability of the heavier generations of fermions, they can only be
found in  cosmic rays, or produced in high energy physics experiments.

\section{Theoretical Background}

\subsection{QED}
This section will introduce quantum electrodynamics and the ideas that drive the
theoretical background to the Standard Model.

\subsubsection{Symmetry and Groups}
All intereaction are dictated by sysmetry principles. Noether's theorem states
that for every symmetry of nature, there is a corresponding quantity that is
conserved and conversely each conservation law is the result of some underlying
symmetry.

For example, the laws of physics are symmetrical with respect to translations in
time ad space, and this leads directly to conservation of energy and momentum
respectivly. Other symmetries and conservation laws are summarised in
\TableRef{}\todo{add a table of conoservation laws.}

Symmetry operations, such as rotations, can be described by mathematical groups.
The most common groups to physics are unitary groups, $U(n)$, which are the
collection of all the $n\times n$ unitary ($U^{-1} = \tilde{U}^{*}$) matrices, and
the special unitary groups, $SU(2)$, which have the additional requirement that
the matrices have a determinant of 1.

In QED an electron can be described as a complex field and the lagrangian is
given by


\begin{equation}
\mathcal{L} = i \bar{psi} \gamma_{\mu} \delta^{\mu} \psi - m \bar{\psi}\psi
\end{equation}

the lagrangian is invariant under a `global' phase transformation

\begin{equation}
\psi(x) \to e^{i\alpha} \psi(x)
\label{eq:global}
\end{equation}

where $\alpha$ is a global arbitary parameter. The lagrangian is said to exhibit
global gauge invariance. The family of transfomations, $R =
e^{i \alpha}$, where $\alpha$ is real and continuous, forms the unitary
abelian \footnote{Abelian refers to the property that all the elements in a
group commute, $R(\alpha)R(\beta) = R(\beta)R(\alpha)$.}
group $U(1)$. 
\todo{this is all horrible}

More generally, the \EquationRef{eq:global} can be rewritten as 
\begin{equation}
\psi(x) \to e^{i\alpha(x)} \psi(x)
\label{eq:local}
\end{equation}

where $\alpha$ is now a function of the space time coordinate $x$. This is known
as a local phase transformation. Unfortunalty under this transformation the
lagrangian is no longer invariant. This can be overcome by defining the
covariant derivative as 
\begin{equation}
D_{\mu} \equiv \delta_{\mu} - i e A_{\mu}
\end{equation}
where an aditional vector field $A_{\mu}$ has been added, that transforms as 
\begin{equation}
A_{\mu} \to A_{\mu} + \frac{1}{e} \delta_{\mu} \alpha
\end{equation}
And the new lagrangian becomes
\begin{equation}
\mathcal{L} = 
\bar{\psi}(i\gamma^{\mu}\delta_{\mu} - m)\psi + 
e \bar{psi} \gamma^{\mu} A_{\mu} \psi - 
\frac{1}{4} F_{\mu\nu} F^{\mu\nu}
\end{equation}

It is seen that the additional vector field, $A_{\mu}$, couples with the dirac
particle, and can be identified as the physical photon field. The additional
term, $\frac{1}{4} F_{\mu\nu} F^{\mu\nu}$, is the kinetic

Local gauge invariance has been restored with the introduction of the photon
field. It is also seen that the photon muset be massless, as the introduction of
a mass term for the photon $\frac{1}{2}A_{\mu}A^{\mu}$\todo{I hope I can prove that in a viva},
similar to the mass term in the original lagrangian for dirac particle, breaks
the gauge invariance.

\subsection{Electroweak Unification}
\todo{Write this, state rather than derive it}

\subsection{Higgs Mechanism}
\todo{Write this, state rather than derive it}

\subsection{QCD}
Quantum chromodynamics follows from similar reasoning to the QED case, but with
the $U(1)$ symmetry group replaced with the SU(3) symmetry group of
transformations on the quark colour fields.

The local gauage phase transformation becomes

\begin{equation}
q(x) \to e^{i\alpha_a(x)T_a} q(x)
\end{equation}

Which breaks the invariance of the lagrangian. Again this is overcome by
introducing the covariant derivative

\begin{equation}
D_{\mu} \equiv \delta_{\mu} + i g T_{a} G_{\mu}^{a}
\end{equation}

Where eight gauge fields have been introduced, instead of the single field in
QED.
The gauge fields transform as 
\begin{equation}
 G_{\mu}^{a} \to G_{\mu}^{a} 
-\frac{1}{g}\delta_{\mu}\alpha_{a}]
-f_{abc}\alpha_{b}G^{c}_{\mu}
\end{equation}
Where the additional term is to produce a gauge invariant lagrangian due to the
non-Abelian gauge transformation.

The final lagrangian for QCD can now be written.
\begin{equation}
\mathcal{L} = 
\bar{q}(i\gamma^{\mu}\delta_{\mu} - m)q -
g \bar{q} \gamma^{\mu} G_{\mu}^{a} - 
\frac{1}{4} G_{\mu\nu}^{a} G^{\mu\nu}_{a}
\end{equation}

where the field strength tensor $G^{\mu\nu}_{a}$ is given by:
\begin{equation}
G^{\mu\nu}_{a} 
= \delta{\mu} G^{a}_{\nu}
- \delta{\nu} G^{a}_{\mu}
-g f_{abc} G^{b}_{\mu} G^{c}_{\nu}
\end{equation}

In a simuilar way to QED, the lagrangian for interacting coloured quarks, $q$, and
vector gluons, $G_{\mu}$, result from the simple erquirement of local colour
phase invariance of the quark fields. Unlike the QED case, eight gauge fields
are needed due to the three quark colour fields.  Similar to QED, the gluons are
required to  be massless.

The field strength tensor, $G^{\mu\nu}_{a}$, introduces terms that are cubic and
quadratic in $G$. These represent three and four vertex gluon interactions that
are a results of the non abelian nature of the $SU(3)$ group.

