\chapter{Standard Model}
\label{chap:sm}
The Standard Model (SM) of particle physics is a quantum field theory that
describes all known fundamental constituents of matter and their interactions
through the strong, weak and electromagnetic forces.  The theory was pieced
together in the 60s and 70s and comprises the theories of quantum
electrodynamics (QED), Glashow-Weinberg-Salam theory of electroweak dynamics and
quantum chromodynamics (QCD) \cite{t1972regularization, glashow1961partial,
weinberg1967model, salam1968weak}.

The Standard Model is remarkable in its accuracy, describing every experimental
test performed to a high degree of precision and accurately predicts all
processes that are observed in nature, with the exception of gravity and
neutrino oscillations.

This chapter reviews the Standard Model of particle physics. The first section will 
introduce the matter constituents of the Standard Model; the leptons and quarks,
and the latter section will describe the fundamental interactions. 
The description follows the example given in \cite{ral} with additional insights
from \cite{perkins2000introduction,griffiths2008introduction,halzen1984quarks}.

\section{Constituents of the Standard Model}
\label{sec:matter}
Within the Standard Model, matter is described as being constructed from a small
number of $\text{spin}=\nicefrac{1}{2}$ particles called fermions that interact with
the electromagnetic, weak and strong forces. 
The fields corresponding to the fermions are described by the Dirac
equation,
\begin{equation}
\left( i \gamma^{\mu} \partial_{\mu} - m \right) \psi(x) = 0
\end{equation}
where $\psi(x)$ is a four-component spinor, $\gamma^{\mu}$ are the Dirac
matrices and $m$ is the fermion mass.

The fermions are divided into two families called leptons and quarks.
Each family of fermions can be further subdivided into three generations with
particles progressively increasing in mass. The fermions of the Standard Model and
some of their properties are summarised in \TableRef{tab:particles}.

\begin{table}[htbp]
\begin{center}
\begin{tabular}{l l l l l l }
\toprule
& 1st gen. & 2nd gen. & 3rd gen. & $Q$ & colour \\ 
\midrule
\multirow{2}{*}{quarks} 
& \Pup   & \Pstrange & \Ptop & $\nicefrac{+2}{3}e$ & \multirow{2}{*}{$R,G,B$} \\
& \Pdown & \Pcharm   & \Pbottom & $\nicefrac{-1}{3}e$ & \\ 
\multirow{2}{*}{leptons} 
& \Pnue      & \Pnum  & \Pnut & $0$ & \multirow{2}{*}{-} \\
& \Pelectron & \Pmuon & \Ptau & $-1e$ & \\ 
\bottomrule
\end{tabular}
\caption[The fermions of the Standard Model.] {The fermions of the Standard
Model. The charge is in units of elementary charge which is positive for the
proton and negative for the electron.  \label{tab:particles}}
\end{center}
\end{table}

\subsection{Leptons}
Each lepton generation contains a charged lepton and a corresponding light
neutral particle called a neutrino.  The charged leptons are the electron
(\Pelectron), the muon (\Pmuon) and the tau (\Ptauon).  The neutrinos are the
electron neutrino (\Pnue), the muon neutrino (\Pnum) and the tau neutrino
(\Pnut).  For each lepton there is a corresponding anti-lepton with opposite
charge, for example the positively charged positron.  Therefore, in total there
are 12 leptons.  All leptons interact via the weak force, whereas only the
charged leptons interact with the electromagnetic force.

The first generation of leptons is the most familiar containing the electron and
the electron neutrino.

\subsection{Quarks}
There are six flavours of quark arranged into three generations.
The first generation contains the up (\Pup) and the down (\Pdown) quarks, the
second generation contains the charm (\Pcharm) and strange (\Pstrange) quarks
and the third generation, the top (\Ptop) and bottom (\Pbottom) quarks, often
called the truth and beauty quarks.
Unlike leptons, quarks carry fractional charge; within each generation there is
a quark with a charge of $\nicefrac{2}{3}e$ and the other with a charge of
$\nicefrac{-1}{3}e$.

As well as the electric charge, quarks carry an additional
charge called the colour charge. The colour charge allows the quarks to interact
via the strong force in addition to the electromagnetic and the weak forces.
For each quark, there is a corresponding anti-quark which has opposite electric charge
and colour charge. Including the anti-quarks and the possible colour charges
there are a total of 36 quarks.

The first generation of quarks is the most familiar, containing the up and
down-type quarks that are the constituents of the proton and neutron, which with
the electron, form atoms and all normal matter.

\section{Fundamental Forces}
\label{sec:forces}

There are, as far as is known, four fundamental forces of nature, the strong,
weak, electromagnetic and gravitational forces.  \TableRef{tab:forces}
summarises the interactions and the theory that describes them in order of
decreasing strength\footnote{The strength quoted in the table is a rough
approximation as the `strength' will depend on the source and distance of a
force\cite{griffiths2008introduction}}

\begin{table}[htbp]
\begin{center}
\begin{tabular}{ l l l l }
\toprule
Force           & Strength   & Theory   & Mediator \\
\midrule
Strong          & $\mathcal{O}(10^{0})  $ & Quantum Chromodynamics  & Gluon \\
Electromagnetic & $\mathcal{O}(10^{-2}) $ & Quantum Electrodynamics & Photon \\
Weak            & $\mathcal{O}(10^{-13})$ & Electroweak Dynamics    & \PW and \PZ \\
Gravitational   & $\mathcal{O}(10^{-42})$ & General Theory of Relativity & Graviton? \\
\bottomrule
\end{tabular}
\caption[The known four fundamental forces.] {The known four fundamental forces
\cite{griffiths2008introduction}.\label{tab:forces}}
\end{center}
\end{table}

The {Standard Model} describes the strong, weak and electromagnetic interactions. Each
force is mediated by exchange of integer spin intermediate particles called
bosons.
The boson constituents of the {Standard Model} are summarised in
\TableRef{tab:boson}. The final boson included in this table is the Higgs boson
which is described later in this chapter.

\begin{table}[htbp]
\begin{center}
\begin{tabular}{l l c c }
\toprule
Name & Mass ($\GeV$) & Charge \\
\midrule
Photon (\Pphoton) & 0            & 0 \\
\PWpm             & 80.4         & $\pm1e$ \\
\PZ               & 91.2         & 0 \\
gluon (\Pgluon)   & 0            & 0 \\
Higgs (\PHiggs)   & $\approx125$ & 0 \\
\bottomrule
\end{tabular}
\caption[The boson content of the Standard Model.]{The boson content of the Standard Model.\label{tab:boson}}
\end{center}
\end{table}

Gravity is not included in the {SM} as a complete quantum theory of gravity
has not yet been found. However, the gravitational force is very weak when
compared to the other three forces so its contribution to particle interactions
is negligible for the results presented in this thesis.

\subsection{Quantum Electrodynamics}
The classical theory for electromagnetism was formulated by Maxwell over a
century ago, and a quantum theory of electrodynamics was
realised by Tomonaga, Feynman and Schwinger in the 1940s.

The electromagnetic interaction is responsible for the interaction between
charged particles. The interaction is mediated by the massless photon, which means
that the {electromagnetic} interaction is effective over an infinite range.
The fundamental process of Quantum Electrodynamics (QED) is shown in \FigureRef{fig:qed}.
\begin{figure}[htbp]
  \centering
  \includegraphics[width=0.5\textwidth]{qed_process}
  \caption[The elementary process for {QED}.] {The elementary process for {QED}
that all electromagnetic interactions can be reduced to, where $f$ is a charged
fermion and \Pphoton is a photon.}
  \label{fig:qed}
\end{figure}
The fine structure constant, $\alpha$, specifies the strength of the interaction
between charged particles and photon, and is given by
\begin{equation}
\alpha = \frac{e^2}{4 \pi} \approx \frac{1}{137}.
\end{equation}

The {Standard Model} describes all the interactions of known particles in terms
of gauge theories. A gauge theory is a theory that is invariant under a set of
gauge transformations.  In electromagnetism the gauge transformations are
complex phase transformations of the charge particle fields. Requiring local
gauge invariance will require that a massless vector particle be introduced, the
photon, that mediates the electromagnetic interaction.

The Lagrangian density for a free Dirac field, $\psi$, is given by,
\begin{equation}
\mathcal{L} = i \bar{\psi} \gamma_{\mu} \partial^{\mu} \psi - m \bar{\psi}\psi .
\end{equation}
This Lagrangian is invariant under a `global' phase transformation of the
fermion field,
\begin{align}
\psi(x)       &\to \psi^{\prime}(x) = e^{i\omega} \psi(x), \label{eq:global} \\
\bar{\psi}(x) &\to \bar{\psi}^{\prime}(x) = e^{-i\omega}\bar{\psi}(x),
\end{align}
where $\omega$ is a global arbitrary parameter. 
The Lagrangian is said to exhibit global gauge invariance. 

The family of transformations, $R = e^{i \omega}$, forms the
Abelian group $U(1)$ which is the group of all unitary $1\time1$ matrices.
Unitary refers to the property that $U^{-1} = {U}^{\dagger}$.
Abelian refers to the property that all the elements in a
group commute, 
\begin{equation}
e^{-i\omega_1} 
e^{-i\omega_2} 
=
e^{-i\omega_2} 
e^{-i\omega_1} .
\end{equation}

If an infinitesimal group transformation is considered then 
\begin{equation}
\psi(x) 
\to e^{i\omega} \psi(x)
\approx (1+i\omega)\psi(x),
\end{equation}
and it can be shown that the Lagrangian is unchanged by the
transformation\cite{halzen1984quarks}.

Global transformations introduce the problem that by making a transformation, it
is required that every space-time point must `know' about that transformation. It would be
preferable to require invariance under local transformations.
More generally, the \EquationRef{eq:global} can be rewritten as
\begin{equation}
\psi(x) \to e^{i\omega(x)} \psi(x),
\label{eq:local}
\end{equation}
where $\omega$ is now a function of the space time coordinate $x$. This is known
as a local phase transformation. The infinitesimal transformation now becomes,
\begin{equation}
\psi(x) 
\to e^{i\omega(x)} \psi(x)
\approx (1+i\omega(x))\psi(x),
\end{equation}
However, under this transformation the Lagrangian is now no longer
invariant\todo{WHY, EXPLAIN BREIFLY.} 

The gauge invariance can be restored if it is assumed that the fermion field
interacts with a vector field, called a gauge field and denoted $A_{\mu}$, with
an interaction,
\begin{equation}
-e\bar{\psi}\gamma^{\mu}A_{\mu}\psi
\end{equation}
where $A_\mu$ transforms under a gauge transformation as 
\begin{align}
-eA_{\mu} \to & -e\left(A_{\mu}+\delta A_{\mu}(x)\right)\\
           =   & -e A_{\mu} + \partial_{\mu} \omega (x)
\end{align}
and the new Lagrangian then becomes
\begin{align}
\mathcal{L} 
&= \bar{\psi}( i\gamma^{\mu} ( \partial_{\mu} +i e A_{\mu}) - m)\psi \\
&= \bar{\psi}( i\gamma^{\mu} D_{\mu} - m)\psi 
\end{align}
where the covariant derivative has been introduced,
\begin{equation}
D_{\mu} \equiv \partial_{\mu} + i e A_{\mu}.
\label{eq:covar_deriv}
\end{equation}
The vector field, $A_{\mu}$, couples with the electron, and can be identified as
the physical photon field. 

To complete the Lagrangian a kinetic term for the photon field should be added
which is quadratic in the derivative of the field.
However, this term should not break the invariance under gauge transformations.
This is achieved by defining the field strength tensor, $F_{\mu\nu}$,
\begin{equation}
F_{\mu\nu}
= \partial_{\mu} A_{\nu} - \partial_{\nu} A_{\mu}.
\label{eq:fieldstrengthtensor}
\end{equation}
which is gauge invariant. Therefore any term constructed only out of 
 $F_{\mu\nu}$ may be added to the Lagrangian.
A suitable term is $\frac{1}{4} F_{\mu\nu} F^{\mu\nu}$ which is quadratic in the
derivative of the field while remaining gauge invariant,
\begin{equation}
\mathcal{L} = \frac{1}{4} F_{\mu\nu} F^{\mu\nu} + \bar{\psi}( i\gamma^{\mu} D_{\mu} - m)\psi 
\end{equation}

Local gauge invariance has been restored with the introduction of the photon
field. An interesting result is that the photon is required to be massless, as
the introduction of a mass term for the photon, for example a term $\propto
A_{\mu}A^{\mu}$, will break the gauge invariance.

\subsection{Quantum Chromodynamics}
\label{sec:QCD}
\todo[inline]{might be nice to include a section about the history of the
interaction}
The strong force is the force responsible for the interaction between particles
that carry a colour charge, quarks and gluons. Unlike the electric charge, the
colour charge can have three possible values, conventionally called `red',
`green' and `blue' , as well as the corresponding anti-colour charges
`anti-red', `anti-green' and `anti-blue'.
The fundamental quark-gluon process of the strong force is shown in
\FigureRef{fig:qcdquark}.
\begin{figure}[htbp]
  \centering
  \includegraphics[width=0.5\textwidth]{qcd_process}
  \caption{The quark-gluon vertex for {QCD}.}
  \label{fig:qcdquark}
\end{figure}
The strong force is mediated by eight massless gluons which themselves carry
colour charge. This allows the gluons to self-interact which results gluon-gluon
vertexes in addition to the quark-gluon vertex which are shown in
\FigureRef{fig:qcd_gluon}.

\begin{figure}[htbp]
  \centering
  \begin{subfigure}{0.45\textwidth}
    \centering
    \includegraphics[width=\textwidth]{qcd_3gluon}
    \caption{Three-gluon vertex.}
    \label{fig:qcd_3gluon}
  \end{subfigure}
  \begin{subfigure}{0.45\textwidth}
    \centering
    \includegraphics[width=\textwidth]{qcd_4gluon}
    \caption{Four-gluon vertex.}
    \label{fig:qcd_4gluon}
  \end{subfigure}
  \caption{Fundamental gluon-gluon vertices.}\label{fig:qcd_gluon} 
\end{figure}

The coupling constant for the strong force, $\alpha_s$, is given by,
\begin{equation}
\alpha = \frac{g_s^2}{4 \pi} \approx 1
\end{equation}
which is approximately 100 times stronger than the {electromagnetic} force.

An interesting property of the strong force is that as the energy of the interaction
increases the strength of the strong coupling decreases. This is known as
asymptotic freedom; at high-energy experiments quarks appear to behave almost as
free particles. However, at lower energies the coupling increases and quarks
only appear in bound states. This phenomenon is known as quark confinement. 

The strong interaction is described by the theory of {QCD}.
Quantum chromodynamics follows from similar reasoning to {QED}, but with
the Abelian $U(1)$ symmetry group replaced with the non-Abelian $SU(3)$ symmetry
group of transformations on the quark colour fields.

Quarks form colour triplets under $SU(3)$ transformations for each quark
flavour,
\begin{equation}
q_{f} =
\left(\begin{matrix} 
q^{red}_{f} \\
q^{green}_{f} \\
q^{blue}_{f} \\
\end{matrix} \right).
\end{equation}
The local gauge phase transformation under $SU(3)$ is given by,
\begin{equation}
q(x) \to e^{i\alpha_a(x)T_a} q(x),
\end{equation}
where $T_a$ are the generators of the $SU(3)$ group. The transformation breaks
the invariance of the Lagrangian. In a similar way to the U(1) transformation of
QED, the gauge invariance can be restored by introducing the covariant
derivative,
\begin{equation}
D_{\mu} \equiv \partial_{\mu} + i g T_{a} G_{\mu}^{a},
\end{equation}
where eight gauge fields have been introduced, instead of the single field in
QED.  The gauge fields transform as,
\begin{equation}
 G_{\mu}^{a} \to G_{\mu}^{a} 
-\frac{1}{g}\partial_{\mu}\alpha_{a}
-f_{abc}\alpha_{b}G^{c}_{\mu},
\end{equation}
where last additional term has been added is to produce a gauge invariant
Lagrangian due to the non-Abelian gauge transformation of $SU(3)$.

The final Lagrangian for QCD can now be written,
\begin{equation}
\mathcal{L} = 
\bar{q}(i\gamma^{\mu}\partial_{\mu} - m)q -
g \bar{q} \gamma^{\mu} G_{\mu}^{a} - 
\frac{1}{4} G_{\mu\nu}^{a} G^{\mu\nu}_{a},
\end{equation}
where the field strength tensor $G^{\mu\nu}_{a}$ is given by,
\begin{equation}
G^{\mu\nu}_{a} 
= \partial_{\mu} G^{a}_{\nu}
- \partial_{\nu} G^{a}_{\mu}
-g f_{abc} G^{b}_{\mu} G^{c}_{\nu}.
\end{equation}

In a similar way to QED, the Lagrangian for interacting coloured quarks, $q$, and
vector gluons, $G_{\mu}$, results from the simple requirement of local colour
phase invariance of the quark fields. Unlike the QED case, eight gauge fields
are needed due to the three quark colour fields.  Similar to QED, the gluons are
required to  be massless.

The field strength tensor, $G^{\mu\nu}_{a}$, introduces terms that are cubic and
quadratic in $G$. These represent three and four vertex gluon interactions
(\FigureRef{fig:qcd_gluon}) and are a result of the non-Abelian nature of the
$SU(3)$ group.

\subsection{Weak Force}
The weak interaction occurs between all fermions and is the interaction
responsible in the radioactive decay of sub-atomic particles.  Unlike the strong
and electromagnetic forces, the weak force is mediated by the exchange of
massive particles, the \PWpm and \PZ bosons, which causes the weak interaction
to have a very short range.

There are two kinds of weak interactions: charged and neutral, mediated by the
\PW and the \PZ respectively.

The fundamental vertex for the neutral interaction is shown in
\FigureRef{fig:neutral}, where the \PZ interacts with any quark or lepton.
The \PZ can mediate any process that can be mediated by the photon as well as
processes involving neutrinos.
\begin{figure}[htbp]
  \centering
  \includegraphics[width=0.5\textwidth]{weak_neut_process}
  \caption{The fundamental vertex of neutral weak interaction.\label{fig:neutral}}
\end{figure}
The fundamental vertices for the charged current are shown in
\FigureRef{fig:weak_charged}. The charged current has the unique ability to change the
flavour of quarks and leptons by the exchange of \PW bosons. The charged current
will only interact with fermions of the same generation
(\HepProcess{\Pelectron\to\Pnue} but never
\HepProcess{\Pelectron\to\Pnum}).

\begin{figure}[htbp]
  \centering
  \begin{subfigure}{0.45\textwidth}
    \centering
    \includegraphics[width=\textwidth]{weak_charged_lepton_process}
    \caption{Leptons.}
    \label{fig:weak_charged_lepton_process}
  \end{subfigure}
  \begin{subfigure}{0.45\textwidth}
    \centering
    \includegraphics[width=\textwidth]{weak_charged_quark_process}
    \caption{Quarks.}
    \label{fig:weak_charged_quark_process}
  \end{subfigure}
  \caption{The fundamental vertices for charged weak interactions.}
  \label{fig:weak_charged}
\end{figure}

In a similar way to {QCD}, there are coupling of the \PW and \PZ to one
another as shown in \FigureRef{fig:weak_boson}.

\begin{figure}[htbp]
  \centering
  \begin{subfigure}{0.3\textwidth}
    \centering
    \includegraphics[width=\textwidth]{weak_WWZ}
    \caption{\HepProcess{\PW\PW\PZ}.}
    \label{fig:weak_WWZ}
  \end{subfigure}
  \begin{subfigure}{0.3\textwidth}
    \centering
    \includegraphics[width=\textwidth]{weak_WWWW}
    \caption{\HepProcess{\PW\PW\PW\PW}.}
    \label{fig:weak_WWWW}
  \end{subfigure}
  \begin{subfigure}{0.3\textwidth}
    \centering
    \includegraphics[width=\textwidth]{weak_WWZZ}
    \caption{\HepProcess{\PW\PW\PZ\PZ}.}
    \label{fig:weak_WWZZ}
  \end{subfigure}
  \caption{Direct couplings of the \PW and \PZ boson to each other.}
  \label{fig:weak_boson}
\end{figure}

The weak interaction violates parity ($P$) and charge conjugation ($C$). A
well known example of parity violation in weak interactions is the Wu experiment
\cite{wu1957experimental}.
In the experiment, the $\beta$ decay of nuclei (Cobalt-60) polarised by an
external magnetic was studied, 
\begin{equation}
\begin{matrix}
\Rightarrow\Rightarrow \\
^{60}\mathrm{Co} \\
~   
\end{matrix}
\to
{^{60}\mathrm{Ni}}
+
\begin{matrix}
\Rightarrow \\
\Pelectron \\
\leftarrow 
\end{matrix}
+
\begin{matrix}
\Rightarrow \\
\APnue \\
\rightarrow 
\end{matrix},
\end{equation}
where the double arrows represent the spins and the single arrows represent the
direction of the particle produced in the decay.
The Cobalt-60 nuclei were aligned to the external magnetic field. By
conservation of angular momentum, the neutrino and electron spins must be
parallel and aligned with the magnetic field. By momentum conservation, the
electron and neutrino must be produced in opposite directions which means that
the electron and neutrino must have opposite helicity.  By changing the direction
of the magnetic field, the system undergoes a parity transformation
\begin{equation}
\begin{matrix}
\Leftarrow\Leftarrow \\
^{60}\mathrm{Co} \\
~   
\end{matrix}
\to
{^{60}\mathrm{Ni}}
+
\begin{matrix}
\Leftarrow \\
\Pelectron \\
\leftarrow 
\end{matrix}
+
\begin{matrix}
\Leftarrow \\
\APnue \\
\rightarrow 
\end{matrix}.
\end{equation}
If parity was conserved then the electron would have no preference in direction.
However, what was seen was that the electrons were preferentially emitted in the
opposite direction to their spin. The weak interaction exhibits maximal parity
violation.
More generally, the \PWpm boson is unique in that it only interacts with left-handed
particles (or right-handed antiparticles).

\subsection{Electroweak Interaction}
The electromagnetic and weak interactions can be described by a unified theory
known as the electroweak theory.
The gauge group for the electroweak theory is given by,
\begin{equation}
U(1)_{Y} \times SU(2)_{L} .
\end{equation}
The $L$ subscript on $SU(2)$ indicates that the weak force couples to the left
handed particles. 
The $Y$ indicates that the $U(1)$ group is not the gauge
group of QED but is that of the hypercharge which is connected to the electric
charge ($Q$) and the charge associated with the weak interaction, called weak
isospin ($I$), by the Gell-Mann-Nishijima relation,
\begin{equation}
Q = I_{3}+ \frac{1}{2}Y.
\end{equation}

The matter content of the theory are written as doublets or singlets. 
For the case of the leptons $l_{L}$ is a left
handed doublet,
\begin{equation}
l_{L} = \left( \begin{matrix} \nu \\ e \end{matrix} \right)_{L},
\end{equation}
and $e_{R}$ is a right-handed singlet.
For the case of the quarks, $q_{L}$ is a left-handed doublet,
\begin{equation}
q_{L} = \left( \begin{matrix} u\\ d \end{matrix} \right)_{L},
\end{equation}
and there are two right-handed singlets, the $u_{R}$ and the $d_{R}$.
In this description the left-handed fermions form isospin doublets and the right
handed fermions are singlets. Therefore, under $SU(2)_L$ gauge transformations,
\begin{align}
e_{R} &\to e_{R}^{\prime} = e_{R}\\
l_{L} &\to l_{L}^{\prime} = e^{-i \omega^{a} {T}^{a} }l_{L}
\end{align}
where $T^{a}$ are the generators of $SU(2)_L$.  The $SU(2)_L$ singlets are invariant
so do not couple with the corresponding gauge bosons.

The matter fields transform under the $U(1)_Y$ gauge transformations as,
\begin{equation}
\psi \to \psi^{\prime} = e^{-i\omega Y(\psi)}\psi
\end{equation}
where Y is the hypercharge of the particle.
\begin{equation}
Y(l_{L}) = -\frac{1}{2}, \qquad Y(e_{R}) = -1,
\end{equation}
\begin{equation}
Y(q_{L}) =  \frac{1}{6}, \qquad Y(u_{R}) =  \frac{2}{3}, \qquad Y(d_{R}) = -\frac{1}{3},
\end{equation}

The Lagrangian for the electroweak interaction can be written as the sum of the
gauge boson and the fermion parts,
\begin{equation}
\mathcal{L}_{electroweak} = 
\mathcal{L}_{fermion}
+ \mathcal{L}_{gauge}.
\end{equation}

The fermion term has a lepton and a quark part,
\begin{equation}
\mathcal{L}_{fermion} =
 \mathcal{L}_{lepton}
+ \mathcal{L}_{quark}.
\end{equation}
The quark and lepton Lagrangians are given by,
\begin{align*}
\mathcal{L}_{lepton} &= 
\bar{l}_{L} i \gamma^{\mu} \mathbf{D}_{\mu} l_{L} +
\bar{e}_{R} i \gamma^{\mu} D_{\mu} e_{R}, \\
\mathcal{L}_{quark} &= 
\bar{q}_{L} i \gamma^{\mu} \mathbf{D}_{\mu} q_{L} +
\bar{u}_{R} i \gamma^{\mu} D_{\mu} u_{R} +
\bar{d}_{R} i \gamma^{\mu} D_{\mu} d_{R},
\end{align*}
where the covariant derivative has again been introduced.
The covariant derivative depends on the fermion field on which it acts. The covariant derivative
for left-handed fermion, for example, is given by,
\begin{equation}
\mathbf{D}_\mu 
= \partial_\mu 
+ ig{T}^{a}W_{\mu}^{a}
+ ig^{\prime}Y(l^{L})B_{\mu},
\end{equation}
whereas the covariant derivative for the right-handed fermion, for example a
down quark, $d_R$,
\begin{equation}
D_\mu = \partial_\mu + ig^{\prime}Y(d^{R})B_{\mu},
\end{equation}
where $g^\prime$ and $g$ and are the two coupling constants,
${T}^{a}$ are the three generators of $SU(2)_L$ group and $W^{a}_{\mu}$ are
the three gauge fields in the theory.

The gauge part of the Lagrangian contains the kinetic terms and self interaction
terms for the gauge fields,
\begin{equation}
\mathcal{L}_{gauge} = 
- \frac{1}{4} B_{\mu\nu} B^{\mu\nu}
- \frac{1}{4} W^{a}_{\mu\nu} W^{a~\mu\nu}
\end{equation}
where the first term contains the hypercharge field strength and the second term 
contains the $SU(2)_L$ field strength where the index, $a$, runs from 1 to 3.
\begin{align*}
B^{\mu\nu}     &= \partial^{\mu} B^{\nu} - \partial^{\nu} B^{\mu},\\
W_{a}^{\mu\nu} &= \partial^{\mu} W_{a}^{\nu} - \partial^{\nu} W_{a}^{\mu} 
                + g \epsilon_{abc} W_{b}^{\mu} W_{c}^{\nu}.
\end{align*}

Four gauge fields have been introduced, 
three fields, ${W}^{a}_{\mu}$ , corresponding to the $SU(2)_{L}$
group and a single gauge field, $B_{\mu}$, corresponding to the $U(1)_{Y}$ group.
The physical $\PWp$ and $\PWm$ bosons are superpositions of the $W^{1}_{\mu}$
and $W^{2}_{\mu}$ gauge fields,
\begin{equation}
W^{\pm}_{\mu} = \frac{1}{\sqrt{2}} \left(W^{1}_{\mu} \mp W^{2}_{\mu}\right),
\label{eq:wgauge}
\end{equation}
and the photon and Z boson are combinations of the $B_{\mu}$ and $W^{3}_{\mu}$
gauge fields,
\begin{equation}
\left( \begin{matrix} A_{\mu}\\ Z_{\mu}\end{matrix}\right) =
\left( \begin{matrix} \cos\theta_{W} && \sin\theta_{W} \\  
                      -\sin\theta_{W} && \cos\theta_{W} \end{matrix}\right) 
\left( \begin{matrix} B_{\mu}\\ W^{3}_{\mu}\end{matrix}\right) ,
\label{eq:bgauge}
\end{equation}
where $\theta_{W}$ is the Weinberg angle which is related to the coupling
constants by
\begin{align*}
\sin\theta_{W} &= \frac{g^{\prime}}{\sqrt{g^{2}+{g^{\prime}}^{2}}},\\
\cos\theta_{W} &= \frac{g}{\sqrt{g^{2}+{g^{\prime}}^{2}}}.
\end{align*}

\subsubsection{Higgs Mechanism}

The Lagrangian, as it has been written so far, does not include terms for the
mass of any of the particles.  Adding in mass terms by hand will break the gauge
invariance rendering the theory meaningless.  \todo[inline]{I could include an
example of why adding the mass terms in by hand will break gauge invariance}

The symmetry needs to be broken in some natural way.  Spontaneous symmetry
breaking (SSB) is a method to break the symmetry by requiring that the Lagrangian of a
system remains invariant under a transformation, but the ground state is not
invariant.

An example of spontaneous symmetry breaking is a point mass in a potential,
\begin{equation}
V(r) = \mu^{2} \vec{r} \cdot \vec{r} + \lambda ( \vec{r} \cdot \vec{r} )^{2}
\end{equation}
where $\lambda$ is positive. This potential is radially symmetric. 
A point mass sits at $\vec{r}=0$. If $\mu^{2}>0$, as shown in
\FigureRef{fig:higgs_pot_mup} then $\vec{r}=0$ is the ground state and
the mass will remain at this point.
If $\mu^{2}<0$ then the potential will look like that given in
\FigureRef{fig:higgs_pot_mum}. The system remains symmetric, but $\vec{r}=0$ is no longer
the ground state. To fall to the ground state the mass has to ``choose'' a
direction to fall. The choice will break the symmetry of the system; the
potential remains symmetric, but the ground state is not. This is an example of
{SSB}.

\begin{figure}[htbp]
  \centering
  \begin{subfigure}{0.45\textwidth}
    \centering
    \includegraphics[width=\textwidth]{higgs_pot_mup}
    \caption{$\mu^{2}>0$.}
    \label{fig:higgs_pot_mup}
  \end{subfigure}
  \begin{subfigure}{0.45\textwidth}
    \centering
    \includegraphics[width=\textwidth]{higgs_pot_mum}
    \caption{$\mu^{2}<0$.}
    \label{fig:higgs_pot_mum}
  \end{subfigure}
  \caption[The potential $ V(r) = \mu^{2} \vec{r} \cdot \vec{r} + \lambda (
\vec{r} \cdot \vec{r} )^{2}$.] {The potential $ V(r) = \mu^{2} \vec{r} \cdot
\vec{r} + \lambda ( \vec{r} \cdot \vec{r} )^{2}$. For simplicity, the potential
is shown for a single component of $\vec{r}$.}
  \label{fig:higgs_pot}
\end{figure}

The application of {SSB} to the {SM} was studied by Higgs, Englert, Brout
and others \cite{higgs1981broken, englert1964broken, guralnik1964global}.
The result was a mechanism that spontaneously breaks the
 $SU(2)_{L} \times U(1)_{Y}$ symmetry called the Higgs mechanism\footnote{or the
Brout-Englert-Higgs mechanism, or even the
Englert-Brout-Higgs-Guralnik-Hagen-Kibble mechanism.}.

The Higgs mechanism introduces four real scalar fields, that can be arranged in a
doublet under $SU(2)$,
\begin{equation}
\Phi = \left( \begin{matrix} \phi^{+} \\ \phi^{0} \end{matrix} \right),
\end{equation}
where,
\begin{align*}
\phi^{+} &=\frac{1}{\sqrt{2}} (\phi_{1} + i \phi_{2}),\\
\phi^{0} &=\frac{1}{\sqrt{2}} (\phi_{3} + i \phi_{4}).
\end{align*}
The additional scalar part of the Lagrangian is,
\begin{equation}
\mathcal{L}_{scalar} = 
\left(D^{\mu}\Phi\right) \left(D_{\mu}\Phi\right) - V(\Phi),
\end{equation}
where the potential is,
\begin{equation}
V(\Phi) = 
\mu^{2}\Phi^{\dagger}\Phi + 
\lambda^{2} \left( \Phi^{\dagger} \Phi \right)^{2},
\end{equation}
where $\lambda$ and $\mu$ are free parameters. This Lagrangian is invariant
under $SU(2)_{L} \times U(1)_{Y}$ transformations.
By choosing  $\lambda>0$ and
$\mu^{2}<0$ there are minima at
\begin{equation}
\Phi^{\dagger} \Phi = \frac{- \mu^{2}}{2 \lambda}.
\end{equation}
The potential at $\Phi=0$ is unstable to small perturbations, and will fall
to a lower energy ground state. 
The ground state does not have the same symmetry as the Lagrangian; by
selecting a minimum the symmetry has become broken. An example choice of a minimum
could be,
\begin{equation}
\phi_{1} = \phi_{2} = \phi_{4} = 0,
\end{equation}
and
\begin{equation}
\phi_{3} = \frac{-\mu^{2}}{\lambda} \equiv v^{2}.
\end{equation}
which results in a non-zero vacuum-expectation value,
\begin{equation}
<0|\Phi|0> = \frac{1}{\sqrt{2}}\left(\begin{matrix}0\\v\end{matrix}\right),
\end{equation}
$\Phi$ can then be expanded around the vacuum minimum in terms of the four real
fields $H$ and $\phi_1$, $\phi_2$ and $\phi_3$.
\begin{equation}
\label{eq:phi}
\Phi = 
\frac{1}{\sqrt{2}}
e^{i\nicefrac{\sigma}{2}\phi_a}
\left(\begin{matrix}0\\v+H\end{matrix}\right)
\approx 
\frac{1}{\sqrt{2}}
\left(\begin{matrix}0\\v+H\end{matrix}\right)
\end{equation}
where the $SU(2)$ invariance of the Lagrangian allows for the choice of $\phi_i$
to be zero\cite{halzen1984quarks,ral}.
\todo{EXPLAIN MORE}

The Lagrangian can be now written in terms of the $H$ field. The kinetic
term,\cite{ral}
\begin{align}
\left(D_{\mu}\Phi\right) \left(D^{\mu}\Phi\right) 
&= \frac{1}{2} \left(\partial_{\mu}H\right) \left(\partial^{\mu}H\right) 
         + \frac{g^{2}v^{2}}{4} W_{\mu}^{+} W^{-~\mu} \nonumber \\
&\qquad{}+ \frac{g^{2}v^{2}}{8 \cos^{2}\theta_{W}} Z_{\mu} Z^{\mu} + 0 A_{\mu} A^{\mu} \nonumber \\
&\qquad{}+ \text{~ interaction terms},
\end{align}
now includes mass terms for the gauge bosons,
\begin{equation}
M_{W} = \frac{1}{2}gv, \qquad 
M_{Z} = \frac{1}{2}\frac{gv}{8\cos^{2}\theta_{W}} .
\end{equation}
while the photon remains massless, $M_{A}=0$.

The mechanism also introduces an additional boson with a mass
$\sqrt{-2\mu^{2}}$, called the Higgs boson.

\subsubsection{Fermion Masses and Yukawa Couplings}

Another feature of the Higgs mechanism is that it also provides a way to
introduce mass terms for the fermions, in a gauge invariant way, via the Yukawa
coupling between the leptons with the Higgs field. The Lagrangian for this
interaction for the electron can be written as, 
\begin{equation}
\mathcal{L}_{yukawa} = -Y_{e}\bar{l}_L\Phi e_R + h.c. \ ,
\end{equation}
where $Y_{e}$ is the coupling to the Higgs field known as the Yukawa coupling
and h.c. stands for hermitian conjugate. On substitution of $\Phi$ (from
\EquationRef{eq:phi}), the Lagrangian for the Yukawa couplings becomes,
\begin{equation}
\mathcal{L}_{yukawa} = 
-\frac{Y_{e}}{\sqrt{2}} v
(\bar{e}_L e_R + \bar{e}_R e_L)
-\frac{Y_{e}}{\sqrt{2}}
(\bar{e}_L e_R + \bar{e}_R e_L)H
\end{equation}
and if $G_e$ is chosen such that,
\begin{equation}
m_{e} = \frac{G_{e}v}{\sqrt{2}}
\end{equation}
then the Lagrangian simplifies to,
\begin{equation}
\mathcal{L}_{yukawa} = 
- m_e \bar{e}e
- \frac{m_e}{v} \bar{e}e H
\end{equation}
which represents a mass term for the electron and a coupling of the electron to
the Higgs field proportional to the mass of the electron.
The quark masses can also be generated in a similar way \cite{halzen1984quarks,ral}.

\subsubsection{Higgs Boson Observation}
On the fourth of July 2012, the two LHC experiments ATLAS and CMS both announced
the discovery of a new boson that is compatible with the Standard Model Higgs
boson \cite{aad2012observation,chatrchyan2012observation}. 

\begin{figure}[htbp]
  \centering
  \includegraphics[width=0.97\textwidth]{higgs}
  \caption[The diphoton invariant mass distribution observed in CMS data.] {The
diphoton invariant mass distribution observed in CMS data.  Each event has been
categorised and then weighted by the S/(S + B) value of its category. The inset
is the unweighted invariant mass distribution. A peak in the region of
\unit{125}{\GeV} can clearly be seen. From \cite{chatrchyan2012observation}. }
  \label{fig:hgg}
\end{figure}

The Higgs search was performed in many different decay channels over a wide
mass range from 100 up to \unit{600}{\GeV}.
In the low-mass range, from 100 up to \unit{160}{\GeV}, a significant channel is the
diphoton channel, where the Higgs decays to a pair of photons via a top or \PW loop.
\FigureRef{fig:hgg} shows the diphoton invariant mass distribution measured in CMS.
A peak in the region of \unit{125}{\GeV} can clearly be seen. 
After combining the observations in each of the channels, an excess of events
was observed above what would be expected from only background events, with a
local significance of \unit{5.0}{\sigma} at a mass near 125 GeV.

