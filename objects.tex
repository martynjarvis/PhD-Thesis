\chapter{Physics Objects}

\todo[inline]{introduction to reconstruction of physics objects.}

\begin{figure}[htb]
  \centering
  %\includegraphics[width=0.5\textwidth]{placeholder}
  \missingfigure{Add standard plot of different particles traversing the
detector.}
  \caption{The path of different particles through a cross section of the CMS detector.}
  \label{phob:crosssec}
\end{figure}

\section{Electrons}
\todo[inline]{Why are electrons important phsyics objects}
\todo[inline]{Describe path that electrons make through the detector.}

\subsection{Triggering}

\subsection{Reconstruction}

\subsubsection{Electron Clustering}
As an elctron traveses the tracker region of the detector it will radiate
bremsstrahlung photons. Instead of a single deposit of energy in the ECAL, the
energy is spread in the azimuthal angle, $\phi$.
\FigureRef{reco:brem} shows the fraction of energy radiated by bremsstrahlung for
electrons of energy $10$, $30$ and \unit{$50$}{\GeV}.

\begin{figure}[htb]
  \centering
  %\includegraphics[width=0.5\textwidth]{placeholder}
  \missingfigure{Fraction of energy radiated by bremsstrahlung from Electron
reconstruction in CMS}
  \caption{Fraction of electron energy, $E^{e}$, radiated away as bremsstrahlung
photons, $\Sum E_{brem}^{\gamma}$ for electrons of energy $10$, $30$ and
\unit{$50$}{\GeV}. From \cite{}.}
  \label{reco:brem}
\end{figure}




\subsubsection{Electron Track Reconstruction}
\todo[inline]{Rewrite this section}
The electron track reconstruction proceeds by matching a track found in the CMS
tracker to an energy cluster in the ECAL.
Due to Bremsstrahlung radiation, electron tracks may suffer from sudden changes
in the radius of curvature. 
Therefore, electron track reconstructions uses a Gaussian-Sum Filter (GSF) fit
to allow for energy losses. 
The first step in the GSF electron reconstruction is to construct
superclusters.
A seed cluster is formed from the closest ECAL cluster to the extrapolated GSF
track. 
Potential Bremsstrahlung photons are identified by creating a straight line in
the ECAL from where the track enters the ECAL and the tangent to the initial
direction of the track. 
Any unlinked ECAL clusters that can be linked to this line are then identified
as Bremsstrahlung photons and their energy is added to the total ECAL
supercluster energy.\cite{eleReco}

From the supercluster, seeds in the pixel detector are found where two hits in
the detector are compatible with a given beam spot. The seed is then used to
build a trajectory by extrapolating the seed to the next layers and looking for
compatible hits using a Bethe Heitler modelling of the electron losses and a
GSF in the forward fit. This is repeated for each new layer until the last
layer or until no hits can be found in two consecutive layers.\cite{eleReco}

\subsection{Electron Identification}
\todo[inline]{Describe eectron id variables}

\subsection{Electron Charge Measurement}
\todo[inline]{Describe electron charge assigning methods}

\section{Missing Energy}
\todo[inline]{Describe how MET is measured in CMS} 

\subsection{Particle Flow at CMS}
\todo[inline]{Tie particle flow in with rest of chapter.} 

New physics will manifest itself in CMS through signatures involving standard
model particles. Important signatures for many new phenomena include high \Pt\
jets, missing transverse energy (\met), jets containing \Pbottom quarks and
hadronically decaying tau leptons. To study these signatures it is important to
reconstruct and identify all particles in events as accurately as possible. The
particle-flow event reconstruction attempts to reconstruct and identify all
stable particles in an event by combining information from all CMS
sub-detectors. The particle reconstruction and identification starts with
collecting information from each subdetector to form elements such as tracks
and energy clusters in the calorimeters. These basic 'elements' are then
combined to form blocks which are then interpreted in terms of particles by the
particle flow algorithm. A list of individual particles is then returned from
the algorithm which can be used to study the event in greater detail by,
amongst other things, building jets, tagging b quarks and calculating missing
transverse energy.\cite{PF}

The first step of the particle-flow reconstruction algorithm is to collect the
fundamental elements. The elements consist of charged particles tracks from the
tracker, clusters of energy deposition in the calorimeters and muon tracks.
These elements need to be identified with a high efficiency and low fake rate
since the particle reconstruction depends on these basic elements and
misidentified elements could lead to missing or double counted
particles.\cite{PF}

As a particle traverses the detector it may interact with many CMS subdetectors
creating several particle-flow elements. A link algorithm is used to connect
the elements together to form blocks that typically contain 1, 2 or 3 elements.
The algorithm returns a distance between the elements as a measure of the
quality of the link. The final step of the particle flow algorithm is to
reconstruct and identify particles from each block of linked elements.\cite{PF}

Once the event has been fully reconstructed with the particle flow technique
the missing transverse energy (\met) in the event can be easily computed by
summing up the transverse momentum of all the reconstructed particles.\cite{PF}

