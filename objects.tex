\chapter{Physics Objects}

\todo[inline]{introduction to reconstruction of physics objects.}

\begin{figure}[htb]
  \centering
  %\includegraphics[width=0.5\textwidth]{placeholder}
  \missingfigure{Add standard plot of different particles traversing the
detector.}
  \caption{The path of different particles through a cross section of the CMS detector.}
  \label{phob:crosssec}
\end{figure}

\section{Electrons}
\todo[inline]{Why are electrons important phsyics objects}
\todo[inline]{Describe path that electrons make through the detector.}

\subsection{Triggering}

\subsection{Reconstruction}
Electrons are reconstructed in CMS using information from the pixel detector,
silicon strip tracker and the ECAL.

\subsubsection{Electron Clustering}
As electrons traverse the CMS tracker, the strong magnetic field causes the path
of the electrons to be curved in the azimuthal, $\phi$, direction. The electrons
radiate bremsstrahlung photons, so that when the electron energy reaches the
ECAL it is spread over a narrow strip in the phi direction.
\FigureRef{reco:brem} shows the fraction of energy radiated by bremsstrahlung for
electrons of energy $10$, $30$ and \unit{$50$}{\GeV}.

\begin{figure}[htb]
  \centering
  %\includegraphics[width=0.5\textwidth]{placeholder}
  \missingfigure{Fraction of energy radiated by bremsstrahlung from Electron
reconstruction in CMS}
  \caption{Fraction of electron energy, $E^{e}$, radiated away as bremsstrahlung
photons, $\Sum E_{brem}^{\gamma}$ for electrons of energy }%$10$, $30$ and \unit{$50$}{\GeV}. From \cite{}.}
  \label{reco:brem}
\end{figure}

To measure the electron energy, including the bremsstrahlung photons, the
seperated deposits of energy need to be collected, using super-clustering
algorithms. 

In the barrel a ``hybrid'' algorithm is used. The hybrid algorithm proceeds by
identifying several hot crystals, with energies above a certain threshold, that
will act as seeds. The algorithm then forms $1\times3$ or $1\times5$ crystal
``dominos'', centered on the seed crystal, depending on the energy within the
domino. The dominos are then collected together in the $\phi$ direction, up to
an extension of \unit{0.3}{\rad}, to form clusters of clusters. This is
demonstrated in \FigureRef{fig:hybrid}.

\begin{figure}[htb]
  \centering
  %\includegraphics[width=0.5\textwidth]{placeholder}
  \missingfigure{hybrid clustering algorithm}
  \caption{Demonstration of the clustering of dominos in the hybrid algorithm.}
  \label{reco:hybrid}
\end{figure}

A ``multi5x5'' algorithm is used in the ECAL endcaps. Energy is collected in
$5\times5$ matrices, which are then collected together if their position lies on
a narrow $\phi$ road to form superclusters.

\subsubsection{Electron Seeding}
The superclusters are then used to select seeds for the track reconstruction.
Starting with a supercluster that passes a \pt and a hadronic veto, the
trajectory of the electron is propogated back through the magnetic field and
matched to the trajectory seeds, pairs or triplets of hits in the inner tracker.
If the trajectory seeds fall within a window of the supercluster path under
either charge hypothesis, they are selected and used to seed the track
reconstruction.
The ECAL driven seeding is comlemented by a tracker driven seeding algorithm.
This starts with high purity tracks and extrapolating them outwards to the ECAL.
This is effective for lower \pt electrons.

Seeds from both of the algorithms are collected and merged in to a single
collection, which is then used to seed the electron track reconstruction.

\subsubsection{Electron Track Reconstruction}
\todo[inline]{Rewrite this section}
The track reconstruction is based on a combinatorial Kalman filter,\todo{find a
citation for this} with the electron energy losses described using Bethe Heitler
modelling.


\subsection{Electron Identification}
\todo[inline]{Describe eectron id variables}

\subsection{Electron Charge Measurement}
\todo[inline]{Describe electron charge assigning methods}

\section{Missing Energy}
\todo[inline]{Describe how MET is measured in CMS} 

\subsection{Particle Flow at CMS}
\todo[inline]{Tie particle flow in with rest of chapter.} 

New physics will manifest itself in CMS through signatures involving standard
model particles. Important signatures for many new phenomena include high \Pt\
jets, missing transverse energy (\met), jets containing \Pbottom quarks and
hadronically decaying tau leptons. To study these signatures it is important to
reconstruct and identify all particles in events as accurately as possible. The
particle-flow event reconstruction attempts to reconstruct and identify all
stable particles in an event by combining information from all CMS
sub-detectors. The particle reconstruction and identification starts with
collecting information from each subdetector to form elements such as tracks
and energy clusters in the calorimeters. These basic 'elements' are then
combined to form blocks which are then interpreted in terms of particles by the
particle flow algorithm. A list of individual particles is then returned from
the algorithm which can be used to study the event in greater detail by,
amongst other things, building jets, tagging b quarks and calculating missing
transverse energy.\cite{PF}

The first step of the particle-flow reconstruction algorithm is to collect the
fundamental elements. The elements consist of charged particles tracks from the
tracker, clusters of energy deposition in the calorimeters and muon tracks.
These elements need to be identified with a high efficiency and low fake rate
since the particle reconstruction depends on these basic elements and
misidentified elements could lead to missing or double counted
particles.\cite{PF}

As a particle traverses the detector it may interact with many CMS subdetectors
creating several particle-flow elements. A link algorithm is used to connect
the elements together to form blocks that typically contain 1, 2 or 3 elements.
The algorithm returns a distance between the elements as a measure of the
quality of the link. The final step of the particle flow algorithm is to
reconstruct and identify particles from each block of linked elements.\cite{PF}

Once the event has been fully reconstructed with the particle flow technique
the missing transverse energy (\met) in the event can be easily computed by
summing up the transverse momentum of all the reconstructed particles.\cite{PF}

