
\chapter{W Bosons at the LHC}

\section{Introduction}
\section{W Boson Production}
\section{W Charge Asymmetry}

A particularly interesting observable involving \PW bosons produced in hadron
collisions is an asymmetry in the production of \PWp and \PWm bosons as a
function of rapidity.
In proton-antiproton colliders, such as Fermilab's Tevatron,  \PWp and \PWm
bosons are produced at equal rates but \PWp are produced more in the direction
on the incoming proton where as the \PWm are produced more in the opposite
direction.
In proton-proton colliders, such as the LHC, the \PWp is produced in greater
rates than the \PWm. Additionally the \PWp are produced at larger rapidities
where as the \PWm bosons are produced more centrally. \cite{phenom}

The difference in production rate of \PWpm boson is mainly due to the quark
content in the proton and the rapidity dependence is due to the \Pup quarks on
average carrying a greater fraction of the proton momentum that \Pdown quarks.
Therefore a measurement of the charge asymmetry can provide important
information on the parton density functions of the proton.\cite{phenom,pdf}.

At the LHC it has been proposed \cite{kom} that an asymmetry measurement could
also be used to measure the contribution of New Physics, since in the Standard
Model, \begin{equation}
\sigma_{SM}(\PWp + jets) \approx 1.3 \times \sigma_{SM}(\PWm + jets)
\end{equation}
where as typically in many New Physics theories and models predict
\begin{equation}
\sigma_{NP}(\PWp + jets) = \sigma_{NP}(\PWm + jets)
\end{equation}
examples include the pair production of gluinos in supersymmetry where the
gluinos decay to one charged lepton, missing energy and jets.\cite{kom}

\section{Electron Charge Asymmetry}

\PW bosons are identified by their decay to a lepton plus neutrino, however at
hadronic colliders the neutrino longitudinal momentum is unmeasured meaning
that the \PW rapidity is unknown. Instead what is studied is the lepton charge
asymmetry.\cite{phenom} The theoretical asymmetry is given by equation
\ref{eq:AsymThe}.

\begin{equation}
A_{the}(\eta)=\frac{  \frac{d\sigma}{d\eta}(\Wpenu) -
\frac{d\sigma}{d\eta}(\Wmenu) }{ \frac{d\sigma}{d\eta}(\Wpenu) +
\frac{d\sigma}{d\eta}(\Wmenu) }
\label{eq:AsymThe}
\end{equation} 

The experimentally measured asymmetry is given by equation
\ref{eq:AsymExp}.\cite{kom}
 
\begin{equation}
A_{exp}(\eta)=\frac{  \frac{dN}{d\eta}(\Pelectron) -
\frac{dN}{d\eta}(\APelectron) }{ \frac{dN}{d\eta}(\Pelectron) +
\frac{dN}{d\eta}(\APelectron) }
\label{eq:AsymExp}
\end{equation} 

The two measurements are related by equation \ref{eq:NumEve} which takes in to
account the experimental effects such as the luminosity (${\cal L}$), high
level trigger ($\epsilon_{HLT}$), offline efficiency ($ \epsilon_{off}$) and
the acceptance ($\epsilon_{acc}$).

\begin{equation}
\frac{dN}{d\eta} = {\cal L } \frac{d\sigma}{d\eta}  \epsilon_{HLT}
\epsilon_{off} \epsilon_{acc}
\label{eq:NumEve}
\end{equation} 

Since the asymmetry is a ratio, the luminosity high level trigger and the
offline efficiency cancel.\cite{me} The acceptance can not be cancelled
however, since it is a function of rapidity and momentum and electrons and
positrons will have different rapidity and momentum distributions. The
experimentally measured asymmetry can therefore be related to the theoretical
asymmetry by correcting for the acceptance.\cite{me}

\begin{align} 
A_{exp}(\eta) &= \frac{ \frac{dN}{d\eta}(\Pelectron) -
\frac{dN}{d\eta}(\APelectron) }{ \frac{dN}{d\eta}(\Pelectron) +
\frac{dN}{d\eta}(\APelectron) }\\   
              &= \frac{ \frac{d\sigma}{d\eta}(\Wpenu) -
\frac{\epsilon^{-}_{acc}}{\epsilon^{+}_{acc}} \frac{d\sigma}{d\eta}(\Wmenu) }{
\frac{d\sigma}{d\eta}(\Wpenu) + \frac{\epsilon^{-}_{acc}}{\epsilon^{+}_{acc}}
\frac{d\sigma}{d\eta}(\Wmenu) }
\label{eq:AsymExpCorr}
\end{align}

