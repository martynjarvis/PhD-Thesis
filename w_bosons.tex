\chapter{W Bosons at the LHC}

\section{Introduction}

\PW boson production is an important process for physics studies at the LHC.

At the \ac{LHC}, \PW bosons are produced at a high rate while offering a clean
experimental signal with a final state consisting of, in the case of a
leptonically decaying \PW, a single high \PT lepton with a large ammount of
missing transverse energy due to the neutrino in the event.

\PW provide an important test of the \ac{SM} and accurate measurements of the
\ac{SM} parameters. 

The production of \PW, through the \ac{DY} process, provides important
information on the interacting partons within the colliding hadrons.

\section{W Boson Production}

The dominant production process for a \PW boson at a hadron-hadron collider is
shown in \EquationRef{wbos:dyproc}

\begin{equation}
  h_1(P_1) + h_2(P_2)
  \to 
  \PW + X
  \to
  \Pe \Pnue + X
  \label{wbos:dyproc}
\end{equation}

The \PW boson is produced by the collision of the two incoming hadrons $h_1$
and $h_2$ with momenta $p_1$ and $p_2$. The \PW then decays leptonically to an
electron and its corresonding neutrino. $X$ represents the accompanying final
state.


%A tree-level feynman diagram representing this process is shown in
%\FigureRef{wbos:feynman}.

%\begin{figure}[htb]
  %\centering
  %%\includegraphics[width=0.5\textwidth]{placeholder}
  %\caption{Tree-level diagram of a W boson production at a hadron-hadron
  %collider.}
  %\label{wbos:feynman}
%\end{figure}

The differential crossection at a fixed rapidity of the \PW boson is denoted by
$\nicefrac{d\sigma_{(\HepProcess{h_1 h_2 \to \PWpm}})}{dy_W}$ where $y_W$ is
the rapidity of the \PW boson. 
At hadron colliders the full cross section is a convolution of the crosssection
at the parton level, and the \acp{PDF}.

\begin{multline}
  d\sigma_{(\HepProcess{h_1 h_2 \to \PWpm})}(p_1,p_2) = \\
  \sum\limits_{a,b}
  \int_0^1 \! \mathrm{d} x_1 
  \int_0^1 \! \mathrm{d} x_2 
  f_a^{h_1}(x_1,Q^2)
  f_b^{h_2}(x_2,Q^2) 
  d\hat{\sigma}_{(\HepProcess{a b \to \PWpm})}(x_1 p_1, x_2 p_2; Q^2)
  \label{wbos:xsec}
\end{multline}


$d\hat{\sigma}_{(\HepProcess{a b \to \PWpm})}(x_1 p_1, x_2 p_2; Q^2)$
represents the partonic sub-process crosssection. At \ac{LO} the process is
simply
\begin{equation}
  \HepProcess{\Pup + \APdown \to \PWp \to \Pleptonplus \Pnulepton} \label{wbos:wprod} \\
  \HepProcess{\APup + \Pdown \to \PWm \to \Pleptonminus \APnulepton}
\end{equation}

In \EquationRef{wbos:xsec}, $f_a^{h}(x,Q^2)$ represent the \acp{PDF}.
The \ac{PDF} represents the number density of parton $a$ that has a momentum
fraction $x+\mathrm{d}x$ of the colliding hadron $h$.

\acp{PDF} are obtained from global fits to experimental data. %TODO more?

\begin{figure}[htb]
  \centering
  %\includegraphics[width=0.5\textwidth]{placeholder}
  \caption{Proton PDF at $Q^2\approx M_{\PW}^2$.}
  \label{wbos:pdf}
\end{figure}

\FigureRef{wbos:pdf} shows the proton PDF at $Q^2\approx M_{\PW}^2$. 
The \ac{PDF} for the up-type quark and the down-type quark differ, the
$\Pup(x)$ is greater than $\Pdown(x)$ due to the valence quark content of the
proton (\HepProcess{\Pup\Pup\Pdown}). 
Due to the \PWp and \PWm production processes in \EquationRef{wbos:wprod} being asymmetric with
respect to quark flavour, the difference in the $\Pup$ and $\Pdown$ \acp{PDF}
will cause the \PWpm cross sections to also differ.  
A measurement of the ratio of the \PWp and \PWm cross sections directly probes
the $\nicefrac{\Pup}{\Pdown}$ ratio in the proton.

\begin{figure}[htb]
  \centering
  %\includegraphics[width=0.5\textwidth]{placeholder}
  \caption{Proton PDF at $Q^2\approx M_{\PW}^2$.}
  \label{wbos:pdf}
\end{figure}


\FigureRef{wbos:pdf} shows PDF $\nicefrac{\Pup}{\Pdown}$ ratio in the proton at
$Q^2\approx M_{\PW}^2$. The ratio increases as $x\to 1$. As a consequence of
this, \Pup quarks tend to carry a greater fraction of the protons momentum,
$x$, than the \Pdown type quarks. This causes the \PWp to be produced at a
larger rapidity, where as the \PWm will tend to be produced more centrally. 
\FigureRef{wbos:wrapid} shows the \PWp and \PWm rapidity distributions at the
LHC. 

\begin{figure}[htb]
  \centering
  %\includegraphics[width=0.5\textwidth]{placeholder}
  \caption{Rapidity distributions for \PWpm production at the LHC.}
  \label{wbos:wrapid}
\end{figure}

Therefore a measurement of the asymmetric production of \PW bosons, as a
function of the rapidity of the boson, at the \ac{LHC} provides important
information on the up-type and down-type quark parton densities within the
proton. 


%In \EquationRef{wbos:xsec}, $\sum\limits_{a,b}$ represents the sum over the
%initial parton states $a$ and $b$. At the Tevatron (a proton-antiproton
%collider) the two incoming partons, $a$ and $b$, are most likely to be valence
%quarks from the proton and antiproton respectivly. Therefore, previous studys
%of the \PW boson at the Tevatron have been mostly sensative to the valence
%quarks of the proton.

%At the \ac{LHC} (a proton-proton collider), one parton is most likely be a
%valence quark with a high fraction of the protons momentum, and the other
%parton will tend to be a sea anti-quark with a lower fraction of the momentum
%than the quark, this causes the \PW bosons to be produced at higher rapidities.
%Therefore a measurement of the distribution of the \PW bosons rapidities at the
%\ac{LHC} provides direct information on the quark and anti-quark densities of
%the proton at a high scale and low values of the parton momentum fraction. 


\section{W Charge Asymmetry}

A particularly interesting observable involving \PW bosons produced in hadron
collisions is an asymmetry in the production of \PWp and \PWm bosons as a
function of rapidity.
In proton-antiproton colliders, such as Fermilab's Tevatron,  \PWp and \PWm
bosons are produced at equal rates but \PWp are produced more in the direction
on the incoming proton where as the \PWm are produced more in the opposite
direction.
In proton-proton colliders, such as the LHC, the \PWp is produced in greater
rates than the \PWm. Additionally the \PWp are produced at larger rapidities
where as the \PWm bosons are produced more centrally. \cite{phenom}

The difference in production rate of \PWpm boson is mainly due to the quark
content in the proton and the rapidity dependence is due to the \Pup quarks on
average carrying a greater fraction of the proton momentum that \Pdown quarks.
Therefore a measurement of the charge asymmetry can provide important
information on the parton density functions of the proton.\cite{phenom,pdf}.

At the LHC it has been proposed \cite{kom} that an asymmetry measurement could
also be used to measure the contribution of New Physics, since in the Standard
Model, \begin{equation}
\sigma_{SM}(\PWp + jets) \approx 1.3 \times \sigma_{SM}(\PWm + jets)
\end{equation}
where as typically in many New Physics theories and models predict
\begin{equation}
\sigma_{NP}(\PWp + jets) = \sigma_{NP}(\PWm + jets)
\end{equation}
examples include the pair production of gluinos in supersymmetry where the
gluinos decay to one charged lepton, missing energy and jets.\cite{kom}

\section{Electron Charge Asymmetry}

\PW bosons are identified by their decay to a lepton plus neutrino, however at
hadronic colliders the neutrino longitudinal momentum is unmeasured meaning
that the \PW rapidity is unknown. Instead what is studied is the lepton charge
asymmetry.\cite{phenom} The theoretical asymmetry is given by equation
\ref{eq:AsymThe}.

\begin{equation}
A_{the}(\eta)=\frac{  \frac{d\sigma}{d\eta}(\Wpenu) -
\frac{d\sigma}{d\eta}(\Wmenu) }{ \frac{d\sigma}{d\eta}(\Wpenu) +
\frac{d\sigma}{d\eta}(\Wmenu) }
\label{eq:AsymThe}
\end{equation} 

The experimentally measured asymmetry is given by equation
\ref{eq:AsymExp}.\cite{kom}
 
\begin{equation}
A_{exp}(\eta)=\frac{  \frac{dN}{d\eta}(\Pelectron) -
\frac{dN}{d\eta}(\APelectron) }{ \frac{dN}{d\eta}(\Pelectron) +
\frac{dN}{d\eta}(\APelectron) }
\label{eq:AsymExp}
\end{equation} 

The two measurements are related by equation \ref{eq:NumEve} which takes in to
account the experimental effects such as the luminosity (${\cal L}$), high
level trigger ($\epsilon_{HLT}$), offline efficiency ($ \epsilon_{off}$) and
the acceptance ($\epsilon_{acc}$).

\begin{equation}
\frac{dN}{d\eta} = {\cal L } \frac{d\sigma}{d\eta}  \epsilon_{HLT}
\epsilon_{off} \epsilon_{acc}
\label{eq:NumEve}
\end{equation} 

Since the asymmetry is a ratio, the luminosity high level trigger and the
offline efficiency cancel.\cite{me} The acceptance can not be cancelled
however, since it is a function of rapidity and momentum and electrons and
positrons will have different rapidity and momentum distributions. The
experimentally measured asymmetry can therefore be related to the theoretical
asymmetry by correcting for the acceptance.\cite{me}

\begin{align} 
A_{exp}(\eta) &= \frac{ \frac{dN}{d\eta}(\Pelectron) -
\frac{dN}{d\eta}(\APelectron) }{ \frac{dN}{d\eta}(\Pelectron) +
\frac{dN}{d\eta}(\APelectron) }\\   
              &= \frac{ \frac{d\sigma}{d\eta}(\Wpenu) -
\frac{\epsilon^{-}_{acc}}{\epsilon^{+}_{acc}} \frac{d\sigma}{d\eta}(\Wmenu) }{
\frac{d\sigma}{d\eta}(\Wpenu) + \frac{\epsilon^{-}_{acc}}{\epsilon^{+}_{acc}}
\frac{d\sigma}{d\eta}(\Wmenu) }
\label{eq:AsymExpCorr}
\end{align}

