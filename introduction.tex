\chapter{Introduction}
\label{chap:introduction}

The {SM} of particle physics is a quantum field theory that describes the
dynamics of elementary particles and their interactions through the
electromagnetic, strong and weak forces. The {SM} has been tested with a high
degree of precision and accuratley predicts all processes that are
observed in nature, with the exception of the gravity and neutrino masses. The
{SM} also predicts the existance of the Higgs boson, which may be the
particle discovered by the {LHC} experiments ({CMS} and {ATLAS}) and
announced on the fourth of July last year.

The {LHC} at {CERN} in Geneva is a particle accelerator designed to colide
protons ath \unit{14}{\TeV} with a luminosity of \unit{$10^{34}$}{cm}. This
thesis is concerned with the {CMS} experiment at the {LHC}, a general
purpose detector with a range of goals including discovery of the Higgs boson
and the precise measurements of the {SM}.

The analysis presented in this thesis is a measurement of the electron charge
asymmetry in inclusive \PW production. The asymmetric production of \PW bosons
is an important observable, in proton-proton collisions the \PWp is  produced at
a higher rate that the \PWm due to the excess of up type quarks with respect to
down type quarks, and furthermore, the \PWp will tend to be produced at larger
rapidities while the \PWm is mostly produced at central rapidities. A
measurement of the asymmetry can provide inportant information on the parton
densities of the colliding protons, specifically the $\nicefrac{u}{d}$ ratio. 

In the electronic decay of the \PW the longditudinal component of the neutrino
momentum is lost so the boson rapidity is unmeasurable. What is measurable is
the electron charge asymmetry as a function of the electron pseudorapidity. 

This thesis is structured as follows. In \ChapterRef{chap:sm} an overview of the
constiuent particles of the the {SM} is given and a mathemtical formulation
of the {SM} Lagrangian is shown. \ChapterRef{chap:wboson} describes the
production of the \PW bosons and their decay to an electron and neutrino. Also
shown are some theoretical predictions of the electron charge asymmetry for
different PDFs. \ChapterRef{chap:LHC} introduces the {LHC} and the
{CMS} detector and its subdetectors.  In \ChapterRef{chap:objects} the
electron reconstruction algorithms are summarised.  The measurement of the
electron charge asymmetry with \unit{36}{\invpb} of data is presented in
\ChapterRef{chap:analysis}. The update to the measurement with
\unit{840}{\invpb} is presented in \ChapterRef{chap:update}.
\ChapterRef{chap:conclusion} provides a summary of the results as well as
summarising the recent evidence for a new boson in {CMS}.




