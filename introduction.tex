
\chapter{Introduction}
The Large Hadron Collider (LHC) is a proton-proton collider that has been
constructed at CERN. 
It started to take data in late 2009 at a centre of mass energy (\sqrtS) of
\unit{0.9}{\TeV} and \unit{2.36}{\TeV} 
and starting in March 2010 at \unit{7}{\TeV} providing the opportunity to make
accurate measurements of the Standard Model 
(SM) and search for New Physics (NP).

In proton-proton colliders such as LHC it is expected that \PWp bosons will be
produced in greater numbers than \PWm bosons 
in contrast to previous proton antiproton colliders such as the Fermilab
Tevatron where the \PWp and \PWm are produced in equal numbers.
This charge asymmetry is due to protons containing more \Pup quarks than \Pdown
quarks and therefore a precise measurement of the 
\PW charge asymmetry at the LHC can provide useful information on the
\nicefrac{\Pup}{\Pdown} ratio\cite{phenom,pdf}.
It has also been proposed that a measurement of the \PWpm boson charge
asymmetry could also be used to investigate 
the contribution of New Physics since many New Physics models predict
$\sigma_{NP}(\PWp + jets) = \sigma_{NP}(\PWm + jets)$ 
which would cause a deviation from the theoretical Standard Model asymmetry
\cite{kom}.

%Electron Asymmetry
In this report the charge asymmetry in the process
\HepProcess{\Pproton\Pproton\to\PW(+X)\to\Pe\Pnu(+X)} is studied at 
$\sqrtS = \unit{7}{\TeV}$. The undetected neutrino measn that the rapidity of the
\PW boson cannot be directly measured 
so instead the electron charge asymmetry is measured, defined by
\begin{equation}
A_{the}(\eta)=\frac{  \frac{d\sigma}{d\eta}(\Wpenu) -
\frac{d\sigma}{d\eta}(\Wmenu) }{ \frac{d\sigma}{d\eta}(\Wpenu) +
\frac{d\sigma}{d\eta}(\Wmenu) }
\end{equation} 
where $\frac{d\sigma}{d\eta}(\Wenu)$ is the differential cross section with
respect to the electron pseudorapidity 
for electrons from \PWpm bosons\cite{kom}.

%Previous measurements
Previous measurements of the charge asymmetry have been performed with
\HepProcess{\Pproton\APproton} collisions at 
$\sqrtS = \unit{1.96}{\TeV}$ by the Tevatron experiments CDF\cite{cdfWAsym} and
D0\cite{d0WAsym}. 

A similar measurement is being investigated in the muon channel
(\HepProcess{\Pproton\Pproton\to\PW(+X)\to\Pmu\Pnu(+X)}) 
which shows that a total error on the charge asymmetry measurement from 0.024
to 0.028 could be obtained with \unit{10}{\invpb} 
of data which is dominated by the systematic error due to the relative
efficiency of electrons and positrons.\cite{wmunu}

This report summarises work done so far and described in the CMS analysis note
\cite{me}. 
In section 2 a brief overview of the LHC is given and in section 3 the CMS
detector is described in detail. 
Section 4 gives the motivation for a measurement of the charge asymmetry and
section 5 describes the method used. 
Section 7 describes the method used to extract the signal yield and section 8
evaluates the statistical and 
systematic uncertainties with the measurements. Finally some results on very
early data are presented.

\section{Standard Model}

