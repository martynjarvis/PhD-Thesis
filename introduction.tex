\chapter{Introduction}
\label{chap:introduction}

The Standard Model (SM) of particle physics is a quantum field theory that
describes the dynamics of elementary particles and their interactions through
the electromagnetic, strong and weak forces. The {SM} has been tested with a
high degree of precision and accurately predicts all processes that are observed
in nature, with the exception of gravity and neutrino oscillations. The {SM}
also predicts the existence of the Higgs boson, which may be the particle
discovered by the Large Hadron Collider (LHC) experiments ({CMS} and {ATLAS}) and announced on the
fourth of July last year\cite{chatrchyan2012observation,aad2012observation}.

The Large Hadron Collider (LHC) at European Laboratory for Particle Physics
(CERN) in Geneva is a particle accelerator designed to collide protons at
\unit{14}{\TeV} with a luminosity of \unit{$10^{34}$}{\lumiunits}. This thesis
is concerned with data collected by the Compact Muon Solenoid (CMS) experiment at
the {LHC}, a general purpose detector with a range of goals including discovery
of the Higgs boson and the precise measurements of the {SM}.

The analysis presented in this thesis is a measurement of the electron charge
asymmetry in inclusive \PW production. The asymmetric production of \PW bosons
is an important observable; in proton-proton collisions the \PWp is  produced at
a higher rate than the \PWm due to the excess of up-type quarks with respect to
down-type quarks. Furthermore, the \PWp will tend to be produced at larger
rapidities while the \PWm is mostly produced at central rapidities. A
measurement of the asymmetry as a function of the rapidity of the boson can
provide important information on the parton densities of the colliding protons,
specifically the ${u}/{d}$ quark ratio.  In the electron decay of the
\PW, the longitudinal component of the neutrino momentum is lost so the boson
rapidity is unmeasurable. What is measurable is the electron charge asymmetry as
a function of the electron pseudorapidity. 

This thesis is structured as follows. In \ChapterRef{chap:sm} an overview of the
constituent particles  and the fundamental interactions of the {SM} is given.
\ChapterRef{chap:wboson} describes the production of the \PW bosons and their
decay to an electron and neutrino. Also shown are some theoretical predictions
of the electron charge asymmetry for different PDFs.  \ChapterRef{chap:LHC}
introduces the {LHC} and the {CMS} detector and its sub-detectors.  In
\ChapterRef{chap:objects}, the particle reconstruction algorithms used in CMS
are summarised with the electron reconstruction and identification methods
described in detail. The measurement of the electron charge asymmetry with
\unit{36}{\invpb} of data is presented in \ChapterRef{chap:analysis}. The update
to the measurement with \unit{840}{\invpb} is presented in
\ChapterRef{chap:update}. \ChapterRef{chap:conclusion} provides a summary of
the results.




