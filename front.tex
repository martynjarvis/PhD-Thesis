%% Title
\maketitle

%% Abstract
\begin{abstract}
   %\thispagestyle{empty}
In this thesis, two measurements of the electron charge asymmetry in inclusive W
boson production with the CMS detector are presented. The measurements are
obtained from proton-proton collision data with $\sqrtS=\unit{7}{\TeV}$. The
first measurement is performed with data corresponding to an integrated
luminosity of \unit{36}{\invpb} collected by the CMS detector in 2010 and the
second one corresponding to \unit{840}{\invpb} collected during the first half
of 2011.

In proton-proton collisions, more \PWp are produced relative to \PWm due to the
prevalence of up type quarks with respect to down type valence quarks in the
proton. A measurement of this asymmetry as a function of the boson rapidity can
provide valuable information on the u/d ratio within the proton.  Since the
boson rapidity cannot be directly measured due to the longitudinal momentum
carried by the undetected neutrino, the asymmetry is measured as a function of
the pseudo-rapidity of the charged lepton, in this case the electron, from the
\PW decay.

In both measurements, events are selected by requiring a single electron with
tight selection criteria on quality of identification and measurement of the
energy of the electron. The signal yield is extracted using extended maximum
likelihood fits to the missing transverse energy spectrum using a set of
reference template shapes for the signal, electroweak background and QCD
background. The templates are obtained using Monte Carlo, corrected with
information from collision data events (for the signal and electroweak
backgrounds), and a control sample of events obtained from an inverted selection
(for the QCD background).

In the first measurement the charge asymmetry is measured in 6 bins of absolute
value of the electron's pseudo-rapidity and compared with predictions from
theory. The statistical error ranges from 0.006 to 0.010. The increased amount
of data in the second measurement allows the results to be presented in 11 bins
of absolute value of the electron's pseudo-rapidity. The statistical error has
been reduced and ranges from 0.003 to 0.004 and the global error in the range
0.006 to 0.014. 
\end{abstract}

\chapter*{Declaration}
\thispagestyle{empty}
  Declaration goes here.
  \vspace*{1cm}
  \begin{flushright}
    My Name
  \end{flushright}
\clearpage

\chapter*{Acknowledgements}
\thispagestyle{empty}
Acknowledgements go here\dots
\clearpage

\chapter*{Preface}
\thispagestyle{empty}
Preface goes here\dots
\clearpage

%o% todo list
\listoftodos

%% ToC
\tableofcontents

