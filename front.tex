%% Title
\maketitle

%% Abstract
\begin{abstract}
   %\thispagestyle{empty}
In this thesis, two measurements of the electron charge asymmetry in inclusive W
boson production with the CMS detector are presented. 

In proton proton collisions, more W+ are produced relative to W- due to the
prevalence of up type quarks with respect to down type valence quarks in the
proton. A measurement of this asymmetry as a function of the boson rapidity can
provide valuable information on the u/d ratio within the proton. Since the boson
rapidity cannot be directly measured due to the longitudinal momentum on the
undetected neutrino, the asymmetry is measured as a function of the
pseudo-rapidity of electron from the W decay.

The first asymmetry measurement is performed with 36 pb-1 of data collected in
CMS detector in 2010 and the latter measurement is performed with 840 pb-1 of
data collected in the first half of 2011.

In both measurements, events are selected by requiring a single electron with a
tight selection criteria. The signal yield is extracted using an extended
maximum likelihood fits to the missing transverse energy spectrum using a set of
reference template shapes for the signal, electroweak background and QCD
background. The templates are obtained using Monte Carlo corrected with
information from data events (for the signal and electroweak backgrounds) and a
control sample of events obtained from an inverted selection (for the QCD
background).

In the first measurement the charge asymmetry is measured in 6 bins of absolute
value of electron pseudo-rapidity and compared with predictions from theory. The
statistical error goes from 0.006 to 0.010. The increased amount of data in the
second measurement allows results to be presented in 11 bins of absolute value
of electron pseudo-rapidity. The statistical error has been reduced to the range
of 0.003 to 0.004 and the global error is in the range 0.006 to 0.014.
\end{abstract}

%% Declaration
%\begin{declaration}
  %This dissertation is awesome
  %\vspace*{1cm}
  %\begin{flushright}
    %My Name
  %\end{flushright}
%\end{declaration}


%% Acknowledgements
%\begin{acknowledgements}
  %Of the many people who deserve thanks, some are particularly prominent,
  %such as my supervisor\dots
%\end{acknowledgements}


%\begin{preface}
 %Preface
 %\noindent
 %More preface
%\end{preface}

%\dedication{To me...}

%o% todo list
\listoftodos

%% ToC
\tableofcontents

%% Strictly optional!
%\frontquote%
  %{quote}%
  %{quoter}


