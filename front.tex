%% Title
\begin{titlepage}
\begin{center}
\vspace{5cm}
~
\\[2cm]
\Large{\textbf{Measurement of the Electron Charge Asymmetry in Inclusive \PW
Production in $pp$ Collisions at $\sqrt{s} = \unit{7}{\TeV}$ in the CMS
Experiment}}
\\[2cm]
\large{Martyn Jarvis}
\\[1cm]
\large{Blackett Laboratory\\Imperial College London}
\vfill
\normalsize{Thesis submitted to Imperial College London\\
       for the degree of Doctor of Philosophy\\
       and the Diploma of Imperial College}
\\[1cm]
\normalsize{Spring 2013}

\end{center}
\end{titlepage}
\cleardoublepage

\begin{abstract}
\thispagestyle{plain}
\setcounter{page}{3}
\begin{adjustwidth}{25pt}{25pt}
In this thesis, two measurements of the electron charge asymmetry in inclusive W
boson production with the CMS detector are presented. The measurements are
obtained from proton-proton collision data with $\sqrtS=\unit{7}{\TeV}$. The
first measurement is performed with data corresponding to an integrated
luminosity of \unit{36}{\invpb} collected by the CMS detector in 2010 and the
second one uses data corresponding to \unit{840}{\invpb} collected during the
first half of 2011.

In proton-proton collisions, more \PWp bosons are produced relative to \PWm due to the
prevalence of up-type quarks with respect to down-type valence quarks in the
proton. A measurement of this asymmetry as a function of the boson rapidity can
provide valuable information on the u/d ratio within the proton.  Since the
boson rapidity cannot be directly measured due to the longitudinal momentum
carried by the undetected neutrino, the asymmetry is measured as a function of
the pseudorapidity of the charged lepton, in this case the electron, from the
\PW decay.

In both measurements, events are selected by requiring a single electron with
tight selection criteria on quality of identification and measurement of the
energy of the electron. The signal yield is extracted using extended maximum
likelihood fits to the missing transverse energy spectrum using a set of
reference template shapes for the signal, electroweak background and QCD
background. The templates are obtained using Monte Carlo simulation, corrected with
information from collision data events (for the signal and electroweak
backgrounds), and a control sample of events obtained from an inverted selection
(for the QCD background).

In the first measurement the charge asymmetry is measured in 6 bins of the absolute
value of the electron's pseudorapidity and compared with predictions from
theory. The statistical error ranges from 0.006 to 0.010. The increased amount
of data in the second measurement allows the results to be presented in 11 bins
of the absolute value of the electron's pseudorapidity. The statistical error has
been reduced and ranges from 0.003 to 0.004 and the global error is in the range
0.006 to 0.014. 
\end{adjustwidth}
\end{abstract}
\clearpage
\thispagestyle{plain}

\chapter*{Declaration}
\begin{adjustwidth}{50pt}{50pt}

This thesis contains the work for two measurements of the electron charge
asymmetry that were performed with data from the CMS experiment taken in 2010
and 2011. The results of the measurements were approved for the CMS
collaboration and are published in \cite{asym36,asym840}. The
measurements are also documented in the analysis notes
\cite{bendavid2011electron,baisini2010electron}. 

The work in Chapters \ref{chap:analysis} and \ref{chap:update} is mine
with the exception of \SectionRef{sec:recoil} which was provided to me and
Sections \ref{sec:corrections1} and \ref{sec:corrections2} which were
either performed by others or in collaboration with others.
\vspace*{5cm}
\small{\textit{
The copyright of this thesis rests with the author and is made available under a
Creative Commons Attribution Non-Commercial No Derivatives licence. Researchers
are free to copy, distribute or transmit the thesis on the condition that they
attribute it, that they do not use it for commercial purposes and that they do
not alter, transform or build upon it. For any reuse or redistribution,
researchers must make clear to others the licence terms of this work.
} }
  \vspace*{1cm}
  \begin{flushright}
    Martyn Jarvis
  \end{flushright}
\end{adjustwidth}
\clearpage

\chapter*{Acknowledgements}
\begin{adjustwidth}{50pt}{50pt}
%\thispagestyle{empty}
I would like to thank my supervisor, Jim Virdee, for his guidance and
suggestions through out my PhD.  I would also like to thank Michele Pioppi for
his supervision, motivation and patience, and Paul Dauncey for his
help in the preparation of this thesis.

Finally, I am grateful for the support my family have given me and I would like to thank
Ju and all her support and patience throughout the past three years.
\end{adjustwidth}
\clearpage

%\chapter*{Preface}
%\thispagestyle{empty}
%\clearpage

% Todo list
\listoftodos

% ToC
\tableofcontents

% List of stuff
\listoffigures
\listoftables
