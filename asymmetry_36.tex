\chapter[Electron Charge Asymmetry]{ Measurement of the electron charge asymmetry with \unit{36}{\invpb} }

In this chapter the first measurement of the electron charge asymmetry in
inclusive \inclusiveWe production with the \ac{CMS} detector is described.
The analysis is performed on the full 2010 dataset, which corresponds to a
luminosity of \unit{36.1}{\invpb}.

\section{Event Selection}

The event selection is performed on a single electron dataset. This dataset is
formed of events that are selected using various single photon and single
electron triggers. From this dataset, electrons are selected that pass a limited
number of cuts. Events that contain only a single electron are then selected for
the analysis.

\subsection{Trigger}

\label{asym36:triggerdef}
Several triggers were used to select the events, due to the increasing
luminosity in the \ac{LHC} 2010 run.
In the initial runs, events were selected using only a single photon trigger. 
As the luminosity increased these triggers became prescaled, it was
necessary to use electron triggers to select events. 
As the luminosity increased even further, it was necesary to use electron
triggers that included cuts on certain electron ID variables.

The triggers used to select the events are sumarised in \TableRef{asym36:triggers}
where ``HLT\_Ele$X$'' indicates a selection requiring an electron with  $\Pt > \unit{X}{\GeV}$. 
``Photon'' in the name indicates that the selection was applied to ECAL
superclusters rather than a reconstructed electron. 
``SW'' stands for small window, where window refers to the electron
pixel-matching window. 
 ``Cleaned'' indicates that spikes in the \ac{ECAL} have been removed.  \todo{define spikes?}

``CaloEleId'' and ``TightCaloIdIso'' represent increasingly tighter selection
based on the shower shape ID and isolation variables from only the \ac{ECAL},
and not the $\Delta\phi$ or $\Delta\eta$ variables.  

``TightCaloIdIso'' indicates a tight selection based on all ID variables. 
This nullifies the inverted cuts used for the control region but
it was the only trigger available for these runs without a prescale applied.
To compensate for the missing events in the control region, a looser prescaled
trigger was also applied in these runs.

\begin{table}[htbp]
  \centering
  \begin{tabular}{ l l }
    \toprule
    Run Ranges & Trigger String\\
    \midrule
    132440-137028 & \verb=HLT_Photon10_L1R= \\
    138564-140401 & \verb=HLT_Photon15_Cleaned_L1R= \\
    141956-144114 & \verb=HLT_Ele15_SW_CaloEleId_L1R= \\
    146428-147116 & \verb=HLT_Ele17_SW_CaloEleId_L1R= \\
    147196-148102 & \verb=HLT_Ele17_SW_TightEleId_L1R= \\
                  & \verb=HLT_Ele17_SW_L1R (prescaled)= \\ 
    148822-149063 & \verb=HLT_Ele22_SW_TighterCaloIdIso1_L1R_v1= \\
    149181-149442 & \verb=HLT_Ele22_SW_TighterCaloIdIso1_L1R_v2= \\
    \bottomrule
  \end{tabular}
  \caption{Triggers used to select the data used in this analysis.}
  \label{asym36:triggers}
\end{table}

\subsection{Event Selection}
An Event is selected if it contains a single electron that passes all the electron
selection requirements.
To remove Drell-Yan events, an event is vetoed if it contains a second lepton
(an electron passing a loose selecton, or an isolated muon) with $\PT > 
\unit{15}{\GeV}$.

The exected composition of the selected events, derived from MC simulation
samples, is shown in \TableRef{asym36:selectedcomp}. With a \pT cut of
\unit{25}{\GeV} it is expected that the selected events are \unit{$\approx
65$}{\GeV} signal events, and of the remaing \unit{35}{\%} background events, the
majority of them are \ac{QCD} background events with a small ammount of \ac{EWK}
events. 

\begin{figure}[htbp]
  \begin{center}
    \includegraphics*[width=0.85\textwidth]{pfmet_dist}
    \caption{Particle Flow transverse missing energy (PFMET) distribution for selected events.}
  \label{fig:pfmet}
  \end{center}
\end{figure}

The Particle Flow \ETm distribution for the events that pass the event
selection, with an electron cut of $\PT > \unit{25}{\GeV}$ and $|\eta| < 2.4$ is
shown in \FigureRef{fig:pfmet}. There are two obvious peaks in the
distribution. The peak at \unit{$\ETm \approx 40$}{\GeV} is the
\HepProcess{\PW\to\Pelectron\Pnue} signal region. This region will also contain
\HepProcess{\PW\to\Ptau\Pnut} backgrounds. The peak at
\unit{$\ETm\approx 10$}{\GeV} is the background region that contains the \ac{QCD}
and \ac{DY} background events.

\begin{table}[htbp]
\begin{center}
%\begin{sideways}
\begin{tabular}{lcrrr}
    \toprule
  $|\eta|$ range & Charge & \multicolumn{3}{c}{Selected Events}\\
                 &        & $\PT>25$ \GeV & $\PT>30$ \GeV & $\PT>35$ \GeV\\
\midrule
$0.0<| \eta |<0.4$ &+& 18956&14232&9885\\
                   &-& 15060&11505&8105\\
$0.4<| \eta |<0.8$ &+& 20118&14966&10345\\
                   &-& 15736&11780&8307\\
$0.8<| \eta |<1.2$ &+& 20681&15091&10184\\
                   &-& 16167&11735&8112\\
$1.2<| \eta |<1.4$ &+& 10646&7606&5161\\
                   &-& 8226&5871&4067\\
$1.6<| \eta |<2.0$ &+& 16426&11877&7814\\
                   &-& 11678&8578&5886\\
$2.0<| \eta |<2.4$ &+& 18885&13239&8726\\
                   &-& 13226&9227&6184\\
    \bottomrule
\end{tabular}
%\end{sideways}
\end{center}
\caption{Number of events passing the event selection for lepton momentum cut of $\PT>25$ \GeV, $\PT>30$ \GeV and $\PT>35$ \GeV .}
    \label{asym36:selectedevents}
\end{table}

The number of selected events that pass the
event selection are shown in \TableRef{asym36:selectedevents}. To obtain a
measurement of the asymmetry from the selected events it is necesary to extract
the signal yield from each pseudorapidity/charge bin.

\begin{table}[htbp]
\begin{center}
\begin{tabular}{llrrr}
    \toprule
& & $\PT>25$ \GeV & $\PT>30$ \GeV & $\PT>35$ \GeV  \\
\midrule
Signal & \HepProcess{\PW\to\Pe\Pnu} & $65.1\%$&$63.5\%$ &$60.2\%$ \\
EWK & \HepProcess{\PZ\to\Ptau\Ptau} & $0.4\%$ &$0.3\%$  &$0.2\%$ \\
    & \HepProcess{\PZ\to\Pe\Pe}     & $4.1\%$ &$3.5\%$  &$3.0\%$\\
    & \HepProcess{\PW\to\Ptau\Pnu}  & $1.9\%$ &$1.1\%$  &$0.6\%$\\
    & \HepProcess{\Ptop\APtop}      & $0.3\%$ &$0.3\%$  &$0.3\%$\\
    & Total                         & $6.7\%$ &$5.2\%$  &$4.1\%$\\
QCD & Total                         & $28.2\%$&$31.3\%$ &$35.7\%$\\
    \bottomrule
\end{tabular}
\caption{Composition of selected events for lepton momentum cut of $\PT>25$ \GeV, $\PT>30$ \GeV and $\PT>35$ \GeV .}
\label{asym36:selectedcomp}
\end{center}
\end{table}

The exected composition of the selected events, derived from MC simulation
samples, is shown in \TableRef{asym36:selectedcomp}. With a \pT cut of
\unit{25}{\GeV} it is expected that the selected events are \unit{$\approx
65$}{\GeV} signal events, and of the remaing \unit{35}{\%} background events, the
majority of them are \ac{QCD} background events with a small ammount of \ac{EWK}
events. 

\subsection{Fit Results}
The results of the extended maximum liklihood fits to each pseudorapidity/charge
bin are shown in \FigureRef{fig:fit1} and \FigureRef{fig:fit2}.
The ratio between the fits and the data for each of the 
bins are shown in \FigureRef{fig:fit1ratio} and \FigureRef{fig:fit2ratio}.
The signal yield in each bin is summarised in \TableRef{tab:rawasym} and the
$\chi^2$ of each fit is noted in \TableRef{tab:chi2}.

\begin{table}[htbp]
\begin{center}
%\begin{sideways}
\begin{tabular}{lcrrr}
    \toprule
$|\eta|$ range & Charge & \multicolumn{3}{c}{Signal Yield}\\
               &        & $\PT>25$ \GeV & $\PT>30$ \GeV & $\PT>35$ \GeV  \\
\midrule
$0.0<| \eta |<0.4$ &+&-&-&-\\
                   &-&-&-&-\\
$0.4<| \eta |<0.8$ &+&-&-&-\\
                   &-&-&-&-\\
$0.8<| \eta |<1.2$ &+&-&-&-\\
                   &-&-&-&-\\
$1.2<| \eta |<1.4$ &+&-&-&-\\
                   &-&-&-&-\\
$1.6<| \eta |<2.0$ &+&-&-&-\\
                   &-&-&-&-\\
$2.0<| \eta |<2.4$ &+&-&-&-\\
                   &-&-&-&-\\
    \bottomrule
\end{tabular}
%\end{sideways}
\end{center}
\caption{Signal yield}
    \label{fig:sigyield}
\end{table}

\begin{table}[htbp]
\begin{center}
\begin{tabular}{lcr}
    \toprule
$|\eta|$ range &Charge & $\chi^2$/ndof of Fit\\
\midrule
$0.0<| \eta |<0.4$ &+&  0.86\\
                   &-&  0.84\\
$0.4<| \eta |<0.8$ &+&  0.99\\
                   &-&  1.36\\
$0.8<| \eta |<1.2$ &+&  1.05\\
                   &-&  1.13\\
$1.2<| \eta |<1.4$ &+&  0.97\\
                   &-&  1.30\\
$1.6<| \eta |<2.0$ &+&  1.38\\
                   &-&  1.61\\
$2.0<| \eta |<2.4$ &+&  1.44\\
                   &-&  1.11\\
    \bottomrule
\end{tabular}
\caption{\label{tab:chi2}$\chi^2$/ndof of Fit.}
\end{center}
\end{table}

\section{Systematic Uncertainties}
The following section will describe the main sources of systematic uncertainty
for this measurement. 

\subsection{Relative Efficiency}

If the reconstruction efficiency of electrons is different to that of positrons
then the measured asymmetry wil be diluted and will need to be corrected for
the relative efficiency.

\begin{equation}
A_{exp}(\eta) = \frac{
                    \frac{dN}{d\eta}(e^+)-
                    \frac{\epsilon^+}{\epsilon^-}\frac{dN}{d\eta}(e^-)
                }
                {
                    \frac{dN}{d\eta}(e^+)+
                    \frac{\epsilon^+}{\epsilon^-}\frac{dN}{d\eta}(e^-)
                }
\end{equation}

The efficiency for electrons and positrons is measured using the tag and probe
method \todo{reference the CMS tag and probe method paper/note}
with a sample of \Zee events from the same datasets used in the analysis. 
The \Zee events offer a high purity source of unbiased electrons with which to
measure the efficiencies.

From the sample of \Zee events a ``tag'' electron is selected with a strict
selection criteria. 
A ``probe'' electron is selected with the same electron selection decribed
earlier.
The invariant mass of the tag-probe pair is required to be
$\unit{60}{\GeV} < M_ee < \unit{120}{\GeV}$ to ensure a high purity sample.

Efficiencies can then be calculated by measuring the signal yield in events
with one tag electron and one probe passing the selection (tag \& pass) and
events where the probe electron fails the selection (tag \& fail).
The signal yield is extracted using an simultaneous maximum likelihood fit to
both the tag \& pass and the tag \& fail samples.

For this analysis the efficiencies are measured in two parts:

\begin{itemize}
    \item GSF tracking efficiency
    \item Identification efficiency, including conversion rejection, unaminous
charge assignment and HLT request.
\end{itemize}

\begin{table}[htbp]
\begin{center}
\begin{sideways}
\begin{tabular}{cccccccc}
    \toprule
$|\eta|$  & \multicolumn{3}{c}{GSF tracking } & \multicolumn{3}{c}{ID } & $R_\epsilon$ \\
region    & $\epsilon_{GSF}^+$ (\%) &$\epsilon_{GSF}^-$ (\%) & $R_{\epsilon_{GSF}}$ 
                                              & $\epsilon_{ID}^+$ (\%) &$\epsilon_{ID}^-$ (\%) & $R_{\epsilon_{ID}}$ &  \\
\midrule
$\left[ 0.0,0.4 \right]$ & 95.7$\pm$1.1 & 97.5$\pm$1.0 & 0.982$\pm$0.015 & 71.2$\pm$1.5 & 68.4$\pm$1.5 & 1.04$\pm$0.03 &1.02$\pm$0.035  \\
$\left[ 0.4,0.8 \right]$ & 98.8$\pm$ 1.0& 98.5$\pm$1.1 & 1.003$\pm$0.015 & 72.5$\pm$1.7 & 75.6$\pm$1.6 & 0.96$\pm$0.04 &0.96$\pm$ 0.04 \\
$\left[ 0.8,1.2 \right]$ & 97.6$\pm$ 1.0& 98.4$\pm$1.0 & 0.992$\pm$0.015 & 77.4$\pm$1.5 & 74.4$\pm$1.7 & 1.04$\pm$0.04 &1.03$\pm$ 0.04 \\
$\left[ 1.2,1.4 \right]$ & 96.2$\pm$ 1.5& 96.3$\pm$1.5 & 0.999$\pm$0.022 & 69.3$\pm$2.7 & 73.0$\pm$2.6 & 0.95$\pm$0.05 &0.95$\pm$0.05  \\
$\left[ 1.6,2.0 \right]$ & 96.8$\pm$ 1.2& 96.9$\pm$1.0 & 0.999$\pm$0.015 & 61.9$\pm$2.0 & 63.6$\pm$2.0 & 0.97$\pm$0.05 &0.97$\pm$0.05  \\
$\left[ 2.0,2.4 \right]$ & 96.4$\pm$ 1.1& 97.0$\pm$1.0 & 0.994$\pm$0.015 & 58.2$\pm$2.1 & 56.7$\pm$2.1 & 1.03$\pm$0.05 &1.02$\pm$0.05  \\
\midrule
$\left[ 0.0,1.4 \right]$ & 98.8$\pm$0.5 & 98.4 $\pm$0.5 & 1.004$\pm$0.007 & 84.1$\pm$0.8 & 83.8$\pm$0.8 & 1.003$\pm$0.014 & 1.007$\pm$ 0.015 \\
$\left[ 1.6,2.4 \right]$ & 98.3$\pm$0.7 & 97.8 $\pm$0.7 & 1.005$\pm$0.010 & 70.7$\pm$1.4 & 71.5$\pm$1.4 & 0.99$\pm$0.03 &0.99$\pm$ 0.03 \\
\midrule 
$\left[ 0.0,2.4 \right]$ & 98.5$\pm$0.4 & 97.8$\pm$0.4 & 1.007$\pm$0.006 & 80.3$\pm$0.7 & 80.3$\pm$0.7 & 1.000$\pm$0.012 &1.007$\pm$0.014  \\
    \bottomrule
\end{tabular}
\end{sideways}
\end{center}
\caption{ GSF tracking and identification efficiency as a function of charge.}
\label{asym36:tagprobe}
\end{table}

The \TableRef{asym36:tagprobe} shows the GSF tracking effiency and identifcation 
effiency as a function of the charge.

The relative efficiency: 

\begin{equation}
R_\epsilon  =  \frac{\epsilon^+}{\epsilon^-}
\end{equation}

is found to be statistically compatible with 1 so the measured asymmetry is not
corrected for this effect.

The main systematic errors on the efficiency measurements are the energy scale
and the signal shape used to extract the signal yield. Fortunately, these
errors will cancel in the calculation of the ratio $R_\epsilon$, the difference
on  $R_\epsilon$ introduced by the energy scale and signal shape is negligable
when compared to the statistical uncertainty of the measurement, so only the
statistical uncertainty is propogated to the error in the ratio,
$dR_\epsilon$.

\begin{equation}
  \label{eq:releff}
  \sigma_{\mathcal{A}} =\mathcal{}(R_\epsilon=1) - \mathcal{A}(R_\epsilon=1\pm dR_\epsilon)  \simeq \frac{dR_\epsilon}{2}(1-\mathcal{A}^2)\simeq \frac{dR_\epsilon}{2}
\end{equation}


\subsection{Signal Extraction Method}

The systematic uncertainty due to the siganl extraction method is evaluated be
considering the error introduced be each \ETm tempalte shape used in the fit
separately.

\subsubsection{Background \ETm Shape}

The \ac{QCD} and \gjet \ETm template shape is obtained from a control sample of
events by antiselecting electrons. This may introduce a systematic bias to the
measurement if there is a difference between the anti-selected \ac{QCD} and \gjet
\ETm samples and the selected \ac{QCD} and \gjet samples.

The systematic uncertainty due to the \ac{QCD} and \gjet \ETm shape is evaluated by
varying the anti-selection used to obtain the control sample and observing the
effect that this has on the measured asymmetry.

For each variation on the antiselection, 500 pseudo-data experiments are
generated with the number of events that are expected in \unit{36.1}{\invpb} of
data. The distribution of the measured asymmetry is then fitted with a
gaussian.
The effect that changing the anti-selection has on the mean of the guassian is
studied, the maximum distance from the asymmetry measured with the nominal
anti-selection is taken as an estimate of the systematic uncertainty.

\begin{table}[htbp]
\begin{center}
\begin{tabular}{crr}
    \toprule
$|\eta|$  &\multicolumn{2}{c}{ $\sigma(\mathcal{A}) \times 10^{-4}$}\\
   range      & MC & Data\\
\midrule
$0.0<|\eta|<0.4$ & 8 & 12\\
$0.4<|\eta|<0.8$ & 7 & 9\\
$0.8<|\eta|<1.2$ & 8 & 21\\
$1.2<|\eta|<1.4$ & 12& 25\\
$1.6<|\eta|<2.0$ & 6 & 10\\
$2.0<|\eta|<2.4$ & 22& 13\\
    \bottomrule
\end{tabular}
\caption{Maximum distance between the asymmetry measured with many different antiselections
and the asymmetry measured with the chosen antiselection in MC pseudo data and real data for each eta bin.}
\label{tab:systQCD}
\end{center}
\end{table}


\subsubsection{Signal \ETm Shape from Boson Recoil}

The signam \ETm shape is constructed using information from the boson recoil.
There are three main sources of uncertainty due to the signal template,

\begin{itemize}
    \item the uncertainty in the recoil corrections,
    \item the effect the energy scale has on the recoil corrections,
    \item the uncertainty on the \ac{PDF} used to generate the events that the
recoil corrections are applied to.
\end{itemize}

To evaluate the effect of the uncertainties of the recoil method, the upper and
lower limits on the corrections are used to generate different tempaltes, and
the effect on the measured asymmetry is evaluated as a measure of the
systematic uncertainty.

The recoil method uses generator level \ac{MC} simulation as an input to the
template shape. To evaluate the efect of the generator used, templates are
generated with the CTEQ 6.6 \todo{cite the CTEQ guys here}
uncertainty \acp{PDF} which contain the central \ac{PDF} and 44 error \acp{PDF}
which contain the \unit{95}{\%} \ac{CL} for each of the 22 free parameters in
the \ac{PDF}. The maximum change in distance with respect to the central value
is taken as a measure of the systematic uncertainty.

Any differences in the energy scale of the electrons between data and \ac{MC}
must be taken in to account to ensure accurate \ETm predictions. The \ac{MC} energy
scale corrections were determined from \PZ data and applied to the \PZ \ac{MC}
before calculating the recoil components \cite{recoil}.

\begin{table}[htbp]
\begin{center}
\begin{tabular}{crrrr}
    \toprule
$|\eta|$   & \multicolumn{4}{c}{$\sigma(\mathcal{A}) \times 10^{-4}$}\\
range      & Recoil Corr. & Energy Scale & PDF & Combined \\
\midrule
$0.0<|\eta|<0.4$ &  4 & 2 & 10  & 11 \\
$0.4<|\eta|<0.8$ &  6 & 3 & 15  & 16 \\
$0.8<|\eta|<1.2$ &  5 & 2 & 15  & 16 \\
$1.2<|\eta|<1.4$ &  9 & 5 & 20  & 22 \\
$1.6<|\eta|<2.0$ & 11 & 4 & 20  & 23 \\
$2.0<|\eta|<2.4$ &  7 & 3 & 20  & 21 \\
    \bottomrule
\end{tabular}
\caption{\label{tab:systSIG}Systematic uncertainty due to the Signal \ETm shape used in the signal
extraction method assigned to each eta bin.}
\end{center}
\end{table}

\subsubsection{\ac{EWK} \ETm Shape}

The \ac{EWK} shape is also generated from \ac{MC} samples. During the fitting
procedure, the \ac{EWK} shape is fixed to the \Wenu signal shape according to
the cross section taken from the \ac{MC} samples. To estimate the effect of the
uncertainty of the cross section has on the asymmetry measurement, the \ac{EWK}
background is artificially varied by \unit{$\pm20$}{ \% } and the effect on the
asymmetry is measured. Even with an over estimation of the uncertainty on the
cross section, the effect on the asymmetry is found to be small.

\begin{table}[htbp]
\begin{center}
\begin{tabular}{cr}
    \toprule
$|\eta|$ range & $\sigma(\mathcal{A}) \times 10^{-4}$\\
\midrule
$0.0<|\eta|<0.4$ & 0\\
$0.4<|\eta|<0.8$ & 3\\
$0.8<|\eta|<1.2$ & 1\\
$1.2<|\eta|<1.4$ & 1\\
$1.6<|\eta|<2.0$ & 0\\
$2.0<|\eta|<2.4$ & 3\\
    \bottomrule
\end{tabular}
\caption{\label{tab:systEWK}Systematic uncertainty due to the electroweak \ETm shape used in the signal extraction method assigned to each eta bin.}
\end{center}
\end{table}


\subsection{Charge Misassignment}

The charge misassignment rate, $\omega$, is the rate at which electrons are
misasigned as positive charge and identified as positrons, and vice versa. The
misassignment induces a dilution factor to the asymmetry as a function of the
electron pseudorapidity. If it is assumed that the misassignment rate of
electrons to positrons is the same as the rate of positrons to electrons, \ie 

\begin{equation}
  \omega( \HepProcess{\APelectron \to \Pelectron} ) =
  \omega( \HepProcess{\Pelectron \to \APelectron} )
\end{equation}

then the dilution factor is given by $(1-2\omega_\eta)$ and the measured
asymmetry must be corrected by the following relation

\begin{equation}
  A_{exp}(\eta) = (1-2\omega_\eta)
                \frac{
                    \frac{dN}{d\eta}(e^+)-
                    \frac{\epsilon^+}{\epsilon^-}\frac{dN}{d\eta}(e^-)
                }
                {
                    \frac{dN}{d\eta}(e^+)+
                    \frac{\epsilon^+}{\epsilon^-}\frac{dN}{d\eta}(e^-)
                }
\end{equation}

The rate of charge misassignment is obtained from \Zee samples selected with
the same selection used in the analysis. The rate is measured by comparing the
same sign \PZ\ yield (\HepProcess{\PZ\to\Pepm\Pepm}) to the opposite sign \PZ\
yield (\HepProcess{\PZ\to\Pepm\Pemp}).

\begin{table}[htbp]
  \begin{center}
\begin{tabular}{lrrrr}
\toprule
$\eta$ range        & $\omega \times 10^{-4}$  & \multicolumn{3}{c}{$\sigma(\mathcal{A})_{misch}\times 10^{-4}$}\\
& & \PT $>$ 25 \GeV & \PT $>$ 30 \GeV & \PT $>$ 35 \GeV \\
\midrule
$0.0<| \eta |<0.4$  & $0^{+8}$          &  2 &  2 & 2 \\ 
$0.4<| \eta |<0.8$  & $8^{+8}_{-8}$     &  3 &  2 & 2 \\
$0.8<| \eta |<1.2$  & $11^{+10}_{-8}$   &  3 &  3 & 3 \\
$1.2<| \eta |<1.4$  & $34^{+21}_{-15}$  &  8 &  7 & 6 \\
$1.6<| \eta |<2.0$  & $41^{+20}_{-15}$  &  9 &  8 & 7 \\
$2.0<| \eta |<2.4$  & $25^{+21}_{-15}$  & 10 & 10 & 9 \\
\bottomrule
\end{tabular}
\caption{\label{tab:mischarge}Charge mismeasurement rate and systematic effect on the charge asymmetry.}
\end{center}
\end{table}

\subsection{Lepton Energy Scale and Resolution}

The energy resolution and scale of the electrons can introduce a systematic
error on the asymmetry due to the effect of of the transverse momentum cut
applied to the electrons. The largest source of electron scale bias is the
radiation induced change to the ECAL crystal transparency.

To correct fot this effect, energy scale and resolution corrrections are
derrived using a \Zee mass distribution. The corrections are parameterised by
six energy scale factors, $s_i$, and six resolutions, $\sigma_i$, one for each
pseudorapidity bin in the asymmetry measurement.
The scale factors represent the average factor each electrons \Pt in data
should be corrected to match the expected \ac{MC} energy scale.
The resolution factors represent the difference of the resolution in data and
\ac{MC}. It is the additional smearing that would need to be applied to
reconstruction level \ac{MC} to match the observed resolution in data.

A sample of \Zee events is  split in to 21 categories which correspond to all
combinations of pseudorapidity bins of the two electrons ($6+\binom{6}{2} = 21$).

A mass template s obtained in each category from \ac{MC} simulation where a
perfect \ac{ECAL} calibration is considered.

A simultaneous fit to the \Zee mass is performed in each of the 21 categories
to determine the six energy scale factors, $s_i$, and the six resolutions, 
$\sigma_i$.

In each category ($category_{ij}$) where one electron is in the $i^{th}$
pseudorapidity bin and the other is in the $j^{th}$ bin, the \ac{MC} template
mass shape is scaled by 

\begin{equation}
    \frac{1}{\sqrt{s_i s_j} } 
\end{equation}

and smeared by an addtional gaussian with width of 

\begin{equation}
    \sqrt{\sigma_i^2+\sigma_j^2}
\end{equation}

%TODO figures
\todo[inline]{Add figures to demonstrate this}

The corrections are applied to the electron before the final \Pt cut so tjat
the measured asymmetry is corrected for the energy scale.

A conservative uncertainty of \unit{1}{\% } is assigned to the electron energy
after the scale corrections. A systematic error is then estimated by measuring
the difference of the measured charge asymmetry with and without the additional
\unit{1}{\% } scale factor.

\begin{table}[htbp]
  \begin{center}
    \begin{tabular}{cccc}
    \toprule
$\eta$ range& $\sigma{\mathcal{A}} \times 10^{-4}$  & Additional $\sigma{E_{e^\pm}}$  & $\sigma{\mathcal{A}} \times 10^{-4}$ \\
& Perfect ECAL  & from fit  &  Realistic ECAL\\
& Calibration & (GeV) & Calibration \\
\midrule
$0.0<| \eta |<0.4$  &  8  & 0.2  &  4 \\
$0.4<| \eta |<0.8$  &  7  & 0.4  &  5\\
$0.8<| \eta |<1.2$  & 17  & 0.3  & 19\\
$1.2<| \eta |<1.4$  & 41  & 1.0  & 43\\
$1.6<| \eta |<2.0$  & 31  & 0.9  & 32 \\
$2.0<| \eta |<2.4$  & 41  & 0.3  & 42\\
    \bottomrule
    \end{tabular}
    \caption{\label{tab:acc}Systematic error for detector effects in the acceptace corrections.}
  \end{center}
\end{table}

\begin{table}[htbp]
  \begin{center}
    \begin{tabular}{cccc}
    \toprule
$\eta$ range& \PT $>$ 25 \GeV & \PT $>$ 30 \GeV & \PT $>$ 35 \GeV \\
\midrule
$0.0<| \eta |<0.4$  & 4 & 5 &-3 \\
$0.4<| \eta |<0.8$  & 5 & 9 & -10\\
$0.8<| \eta |<1.2$  & 19 & 9 & 32\\
$1.2<| \eta |<1.4$  & 43 &37 & 24\\
$1.6<| \eta |<2.0$  & 32 &50 & 43\\
$2.0<| \eta |<2.4$  & 42 &52 & 27\\
    \bottomrule
\end{tabular}
\caption{\label{tab:bias}Bias values due to the electron resolution for different lepton \PT cuts.}
  \end{center}
\end{table}

\begin{table}[htbp]
  \begin{center}
    \begin{tabular}{cccc}
    \toprule
$\eta$ range& \PT $>$ 25 \GeV & \PT $>$ 30 \GeV & \PT $>$ 35 \GeV \\
\midrule
$0.0<| \eta |<0.4$  & 10 & 5 & 14\\
$0.4<| \eta |<0.8$  & 7 & 15 & 44\\
$0.8<| \eta |<1.2$  & 2 & 24 & 31\\
$1.2<| \eta |<1.4$  & 19 & 27 & 46\\
$1.6<| \eta |<2.0$  & 24 & 17 & 28\\
$2.0<| \eta |<2.4$  & 16 & 17  & 44\\
    \bottomrule
\end{tabular}
\caption{\label{tab:AddScale}Systematic error due to the additional scale factor of 1\% on the energy.}
  \end{center}
\end{table}


\subsection{Systematic Uncertainty Summary}

\begin{table}[htbp]
\begin{center}
\begin{tabular}{ccccccc}
    \toprule
\multicolumn{7}{c}{$\sigma(\mathcal{A}) \times 10^{-4}$}\\
 & Relative   & Electron  & Signal     & Charge & \ETm & Total \\
 & Efficiency & Scale/Res & Estimation & MisID  & Scale/Res & \\
\midrule 
\multicolumn{7}{c}{$\PT > 25$ \GeV}\\
$0.0<|\eta|<0.4$ & 70 & 11 & 16 &  2 &  0 &  73\\
$0.4<|\eta|<0.8$ & 70 &  9 & 19 &  3 &  0 &  73\\
$0.8<|\eta|<1.2$ & 70 & 19 & 26 &  3 &  0 &  77\\
$1.2<|\eta|<1.4$ & 70 & 47 & 33 &  8 &  0 &  90 \\
$1.6<|\eta|<2.0$ & 70 & 40 & 25 &  9 &  0 &  85\\
$2.0<|\eta|<2.4$ & 70 & 45 & 25 & 10 &  0 &  87\\
\midrule
\multicolumn{7}{c}{$\PT > 30$ \GeV}\\
$0.0<|\eta|<0.4$ & 70 &  7 & 16 &  2 &  0 &  72 \\
$0.4<|\eta|<0.8$ & 70 & 17 & 19 &  2 &  0 &  75 \\
$0.8<|\eta|<1.2$ & 70 & 26 & 26 &  3 &  0 &  79 \\
$1.2<|\eta|<1.4$ & 70 & 46 & 33 &  7 &  0 &  91 \\
$1.6<|\eta|<2.0$ & 70 & 53 & 25 &  8 &  0 &  92 \\
$2.0<|\eta|<2.4$ & 70 & 55 & 25 & 10 &  0 &  93 \\
\midrule 
\multicolumn{7}{c}{$\PT > 35$ \GeV}\\
$0.0<|\eta|<0.4$ & 70 & 14 & 16 &  2 &  0 & 73 \\
$0.4<|\eta|<0.8$ & 70 & 45 & 19 &  2 &  0 & 85 \\
$0.8<|\eta|<1.2$ & 70 & 44 & 26 &  3 &  0 & 87 \\
$1.2<|\eta|<1.4$ & 70 & 52 & 33 &  6 &  0 & 93 \\
$1.6<|\eta|<2.0$ & 70 & 51 & 25 &  7 &  0 & 94 \\
$2.0<|\eta|<2.4$ & 70 & 52 & 25 &  9 &  0 & 94 \\
\bottomrule
\end{tabular}
\caption{\label{tab:summarysyst}Summary of the systematic errors}
\end{center}
\end{table}

\section{Results}
The measurement of the electron charge assymetry is presented with three
different \pT cuts of 25, 30 and \unit{35}{\GeV}. 

The results of the electron charge asymmetry with a \pT cut of \unit{25}{\GeV}
are summarised in \TableRef{tab:results25} and shown in
\FigureRef{fig:results25}.

\begin{figure}[htbp]
  \begin{center}
  \includegraphics*[width=0.45\textwidth,angle=90]{Asym_25}
  \caption{\label{fig:asym25} Measured electron charge asymmetry corrected with predictions from CTEQ10W and MSTW08NNLO.}
  \end{center}
\end{figure}

\begin{table}[htbp]
\begin{center}
\begin{tabular}{crrrr}
    \toprule
$|\eta|$ range & $<|\eta|>$ & Data & CTEQ6.6 & MSTW \\
\midrule 
$0.0<|\eta|<0.4$ & 0.2 & $0.1541\pm0.0064\pm0.0073$ & $0.1502^{+0.0062}_{-0.0045}$ & $0.1296^{+0.0022}_{-0.0032}$\\
$0.4<|\eta|<0.8$ & 0.6 & $0.1666\pm0.0064\pm0.0073$ & $0.1682^{+0.0060}_{-0.0055}$ & $0.1458^{+0.0023}_{-0.0031}$\\
$0.8<|\eta|<1.2$ & 1.0 & $0.1728\pm0.0065\pm0.0077$ & $0.1944^{+0.0051}_{-0.0072}$ & $0.1737^{+0.0026}_{-0.0030}$\\
$1.2<|\eta|<1.4$ & 1.3 & $0.1895\pm0.0096\pm0.0090$ & $0.2216^{+0.0050}_{-0.0087}$ & $0.1976^{+0.0032}_{-0.0026}$\\
$1.6<|\eta|<2.0$ & 1.8 & $0.2331\pm0.0076\pm0.0085$ & $0.2672^{+0.0047}_{-0.0105}$ & $0.2454^{+0.0039}_{-0.0018}$\\
$2.0<|\eta|<2.4$ & 2.2 & $0.2670\pm0.0077\pm0.0087$ & $0.2821^{+0.0037}_{-0.0110}$ & $0.2619^{+0.0039}_{-0.0018}$\\
    \bottomrule
\end{tabular}
\caption{Measured electron charge asymmetry with predictions from CTEQ6.6 and MSTW PDFs.  36 Uncertainties on measured asymmetry are statistical and systematic respectivly and the Uncertainties on predictions are due to the uncertainties on the PDFs}
\label{tab:results25}
\end{center}
\end{table}

The results of the electron charge asymmetry with a \pT cut of \unit{30}{\GeV}
are summarised in \TableRef{tab:results30} and shown in
\FigureRef{fig:results30}.

\begin{figure}[htbp]
  \begin{center}
  \includegraphics*[width=0.45\textwidth,angle=90]{Asym_30}
  \caption{\label{fig:asym30}Measured electron charge asymmetry for lepton momentum $\PT>30$ with predictions from CTEQ10W and MSTW08NNLO.}
  \end{center}
\end{figure}

\begin{table}[htbp]
\begin{center}
\begin{tabular}{crrr}
    \toprule
$|\eta|$   & $<|\eta|>$ & \multicolumn{2}{c}{$\PT>30$ \GeV} \\
range                  &      & Data & Prediction                   \\
\midrule    
$0.0<|\eta|<0.4$ & 0.2 & $0.1330\pm0.0071\pm0.0072$ & $0.1331^{+0.0058}_{-0.0026}$\\
$0.4<|\eta|<0.8$ & 0.6 & $0.1501\pm0.0071\pm0.0075$ & $0.1501^{+0.0030}_{-0.0028}$\\
$0.8<|\eta|<1.2$ & 1.0 & $0.1508\pm0.0073\pm0.0079$ & $0.1713^{+0.0035}_{-0.0034}$\\
$1.2<|\eta|<1.4$ & 1.3 & $0.1651\pm0.0106\pm0.0091$ & $0.1947^{+0.0032}_{-0.0050}$\\
$1.6<|\eta|<2.0$ & 1.8 & $0.2082\pm0.0087\pm0.0092$ & $0.2417^{+0.0058}_{-0.0063}$\\
$2.0<|\eta|<2.4$ & 2.2 & $0.2451\pm0.0086\pm0.0093$ & $0.2625^{+0.0070}_{-0.0080}$\\
    \bottomrule
\end{tabular}
\caption{Measured electron charge asymmetry in lepton momentum $\PT>30$ \GeV
with predictions from CTEQ6.6.  Uncertainties on measured asymmetry are
statistical and systematic respectivly and the Uncertainties on predictions are
due to the uncertainties on the PDF}.
\label{tab:results30}
\end{center}
\end{table}

The results of the electron charge asymmetry with a \pT cut of \unit{35}{\GeV}
are summarised in \TableRef{tab:results35} and shown in
\FigureRef{fig:results35}.

\begin{figure}[htbp]
  \begin{center}
\includegraphics*[width=0.45\textwidth,angle=90]{Asym_35}
  \caption{\label{fig:asym35}Measured electron charge asymmetry for lepton momentum $\PT>35$ \GeV with predictions from CTEQ10W and MSTW08NNLO.}
  \end{center}
\end{figure}

\begin{table}[htbp]
\begin{center}
\begin{tabular}{crrr}
    \toprule
$|\eta|$ & $<|\eta|>$ & \multicolumn{2}{c}{$\PT>35$ \GeV}    \\
range                  &     &  Data                        & Prediction    \\
\midrule
$0.0<|\eta|<0.4$ & 0.2 & $0.1191\pm0.0085\pm0.0073$ & $0.1147^{+0.0025}_{-0.0023}$\\
$0.4<|\eta|<0.8$ & 0.6 & $0.1259\pm0.0084\pm0.0085$ & $0.1267^{+0.0024}_{-0.0026}$\\
$0.8<|\eta|<1.2$ & 1.0 & $0.1350\pm0.0087\pm0.0087$ & $0.1495^{+0.0034}_{-0.0036}$\\
$1.2<|\eta|<1.4$ & 1.3 & $0.1385\pm0.0128\pm0.0093$ & $0.1686^{+0.0043}_{-0.0043}$\\
$1.6<|\eta|<2.0$ & 1.8 & $0.1834\pm0.0105\pm0.0094$ & $0.2170^{+0.0057}_{-0.0069}$\\
$2.0<|\eta|<2.4$ & 2.2 & $0.2220\pm0.0105\pm0.0094$ & $0.2432^{+0.0081}_{-0.0089}$\\
    \bottomrule
\end{tabular}
\caption{Measured electron charge asymmetry in lepton momentum $\PT>35$ \GeV with predictions from CTEQ6.6.
Uncertainties on measured asymmetry are statistical and systematic respectivly and the
Uncertainties on predictions are due to the uncertainties on the PDF}
\label{tab:results35}
\end{center}
\end{table}

