\chapter{ 
Measurement of the electron charge asymmetry with \unit{36}{\invpb} }


In this chapter the first measurement of the electron charge asymmetry in
inclusive \inclusiveWe production with the CMS detecotr is described.

The analysis is performed on the full 2010 dataset, which corresponds to a
luminosity of \unit{36.1}{\invpb}.

%% TODO MC SAMPLES USED

\section{Event Selection}

The event selection used in this analysis is based on a limited number of cuts.
This is due to the limited statistics available in early data taking.

\subsection{Trigger}

Several triggers were used to select the events, due to the increasing
luminosity in the LHC 2010 run.

In the initial runs, events were selected using only a single photon trigger. 
As the luminosity increased these triggers became prescaled it was
necessary to use electron triggers to select events. 
As the luminosity increased even further, it was necesary to use electron
triggers that included cuts on electron ID variables.

%% TODO Trigger tables 
\begin{table}[htb]
  \centering
  \begin{tabular}{| l c r |}
    \hline
    1 & 2 & 3 \\
    4 & 5 & 6 \\
    7 & 8 & 9 \\
  \hline
  \end{tabular}
  \caption{A placeholder table}
  \label{asym36:triggers}
\end{table}

\subsection{Electron Selection}

The available unprescaled triggers constrain the electron momentum to be above
\unit{25}{\GeV}.

Electrons candidates are identified using a cut based approach on an limited
number of variables.

%% TODO finish ZZZzzzzz,,,,

\subsection{Event Selection}

An Event is selected if if contains a single electron that passes all electron
selection.
To removed Drell-Yan events an event is vetoed if it contasins a second lepton
(an electron passing a loose selecton, or an isolated muon) with $\pt > 
\unit{15}{\GeV}$

The Particle Flow \ETm distribution for the events that pass the event
selection, with $\pt > \unit{25}{\GeV}$ and $|\eta| < 2.4$ is shown in
\fig{asym36:pfmet}.
The number of selected events that pass the event selection, binned in charge 
and eta, are shown in \tab{asym36:selectedevents}. 

\begin{figure}[htb]
  \centering
  \includegraphics[width=0.5\textwidth]{placeholder}
  \caption{Particle Flow \ETm distribution for selected events.}
  \label{asym36:pfmet}
\end{figure}

\begin{table}[htb]
  \centering
  \begin{tabular}{| l c r |}
    \hline
    1 & 2 & 3 \\
    4 & 5 & 6 \\
    7 & 8 & 9 \\
  \hline
  \end{tabular}
  \caption{A placeholder table}
  \label{asym36:selectedevents}
\end{table}
                           
The exected composition of the selected events, derived from MC simulation
samples, is show in \tab{asym36:selectedcomp}. 

\begin{table}[htb]
  \centering
  \begin{tabular}{| l c r |}
    \hline
    1 & 2 & 3 \\
    4 & 5 & 6 \\
    7 & 8 & 9 \\
  \hline
  \end{tabular}
  \caption{A placeholder table}
  \label{asym36:selectedcomp}
\end{table}


\section{Signal Yield Extraction}

The number of signal and background events in each bin is extracted using a fit
to the \ETm distribution using two templates.
The first is the sum of the \Wenu signal and the \EWK background shapes,
and the second is the sum of the \QCD plus \gjet processes.

\subsection{\QCD \ETm Shape}

The \QCD and \gjet background distribution is obtained from a control sample of
events. The control sample is selected by requiring that the electrons pass the
isolation and H over E cuts but fail the $\Delta\phi$ and $\Delta\eta$ cuts as
shown in \tab{asym36:antisel}.

\begin{table}[htb]
  \centering
  \begin{tabular}{| l c r |}
    \hline
    1 & 2 & 3 \\
    4 & 5 & 6 \\
    7 & 8 & 9 \\
  \hline
  \end{tabular}
  \caption{A placeholder table}
  \label{asym36:antisel}
\end{table}

To validate the antiselection used, QCD Monte Carlo samples are used. The
distribution of QCD events passing the event selection are comapred to the
antselected MC events. This is shown for each eta bin in
\fig{asym36:antiselclosure}.

\begin{figure}[htb]
  \centering
  \includegraphics[width=0.5\textwidth]{placeholder}
  \caption{The \ETm distribution on antiselected Monte Carlo simulated events
  and selected \QCD and \gjet events in each pseudorapidity bin.}
  \label{asym36:antiselclosure}
\end{figure}

\subsection{Signal \ETm Shape from Boson Recoil}

The Signal \ETm shape was obtained using information from the recoil of the
boson
%TODO write

\subsection{\EWK \ETm Shape}

The \EWK background \ETm distributions were obtained from Pythia Monte Carlo
simulations.

%TODO write more

\subsection{Validation of Signal Extraction Method on Simulation}

The signal yield extraction procedure was validated using pseudodata
experiments. 1000 pseudodata experiments were generated with the number of
events expected in \unit{36.1}{\invpb} of data. The signal yields are extracted
in each experiment and the asymmetry is calculated. The distribution of
asymmetries is then fitted with a gaussian.

The width of the gaussian is the statistical uncertainty on the measurement.

The statistical uncertainty can aslo be estimated from the following formula

\begin{equation}
  blah
  \label{asym36:statuncert}
\end{equation}

The uncertainty from \eq{asym36:statuncert} evaulated with Monte Carlo truth
values, and the uncertainty measured from pseudodata experiments for an
integrated luminosity of \unit{36.1}{\invpb} are summarised in
%\table{asym36:statuncertsum}.

%\begin{table}[htb]
  %\centering
  %\begin{tabular}{| l c r |}
    %\hline
    %1 & 2 & 3 \\
    %4 & 5 & 6 \\
    %7 & 8 & 9 \\
  %\hline
  %\end{tabular}
  %\caption{A placeholder table}
  %\label{asym36:statuncertsum}
%\end{table}

\subsection{Fit on Real Data}

\section{Systematic Uncertainties}
\subsection{Relative Efficiency}
\subsection{Signal Extraction Method}
\subsubsection{QCD \ETm Shape}
\subsubsection{Signal \ETm Shape from Boson Recoil}
\subsubsection{EWK \ETm Shape}
\subsection{Charge Misassignment}
\subsection{Lepton Energy Scale and Resolution}
\subsection{\ETm Resolution}
\subsection{Systematic Uncertainty Summary}

% This sectionn may not be included
\section{Additional Systematic Studies}
\subsection{Cross Checks}
\subsubsection{Positive vs Negative Eta}
\subsubsection{Use of pure ECAL Energy Measurement}
\subsection{Asymmetric QCD Background}

\section{Results}
\subsection{Results in Bins of Lepton Momentum and \ETm}
\subsection{Correction for the Electron and \ETm Resolution Effect}


