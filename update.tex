\chapter[Update to the Electron Charge Asymmetry]{Update to the electron charge asymmetry
measurement with \unit{840}{\invpb} }
\label{chap:update}

The measurement detailed in the previous chapter was performed with
\unit{36}{\invpb} of data from the full 2010 dataset. 
In this chapter, the update to the measurement of the electron charge asymmetry in
inclusive \inclusiveWe production with \unit{840}{\invpb} is presented
\cite{asym840,bendavid2011electron}.
The data were collected with the {CMS} detector from collisions from the
first 2011 {LHC} run (Run A) and correspond to a nearly 25 times increase in
statistics over the 2010 measurement.

The majority of the 2011 analysis is identical to the 2010 analysis,
however small changes to the methodology have been implemented due to changes
in the instantaneous luminosity and the increased amount of data in the 2011 {LHC} run.

The results are presented in 11 bins of absolute value of pseudorapidity with a
fixed width of 0.2. To avoid the gap between the ECAL barrel and ECAL endcap the
region $1.4<|\eta|<1.6$ is excluded.

\section{Event Selection}

\subsection{Trigger}
\label{sec:trigger2}

The triggers used in the updated measurement are summarised in
\TableRef{tab:updatedtriggers} with the run ranges to which they are applied.
Due to the increased instantaneous luminosity in the 2011 Run A, the
identification and isolation cuts applied to the trigger have been tightened.
The \PT cut applied to the electron was increased, first to
\unit{27}{\GeV} and then in later runs to \unit{32}{\GeV}. 

\begin{table}[htbp]
  \begin{center}
    \leavevmode
     \begin{tabular}{ll} 
\toprule
      Run Ranges & Trigger  \\
     \midrule
     160404-161176 & HLT\_Ele27\_CaloIdVT\_CaloIsoT\_TrkIdT\_TrkIsoT\_v1  \\
     161217-163261 & HLT\_Ele32\_CaloIdVT\_CaloIsoT\_TrkIdT\_TrkIsoT\_v1  \\
     163270-163869 & HLT\_Ele32\_CaloIdVT\_CaloIsoT\_TrkIdT\_TrkIsoT\_v2  \\
     165088-165633 & HLT\_Ele32\_CaloIdVT\_CaloIsoT\_TrkIdT\_TrkIsoT\_v3  \\
     165970-166967 & HLT\_Ele32\_CaloIdVT\_CaloIsoT\_TrkIdT\_TrkIsoT\_v4  \\
\bottomrule
     \end{tabular}
  \caption{Triggers used to select the data used in this measurement.}
  \label{tab:updatedtriggers}
   \end{center}
\end{table}

\subsection{Electron Selection}

The electron transverse momentum cut is increased to \unit{35}{\GeV} due to the
constraints imposed by the high-level triggers available.  This is the only
change to the electron selection with respect to the 2010 analysis, detailed in
\TableRef{tab:electronselection}.

\subsection{Event Selection}
The event selection remains the same with respect to the 2010 measurement.
An event is selected if it contains a single electron that passes all electron
selections and it is vetoed if it contains a second charged lepton with
$\pT>\unit{15}{\GeV}$.

\begin{figure}[htbp]
  \centering
  \includegraphics*[width=0.8\textwidth]{pfmet_update}
  \caption{Particle Flow \ETm distribution for selected events that pass a
transverse momentum cut of \unit{35}{\GeV}.}
  \label{fig:pfmet}
\end{figure}

\begin{table}[htbp]
 \begin{center}
 \begin{tabular}{lcc}
\toprule
 $|\eta|$ range & Selected Positrons & Selected Electrons\\
 \midrule
 $0.0<| \eta |<0.2$ & 108235 &  90860 \\
 $0.2<| \eta |<0.4$ & 112870 &  93996 \\
 $0.4<| \eta |<0.6$ & 114148 &  94721 \\
 $0.6<| \eta |<0.8$ & 117099 &  95295 \\
 $0.8<| \eta |<1.0$ & 116956 &  94580 \\
 $1.0<| \eta |<1.2$ & 113336 &  90755 \\
 $1.2<| \eta |<1.4$ & 115265 &  91572 \\
 $1.6<| \eta |<1.8$ &  92137 &  70205 \\
 $1.8<| \eta |<2.0$ & 105596 &  80921 \\
 $2.0<| \eta |<2.2$ & 112419 &  86240 \\
 $2.2<| \eta |<2.4$ & 121224 & 102111 \\
\bottomrule
 \end{tabular}
 \caption[Number of events passing the event selection for a lepton momentum cut
 of \unit{$\PT>35$}{\GeV}]{Number of events passing the event selection for a lepton momentum cut
 of \unit{$\PT>35$}{\GeV} \cite{bendavid2011electron}.}
\label{tab:updatedselectedevents}
\end{center}
\end{table}

The particle flow \ETm distribution for the events that pass the event
selection, with $\PT > \unit{35}{\GeV}$ and $|\eta| < 2.4$ is shown in
\FigureRef{fig:pfmet}.
The numbers of selected events that pass the event selection are shown in
\TableRef{tab:updatedselectedevents}. 
The expected composition of the selected events, derived from MC simulations, is
shown in \TableRef{tab:updatedselectedcomp}. 

\begin{table}[htbp]
\begin{center}
\begin{tabular}{llr}
    \toprule
& & $\PT>\unit{35}{\GeV}$\\
\midrule
Signal & \HepProcess{\PW\to\Pe\Pnu} & $76.2\%$ \\
Electroweak & \HepProcess{\PZ\to\Ptau\Ptau} & $0.2\%$  \\
    & \HepProcess{\PZ\to\Pe\Pe}     & $6.4\%$  \\
    & \HepProcess{\PW\to\Ptau\Pnu}  & $0.8\%$  \\
    & \HepProcess{\Ptop\APtop}      & $0.4\%$  \\
    & Total                         & $7.8\%$  \\
QCD & Total                         & $16.0\%$ \\
\bottomrule
\end{tabular}
\caption[Composition of selected events for a lepton momentum cut of
$\PT>\unit{35}{\GeV}$.] {Composition of selected events for a lepton momentum
cut of $\PT>\unit{35}{\GeV}$. Numbers are evaluated using simulation
\cite{bendavid2011electron}.}
\label{tab:updatedselectedcomp}
\end{center}
\end{table}
\todo[inline]{remove this and replace with a sentance explaining it.}

\section{Signal Yield Extraction Method}
The number of signal and background events in each bin is extracted using a fit
to the \ETm distribution using two templates: the sum of the \Wenu signal and
the {electroweak} background shapes, and the {QCD} plus \gjet processes.

The {QCD} and \gjet background distribution is obtained from a control sample of
events that pass an anti-selection described in \TableRef{tab:antisel}.  The
\ETm distributions for the {electroweak} background were obtained from Madgraph
{MC} simulations with the Z2 tune\todo{reference}.  The signal \ETm shape is
obtained by modelling the recoil of the W boson. 

\subsection{Signal \ETm Shape from Boson Recoil}
The \ETm shape for signal is obtained by modelling the recoil of the W boson.  The
recoil response and resolution is measured in in \HepProcess{\PZ\to\Pe\Pe} data
events, and \PW and \PZ {MC} simulations to derive a correction to \ETm as a
function of \PW \pT \cite{bauer2010modeling,alcaraz2010updated}.

The methodology in deriving the details is described in \SectionRef{sec:recoil}.
The Gaussian used to fit the recoil components was modified to a double
Gaussian. 
The corrections were obtained by selecting \HepProcess{\PZ\to\Pe\Pe} events
using the same selection criteria as the analysis, but with the requirement of
two electrons instead of one.

\subsection{Fit on Real Data}
The results of the extended maximum likelihood fits to each pseudorapidity/charge
bin are shown in Figures \ref{fig:data1}, \ref{fig:data2}, \ref{fig:data3} and
\ref{fig:data4} in the appendix.
The signal yield in each bin is summarised in \TableRef{tab:updatedsigyield} and
the uncorrected asymmetry values are shown in \TableRef{tab:uncorRes}.
%the Kolmogorov-Smirnov probability of fit in \TableRef{tab:ks2}.

\begin{table}[htbp]
 \begin{center}
 \begin{tabular}{lcrr}
\toprule
$|\eta|$ range &  Charge &  N$_{QCD}$     & N$_{W\rightarrow e \nu}$  \\
               &         & fitted events & fitted events            \\
\midrule
$0.0<| \eta |<0.2$ &  $+$ & $12794.5 \pm 202.6$ &$89698.3\pm330.5$ \\
                   &  $-$ & $12926.1 \pm 194.3$ &$72369.8\pm297.6$ \\ 
$0.2<| \eta |<0.4$ &  $+$ & $13773.9 \pm 209.6$ &$93109.0\pm337.7$ \\
                   &  $-$ & $14071.4 \pm 199.8$ &$74215.4\pm302.0$ \\ 
$0.4<| \eta |<0.6$ &  $+$ & $14803.9 \pm 212.9$ &$93211.7\pm338.1$ \\
                   &  $-$ & $14554.2 \pm 203.5$ &$74360.3\pm303.4$ \\ 
$0.6<| \eta |<0.8$ &  $+$ & $15484.5 \pm 217.0$ &$94943.5\pm341.0$ \\
                   &  $-$ & $15523.8 \pm 206.4$ &$73764.2\pm302.2$ \\ 
$0.8<| \eta |<1.0$ &  $+$ & $17177.5 \pm 221.7$ &$92971.3\pm338.1$ \\
                   &  $-$ & $16871.6 \pm 210.5$ &$71476.6\pm298.3$ \\ 
$1.0<| \eta |<1.2$ &  $+$ & $20001.2 \pm 229.9$ &$86449.0\pm329.0$ \\
                   &  $-$ & $19930.1 \pm 218.4$ &$64643.8\pm286.6$ \\ 
$1.2<| \eta |<1.4$ &  $+$ & $24214.9 \pm 243.7$ &$83718.6\pm326.6$ \\
                   &  $-$ & $24425.4 \pm 233.0$ &$60683.8\pm281.4$ \\ 
$1.6<| \eta |<1.8$ &  $+$ & $15759.0 \pm 228.6$ &$68726.0\pm302.3$ \\
                   &  $-$ & $16093.2 \pm 220.1$ &$47154.5\pm256.2$ \\ 
$1.8<| \eta |<2.0$ &  $+$ & $23809.6 \pm 241.7$ &$73116.7\pm305.0$ \\
                   &  $-$ & $23135.2 \pm 234.7$ &$49577.2\pm257.0$ \\ 
$2.0<| \eta |<2.2$ &  $+$ & $33773.0 \pm 261.7$ &$69872.7\pm299.1$ \\
                   &  $-$ & $32104.6 \pm 253.1$ &$45913.9\pm248.8$ \\ 
$2.2<| \eta |<2.4$ &  $+$ & $61794.8 \pm 309.3$ &$52495.9\pm269.8$ \\
                   &  $-$ & $60449.1 \pm 306.6$ &$34746.0\pm228.7$ \\ 
\bottomrule
 \end{tabular}
 \caption[The signal and {QCD} background yields with the corresponding
statistical uncertainty]{\label{tab:updatedsigyield} The signal and {QCD}
background yields with the corresponding statistical
uncertainty\cite{bendavid2011electron}.}
 \end{center}
\end{table}
\todo{chi2 here?}

%\begin{table}[htbp]
%\begin{center}
%\begin{tabular}{lrr}
%\toprule
%$|\eta|$ range &  \multicolumn{2}{c}{KS probability of fit} \\
               %& Positrons & Electrons \\
%\midrule
%$0.0<| \eta |<0.2$ & 0.999 & 0.999 \\ 
%$0.2<| \eta |<0.4$ & 0.999 & 0.999 \\ 
%$0.4<| \eta |<0.6$ & 0.999 & 0.872 \\ 
%$0.6<| \eta |<0.8$ & 0.999 & 0.971 \\ 
%$0.8<| \eta |<1.0$ & 0.950 & 0.958 \\ 
%$1.0<| \eta |<1.2$ & 0.770 & 0.676 \\ 
%$1.2<| \eta |<1.4$ & 0.985 & 0.857 \\ 
%$1.6<| \eta |<1.8$ & 0.480 & 0.797 \\ 
%$1.8<| \eta |<2.0$ & 0.915 & 0.416 \\ 
%$2.0<| \eta |<2.2$ & 0.581 & 0.575 \\ 
%$2.2<| \eta |<2.4$ & 0.386 & 0.274 \\ 
%\bottomrule
%\end{tabular}
%\caption{\label{tab:ks2} Kolmogorov-Smirnov probability of fit\cite{bendavid2011electron}.}
%\end{center}
%\end{table}

\begin{table}[htbp]
  \begin{center}
    \begin{tabular}{lc}
\toprule
    $|\eta|$ range & $\mathcal{A}_{exp} (\times 10^{-3})$\\
    \midrule
    $0.0<|\eta|<0.4$ & 106.9 $\pm$ 2.7\\
    $0.2<|\eta|<0.4$ & 112.9 $\pm$ 2.7\\
    $0.4<|\eta|<0.6$ & 112.5 $\pm$ 2.7\\
    $0.6<|\eta|<0.8$ & 125.5 $\pm$ 2.7\\
    $0.8<|\eta|<1.0$ & 130.7 $\pm$ 2.7\\
    $1.0<|\eta|<1.2$ & 144.3 $\pm$ 2.9\\
    $1.2<|\eta|<1.4$ & 159.5 $\pm$ 3.0 \\
    $1.6<|\eta|<1.8$ & 186.1 $\pm$ 3.4\\
    $1.8<|\eta|<2.0$ & 191.8 $\pm$ 3.2\\
    $2.0<|\eta|<2.2$ & 206.9 $\pm$ 3.3\\
    $2.2<|\eta|<2.4$ & 203.4 $\pm$ 4.1\\
\bottomrule
    \end{tabular}
  \caption[Uncorrected values of the charge asymmetry for an integrated
luminosity of \unit{840}{\invpb}.]{\label{tab:uncorRes}Uncorrected values
($\times 10^{-3}$) of the charge asymmetry  for an integrated luminosity of
\unit{840}{\invpb}\cite{bendavid2011electron}.}
  \end{center}
\end{table}

\section{Corrections}
\label{sec:corrections2}

To compare with the theoretical value, the measured lepton asymmetry has to be
corrected for the charge-dependent reconstruction efficiency, the incorrect charge
assignment rate and the effect of the electron energy resolution.

\subsection{Relative Efficiency}
The relative efficiency is again measured using a tag and probe method \cite{adam2009tag}.
The {GSF} tracking efficiency,
identification efficiency and the {HLT} efficiency are measured individually
for each bin of pseudorapidity and for electrons and positrons separately. 

The {GSF} tracking efficiency, identification efficiency and the {HLT}
efficiency are summarised in \TableRef{tab:updatedefficiency}
in each pseudorapidity bin, as well as inclusively
across the range $0<| \eta |< 2.4$. 

The main systematic errors on the measurements of the efficiencies are the
energy scale and the signal shape. It was verified that these systematic effects
cancel in the measurement of the ratio of the efficiencies, $R$, and are
negligible with respect to the statistical error on the measurement
\cite{bendavid2011electron}.  Only the statistical error on the measurement is
propagated to the error on the measurement of the ratio of the efficiencies.
Unlike in the previous chapter the statistical error on the measurement of $R$
in each pseudorapidity bin is propagated to the asymmetry as a systematic error. 

\begin{table}[htbp]
\begin{center}
\begin{tabular}{lcrrrr}
\toprule
$|\eta|$ range & Charge & $\epsilon_{GSF}$ &$\epsilon_{ID}$&$\epsilon_{HLT}$& R\\
\midrule
$0.0<| \eta |<0.2$ &+& 98.5$\pm$0.2 &83.6$\pm$0.4 &97.9$\pm$0.2 &1.003$\pm$0.009\\
                   &-& 98.4$\pm$0.2 &83.5$\pm$0.5 &97.8$\pm$0.2 & \\
$0.2<| \eta |<0.4$ &+& 98.9$\pm$0.2 &82.6$\pm$0.5 &97.9$\pm$0.2 & 0.998$\pm$0.009\\
                   &-& 98.8$\pm$0.2 &82.8$\pm$0.5 &98.0$\pm$0.2 & \\
$0.4<| \eta |<0.6$ &+& 98.9$\pm$0.2&83.6$\pm$0.5 &98.1$\pm$0.2 & 0.995$\pm$0.009\\
                   &-& 99.0$\pm$0.2 &84.1$\pm$0.5 &97.9$\pm$0.2 & \\
$0.6<| \eta |<0.8$ &+& 98.7$\pm$0.2 &83.7$\pm$0.5 &98.4$\pm$0.2 & 0.999$\pm$0.009\\
                   &-& 98.7$\pm$0.2 &84.0$\pm$0.5&98.2$\pm$0.2 & \\
$0.8<| \eta |<1.0$ &+& 98.6$\pm$0.2 &84.3$\pm$0.5 &97.3$\pm$0.2 & 0.998$\pm$0.009\\
                   &-& 98.3$\pm$0.2 &84.4$\pm$0.5 &97.7$\pm$0.2 & \\
$1.0<| \eta |<1.2$ &+& 98.2$\pm$0.2 &82.6$\pm$0.5 &97.6$\pm$0.2 & 1.010$\pm$0.010\\
                   &-& 98.2$\pm$0.2 &81.7$\pm$0.5 &97.7$\pm$0.2 & \\
$1.2<| \eta |<1.4$ &+& 97.4$\pm$0.2 &79.4$\pm$0.5 &97.3$\pm$0.2 & 1.005$\pm$0.011\\
                   &-& 97.5$\pm$0.2 &78.8$\pm$0.6 &97.5$\pm$0.2 & \\
$1.6<| \eta |<1.8$ &+& 97.3$\pm$0.3 &62.3$\pm$0.7 &97.1$\pm$0.3 & 1.032$\pm$0.019\\
                   &-& 96.9$\pm$0.3 &61.2$\pm$0.7 &96.2$\pm$0.4 & \\
$1.8<| \eta |<2.0$ &+& 97.3$\pm$0.3 &62.1$\pm$0.8 &97.6$\pm$0.3 & 0.976$\pm$0.018\\
                   &-& 97.4$\pm$0.3 &63.7$\pm$0.8 &97.4$\pm$0.3 & \\
$2.0<| \eta |<2.2$ &+& 97.5$\pm$0.3 &58.6$\pm$0.9 &98.5$\pm$0.3 & 0.966$\pm$0.021\\
                   &-& 97.3$\pm$0.3 &60.7$\pm$0.8 &98.6$\pm$0.3 & \\
$2.2<| \eta |<2.4$ &+& 95.9$\pm$0.4 &55.1$\pm$1.0 &97.5$\pm$0.4 & 0.967$\pm$0.026\\
                   &-& 96.5$\pm$0.4 &56.3$\pm$1.0 &98.0$\pm$0.4 & \\
\midrule
$0.0<| \eta |<1.4$ &+& 98.42$\pm$0.07 &82.9$\pm$0.2 &97.82$\pm$0.08 & 0.999$\pm$0.004\\
                   &-& 98.51$\pm$0.07 &82.9$\pm$0.2 &97.82$\pm$0.08 & \\
$1.6<| \eta |<2.4$ &+& 97.12$\pm$0.15 &60.1$\pm$0.4 &97.63$\pm$0.17 & 0.987$\pm$0.010\\
                   &-& 97.12$\pm$0.15 &61.0$\pm$0.4 &97.42$\pm$0.17 & \\
\midrule
$0.0<| \eta |<2.4$ &+& 98.09$\pm$0.06 &77.41$\pm$0.17 &97.76$\pm$0.07 & 0.999$\pm$0.003\\
                   &-& 98.16$\pm$0.06 &77.49$\pm$0.17 &97.73$\pm$0.07 & \\
\bottomrule
\end{tabular}
\end{center}
\caption[GSF tracking, identification and HLT efficiency as a function of
charge.] {\label{tab:updatedefficiency} GSF tracking, identification and HLT
efficiency as a function of charge\cite{bendavid2011electron}.}
\end{table}

\subsection{Incorrect Charge Assignment}
The rate that electrons and positrons are incorrectly assigned charge is measured
in data from the same-sign and opposite-sign \PZ yields using the same electron
selection as in the asymmetry measurement.

The \HepProcess{\PZ\to\Pe\Pe} sample is split into 66 sub-samples, representing
the combinations of pseudorapidity bins of the two electrons.  For each \PZ
event with two electrons in pseudorapidity bin $i$ and $j$ respectively, the
probability of an event being same-sign or opposite-sign is given by the
\EquationRef{eq:chargepdf}. The incorrect charge rates, $\omega_i$, are
extracted from a simultaneous fit to the 66 samples.  The resulting values of
$\omega$ are shown in \TableRef{tab:mischarge}.
The statistical uncertainty on $\omega$ is propagated to the
measurement of the asymmetry as a systematic uncertainty.

\begin{table}[htbp]
  \begin{center}
\begin{tabular}{lr}
\toprule
$\eta$ range        & $\omega \times 10^{-4}$    \\
\midrule
$0.0<| \eta |<0.2$  & $ 1 \pm 1 $    \\ 
$0.2<| \eta |<0.4$  & $ 1 \pm 1 $    \\
$0.4<| \eta |<0.6$  & $ 1 \pm 1 $    \\
$0.6<| \eta |<0.8$  & $ 2 \pm 1 $    \\
$0.8<| \eta |<1.0$  & $ 4 \pm 2 $    \\ 
$1.0<| \eta |<1.2$  & $ 3 \pm 2 $    \\
$1.2<| \eta |<1.4$  & $ 3 \pm 2 $    \\
$1.6<| \eta |<1.8$  & $17 \pm 4 $    \\
$1.8<| \eta |<2.0$  & $12 \pm 4 $    \\
$2.0<| \eta |<2.2$  & $26 \pm 7 $    \\
$2.2<| \eta |<2.4$  & $ 7 \pm 7 $    \\
\bottomrule
\end{tabular}
\caption[Incorrect charge assignment rates and related systematic
uncertainties.]{\label{tab:mischarge}Incorrect charge assignment rates and
related systematic uncertainties\cite{bendavid2011electron}.}
\end{center}
\end{table}

\subsection{Lepton Energy Scale and Resolution}
The effect of the electron energy resolution on the asymmetry measurement is
studied using a {MC} study. A correction to the asymmetry can be evaluated by
comparing the asymmetry measured at the generator level and the asymmetry
measured in {MC} after the simulation of detector effects.

The correction relies on the simulation reproducing the energy scale and
\todo{more needed here  }
resolution of the electrons perfectly. Unfortunately the simulation does not
take into account all the detector effects present in real data, so additional
residual corrections to the simulated electron \pT distributions are necessary,
similar to the ad-hoc corrections in \SectionRef{sec:adhoc}. 
An additional Gaussian smearing of the \pT distribution ($\sigma_{corr}$) and
electron energy scale corrections ($k_{corr}$) for each pseudorapidity bin are
applied\cite{bauer2011higgs}.

The values for $k_{i}$ and $\sigma_{i}$ are obtained from the
\HepProcess{\PZ\to\Pe\Pe} mass distribution. The \HepProcess{\PZ\to\Pe\Pe}
sample is again split into 66 sub-samples for the combinations of
pseudorapidity bins of the two electrons for real data
and {MC}.  The \HepProcess{\PZ\to\Pe\Pe} mass distribution is then fitted with a
Breit-Wigner function convoluted with a Crystal Ball function that represents
the resolution function, instead of the simple Gaussian used previously. The
width and the mass of the Breit-Wigner function are taken from the PDG values
\cite{beringer2012review}. The resolution of the Crystal Ball function is
defined as,
\begin{equation}
\sigma_{ij} = \sqrt{\sigma_{i}^{2} + \sigma_{j}^{2}} ,
\end{equation}
where $\sigma_{i}$ is the resolution in each individual pseudorapidity bin.
The mean value of the Crystal Ball function is defined as,
\begin{equation}
m_{0_{ij}} = m_{\PZ}^{PDG} 
          \left( \sqrt{k_{i} k_{j}} - 1 \right) ,
\end{equation}
where $k_{i}$ is the scale factor in each pseudorapidity bin.  The 11 scale
factors $k_{i}$ and the 11 additional smearing factors $\sigma_{i}$ for each
pseudorapidity bin are obtained from a simultaneous fit to each of the mass
distributions from 66 sub-samples to both {MC} and data. 

To have a good agreement between data and simulation the simulated \pT
distribution of the leptons must be smeared with a Gaussian of width,
\begin{equation}
\sigma^{corr}_{i} = 
\sqrt{
\left(\sigma^{data}_{i}\right)^{2} -
\left(\sigma^{MC}_{i}  \right)^{2} 
},
\end{equation}
and then scaled by a factor,
\begin{equation}
k^{corr}_i = \frac{ k^{data}_i}{k^{MC}_i} ,
\end{equation}
The values of the
residual corrections, obtained with this method are given in
\TableRef{tab:ResCorr}.

\begin{table}[htbp]
  \begin{center}
\begin{sideways}
    \begin{tabular}{ccccccc}
\toprule
$\eta$ bin & $\sigma_{DATA}$ &  $\sigma_{MC}$ & $\sigma_{corr}$ & $k_{DATA}$ &  $k_{MC}$ & $k_{corr}$ \\
&GeV &GeV &GeV & & & \\
\midrule
0.0$<|\eta|<$0.2 &1.08$\pm$ 0.03 & 0.83$\pm$ 0.03 & 0.70$\pm$ 0.06 & 1.0007$\pm$ 0.0003 & 1.0012$\pm$ 0.0003 & 0.9995$\pm$ 0.0004 \\ 
0.2$<|\eta|<$0.4 &1.01$\pm$ 0.03 & 0.85$\pm$ 0.02 & 0.54$\pm$ 0.07 & 1.0036$\pm$ 0.0003 & 1.0020$\pm$ 0.0003 & 1.0016$\pm$ 0.0004 \\ 
0.4$<|\eta|<$0.6 &1.16$\pm$ 0.03 & 0.95$\pm$ 0.02 & 0.67$\pm$ 0.06 & 1.0019$\pm$ 0.0003 & 1.0006$\pm$ 0.0003 & 1.0013$\pm$ 0.0004 \\ 
0.6$<|\eta|<$0.8 &1.16$\pm$ 0.03 & 0.98$\pm$ 0.02 & 0.61$\pm$ 0.07 & 1.0039$\pm$ 0.0003 & 1.0008$\pm$ 0.0003 & 1.0031$\pm$ 0.0004 \\ 
0.8$<|\eta|<$1.0 &1.28$\pm$ 0.03 & 1.13$\pm$ 0.02 & 0.61$\pm$ 0.08 & 1.0003$\pm$ 0.0004 & 0.9987$\pm$ 0.0003 & 1.0016$\pm$ 0.0005 \\ 
1.0$<|\eta|<$1.2 &1.68$\pm$ 0.03 & 1.42$\pm$ 0.02 & 0.90$\pm$ 0.07 & 0.9906$\pm$ 0.0004 & 0.9941$\pm$ 0.0003 & 0.9964$\pm$ 0.0005 \\ 
1.2$<|\eta|<$1.4 &2.02$\pm$ 0.03 & 1.62$\pm$ 0.02 & 1.21$\pm$ 0.06 & 0.9859$\pm$ 0.0005 & 0.9962$\pm$ 0.0004 & 0.9896$\pm$ 0.0006 \\ 
1.6$<|\eta|<$1.8 &2.78$\pm$ 0.04 & 2.26$\pm$ 0.03 & 1.62$\pm$ 0.08 & 0.9977$\pm$ 0.0007 & 0.9797$\pm$ 0.0005 & 1.0184$\pm$ 0.0009 \\ 
1.8$<|\eta|<$2.0 &2.38$\pm$ 0.04 & 1.83$\pm$ 0.03 & 1.53$\pm$ 0.07 & 0.9976$\pm$ 0.0006 & 0.9768$\pm$ 0.0005 & 1.0213$\pm$ 0.0008 \\ 
2.0$<|\eta|<$2.2 &2.16$\pm$ 0.04 & 1.49$\pm$ 0.03 & 1.56$\pm$ 0.06 & 0.9922$\pm$ 0.0006 & 0.9837$\pm$ 0.0005 & 1.0087$\pm$ 0.0008 \\ 
2.2$<|\eta|<$2.4 &2.21$\pm$ 0.04 & 1.39$\pm$ 0.04 & 1.72$\pm$ 0.06 & 0.9510$\pm$ 0.0008 & 0.9881$\pm$ 0.0005 & 0.9625$\pm$ 0.0009 \\ 
\bottomrule
    \end{tabular}
\end{sideways}
    \caption[Scale and residual smearing factors measured in $Z\rightarrow ee$
data.] {\label{tab:ResCorr}Scale and residual smearing factors measured in
$Z\rightarrow ee$ data\cite{bendavid2011electron}. }
  \end{center}
\end{table}

%\begin{figure}[htbp]
  %\begin{center}
%\includegraphics*[width=0.45\textwidth]{ENSmeChargeSep.png}
%\includegraphics*[width=0.45\textwidth]{ENKfactChargeSep.png}
 %\caption{\label{Scale_sep} $\sigma_{corr}$(left) and $k_{corr}$(right) for electrons and positrons.}
%\end{center}
%\end{figure}
%% this is probably not needed here...

A {MC} sample of \HepProcess{\PW\to\Pe\Pnue} events is used to evaluate the
corrections introduced by the effect of the energy scale and resolution for each
pseudorapidity bin. The generator level electron \pT is first smeared by the
nominal \pT resolution of the simulated sample after the simulation of detector
effects. It is then smeared and scaled according to the residual energy
corrections, $\sigma^{corr}$ and $k^{corr}$. The \pT cut is then applied to the
generator and corrected samples, and the asymmetry corrections are obtained by
comparing the number of events that pass the threshold in either sample.

There are three sources of systematic uncertainty that affect the evaluated
correction, the {PDF} uncertainty on the generated \pT spectra, the modelling
of {FSR} and the uncertainty on the residual correction. The {PDF}
uncertainty is evaluated by generating a \pT spectrum for each error {PDF}  in
the {PDF} set and summing in quadrature the variations. The uncertainty on
the residual correction is evaluated by scaling and smearing the \pT spectra
within the uncertainty taking the maximum distance from a central value as the
uncertainty. 
\todo{FSR?}

The correction factor introduced by the electron energy scale and resolution and
the accompanying uncertainties are described in \TableRef{tab:energyscalecorr}.

\begin{table}[htbp]
  \begin{center}
    \begin{tabular}{cccccc}
\toprule
$\eta$ range & Correction  & Stat.   &PDF  Syst. &  FSR Syst. & Scale Syst.\\
          & factor & Error & Error   & Error  & Error  \\
     \midrule
 $0.0<|\eta|<0.2$ & $-1.1$ & 0.1 & 0.3  &0.1 & 0.0\\
 $0.2<|\eta|<0.4$ & $ 0.2$ & 0.1 & 0.6  &0.0 & 0.0\\
 $0.4<|\eta|<0.6$ & $-0.4$ & 0.1 & 0.3  &0.0 & 0.0\\
 $0.6<|\eta|<0.8$ & $-0.9$ & 0.1 & 0.3  &0.0 & 0.0\\
 $0.8<|\eta|<1.0$ & $-0.9$ & 0.1 & 0.6  &0.0 & 0.0\\
 $1.0<|\eta|<1.2$ & $-1.4$ & 0.1 & 1.0  &0.0 & 0.0\\
 $1.2<|\eta|<1.4$ & $-3.5$ & 0.1 & 0.8  &0.0 & 0.1\\
 $1.6<|\eta|<1.8$ & $-2.5$ & 0.1 & 0.8  &0.0 & 0.0\\
 $1.8<|\eta|<2.0$ & $-4.4$ & 0.1 & 1.6  &0.0 & 0.0\\
 $2.0<|\eta|<2.2$ & $-0.2$ & 0.1 & 2.6  &0.0 & 0.1\\
 $2.2<|\eta|<2.4$ & $-3.5$ & 0.1 & 2.4  &0.0 & 0.2\\
\bottomrule
    \end{tabular}
    \caption[Bias in the charge asymmetry introduced by the electron energy
scale and resolution.] {\label{tab:energyscalecorr}Bias in the charge asymmetry
introduced by the electron energy scale and resolution.  All values are in units
$\times 10^{-3}$\cite{bendavid2011electron}.}
  \end{center}
\end{table}
\todo{why is the stat error all equal?}

\subsection{Correction Factors}
The summary of each of the correction factors applied to the asymmetry
measurement in each pseudorapidity bin is given in
\TableRef{tab:correctionfactors}. At central pseudorapidities the corrections are
small, but at larger pseudorapidities the corrections due to the incorrect
charge assignment become larger.

\begin{table}[htbp]
  \begin{center}
    \begin{tabular}{ccccc}
\toprule
$\eta$ range & $\mathcal{A}_M$ & Rel. Eff & Energy & $\mathcal{A}_C$ \\
&                              & MisCharge & Resolution &  \\
&                              & Correction  & Correction & \\
\midrule
 $0.0<|\eta|<0.4$ & 106.9 &$- 1.5\pm 4.5$ & $-1.1\pm0.3$ & 104.3\\ 
 $0.2<|\eta|<0.4$ & 112.9 &$+ 1.0\pm 4.4$ & $ 0.2\pm0.6$ & 114.1\\ 
 $0.4<|\eta|<0.6$ & 112.5 &$+ 2.5\pm 4.4$ & $-0.4\pm0.3$ & 114.6\\
 $0.6<|\eta|<0.8$ & 125.5 &$+ 0.5\pm 4.4$ & $-0.9\pm0.3$ & 125.1\\ 
 $0.8<|\eta|<1.0$ & 130.7 &$+ 0.2\pm 4.4$ & $-0.9\pm0.6$ & 130.0\\ 
 $1.0<|\eta|<1.2$ & 144.3 &$- 4.8\pm 4.9$ & $-1.4\pm1.0$ & 138.1\\ 
 $1.2<|\eta|<1.4$ & 159.5 &$- 2.3\pm 5.4$ & $-3.5\pm0.8$ & 153.7\\ 
 $1.6<|\eta|<1.8$ & 186.1 &$-14.9\pm 9.2$ & $-2.5\pm0.8$ & 168.7\\
 $1.8<|\eta|<2.0$ & 191.8 &$+12.0\pm 8.7$ & $-4.4\pm1.6$ & 199.5\\
 $2.0<|\eta|<2.2$ & 206.9 &$+17.4\pm10.1$ & $-0.2\pm2.6$ & 224.1\\
 $2.2<|\eta|<2.4$ & 203.4 &$+16.1\pm12.5$ & $-3.5\pm2.4$ & 216.0\\
\bottomrule
    \end{tabular}
    \caption[Summary of the corrections that are applied to the measured
asymmetry.]{\label{tab:correctionfactors}Summary of the corrections ($\times
10^{-3}$) that are applied to the measured asymmetry\cite{bendavid2011electron}.}
  \end{center}
\end{table}


\section{Systematic Uncertainties}
The systematic uncertainties due to the relative efficiency, the incorrect
charge assignment and the electron energy scale and resolution are included in
the uncertainties on the correction factors applied to the asymmetry that were
described in the preceding section.

\subsection{Signal Extraction Method}
Biases may be introduced into the measurement if there is a
difference between the \ETm templates used and the true \ETm distribution. 
The systematic uncertainties due to the signal extraction method are evaluated
in a similar way to the measurement described in the preceding chapter.  The
template shapes used in the signal yield extraction method are varied within
certain uncertainties. 

\subsubsection{QCD \ETm Shape}
The systematic uncertainty due to the {QCD} and \gjet \ETm shape is evaluated
by varying the anti-selection that is used to obtain the control sample. The
effect that changing the values of the inverted $\Delta\eta$ and $\Delta\phi$
cuts used in anti-selection has on the asymmetry is observed and the maximum
distance from the measured asymmetry with the nominal anti-selection is used as a
measure of the systematic uncertainty.

\TableRef{tab:updatedsystQCD} summarises the systematic uncertainty due to the
{QCD} \ETm shape for each pseudorapidity bin.

\begin{table}[htbp]
\begin{center}
\begin{tabular}{cr}
\toprule
$|\eta|$ range  & $\sigma(\mathcal{A}) \times 10^{-3}$\\
\midrule
$0.0<|\eta|<0.2$ & 1.0\\
$0.2<|\eta|<0.4$ & 1.9\\
$0.4<|\eta|<0.6$ & 2.2\\
$0.6<|\eta|<0.8$ & 1.8\\
$0.8<|\eta|<1.0$ & 0.5\\
$1.0<|\eta|<1.2$ & 1.5\\
$1.2<|\eta|<1.4$ & 1.4\\
$1.6<|\eta|<1.8$ & 2.5\\
$1.8<|\eta|<2.0$ & 1.3\\
$2.0<|\eta|<2.2$ & 0.6\\
$2.2<|\eta|<2.4$ & 1.6\\
\bottomrule
\end{tabular}
\caption[Maximum distance between the asymmetry measured with many different
anti-selections and the asymmetry measured with the nominal anti-selection in MC
pseudo-data.]{Maximum distance between the asymmetry measured with
many different anti-selections and the asymmetry measured with the nominal
anti-selection in MC pseudo-data for each $\eta$
bin\cite{bendavid2011electron}.}

\label{tab:updatedsystQCD}
\end{center}
\end{table}

\subsubsection{Signal \ETm Shape from Boson Recoil}
The two sources of systematic error due to the signal \ETm shape are considered:
the uncertainties on the recoil method and the uncertainty due to the choice of
PDF used as an input to the recoil method.

The uncertainties on the recoil method corrections are propagated to the
asymmetry measurement as systematic uncertainties.  The corrections obtained
from the recoil method vary between an upper and lower
value\cite{bauer2010modeling}. The effect of these uncertainties on the
asymmetry measurement is evaluated by using the upper and lower limits on the
corrections to observe the effect that they have on the measured asymmetry, and
use this as a measure of the systematic uncertainty.

The systematic uncertainty due to the {PDF} uncertainty used to generate the
template shapes is evaluated by considering the {PDF} error sets. The
CTEQ6.6\cite{lai2010vv} error {PDF} sets have 45 individual PDFs which represent
the central value {PDF}, 22 upper and 22 lower error {PDF} for each of the 22
parameters the CTEQ collaboration use in their global fits to data.  The
uncertainty is evaluated by generating template shapes for each of the
parameters and measuring the effect of these templates on the measured
asymmetry. The effects due to each parameter are combined using the
prescription in \SectionRef{sec:asymuncert} and taken as a measure of the
systematic uncertainty.

The systematic uncertainty due to the signal \ETm signal shape is summarised in
\TableRef{tab:updatedsystsig}.

\begin{table}[htbp]
\begin{center}
\begin{tabular}{crrr}
\toprule
$|\eta|$  & \multicolumn{3}{c}{$\sigma(\mathcal{A}) \times 10^{-3}$}\\
range     & Recoil Corr.& PDF & Combined \\
\midrule
$0.0<|\eta|<0.2$ &  0.2 &  1.5  & 1.5 \\
$0.2<|\eta|<0.4$ &  0.4 &  1.6  & 1.7 \\
$0.4<|\eta|<0.6$ &  0.6 &  1.3  & 1.4 \\
$0.6<|\eta|<0.8$ &  0.8 &  1.5  & 1.7 \\
$0.8<|\eta|<1.0$ &  0.9 &  1.6  & 1.8 \\
$1.0<|\eta|<1.2$ &  0.8 &  1.7  & 1.9 \\
$1.2<|\eta|<1.4$ &  1.5 &  1.6  & 2.3 \\
$1.6<|\eta|<1.8$ &  0.7 &  1.7  & 1.9 \\
$1.8<|\eta|<2.0$ &  0.4 &  1.5  & 1.6 \\
$2.0<|\eta|<2.2$ &  1.1 &  1.5  & 1.9 \\
$2.2<|\eta|<2.4$ &  0.9 &  2.3  & 2.4 \\
\bottomrule
\end{tabular}
\caption[Systematic uncertainty due to the signal \MET shape used in the signal
extraction method.] {\label{tab:updatedsystsig} Systematic uncertainty due to
the signal \MET shape used in the signal extraction method assigned to each
$\eta$ bin\cite{bendavid2011electron}.}
\end{center}
\end{table}

\subsubsection{Electroweak \ETm Shape}

The {electroweak} \ETm shape is obtained from {MC} samples. During the fitting
procedure, the {electroweak} shape is fixed to the \Wenu signal shape according to
the cross section from NLO {MC}.

To estimate the effect of the uncertainty of the cross section has on the
asymmetry measurement, the {electroweak} background is varied by
the measured uncertainty on the relative \PW to \PZ cross sections of
$\pm2\%$. The effect on the asymmetry is measured and found to be
negligible \cite{bendavid2011electron}.

\subsection{Systematic Uncertainty Summary}
The summary of the systematic uncertainties due to the signal yield method, the
electron energy scale and resolution, the incorrect assignment of charge and the
relative efficiency of electrons and positrons are given in
\TableRef{tab:updatedsummarysyst}.

A full correlation matrix of the systematic errors in the measurement is
presented in \TableRef{tab:corr_tot}.

\begin{table}[htbp]
 \begin{center}
   \begin{tabular}{lcccc}
\midrule
      &Signal & Energy & Charge &  Efficiency \\
     & Yield & Scale and Res. & MisId. & Ratio \\ \midrule
$0.0<|\eta|<0.2$ & 1.8 & 0.6 & 0.0 &  4.5 \\
$0.2<|\eta|<0.4$ & 2.5 & 0.6 & 0.0 &  4.4 \\
$0.4<|\eta|<0.6$ & 2.7 & 0.3 & 0.0 &  4.4 \\
$0.6<|\eta|<0.8$ & 2.5 & 0.3 & 0.0 &  4.4 \\
$0.8<|\eta|<1.0$ & 1.9 & 0.6 & 0.1 &  4.4 \\
$1.0<|\eta|<1.2$ & 2.4 & 1.0 & 0.1 &  4.9 \\
$1.2<|\eta|<1.4$ & 2.6 & 0.8 & 0.1 &  5.4 \\
$1.6<|\eta|<1.8$ & 3.1 & 0.8 & 0.1 &  9.2 \\
$1.8<|\eta|<2.0$ & 2.0 & 1.6 & 0.2 &  8.7 \\
$2.0<|\eta|<2.2$ & 2.0 & 2.6 & 0.3 & 10.0 \\
$2.2<|\eta|<2.4$ & 2.9 & 2.4 & 0.3 & 12.5 \\
\bottomrule
    \end{tabular}
  \end{center}
 \caption[Summary of the systematic uncertainties.]
{\label{tab:updatedsummarysyst}Summary of the systematic uncertainties.  All
values are in units $\times 10^{-3}$\cite{bendavid2011electron}. }
\end{table}
                  

\begin{table}[htbp]
   \begin{center}
      \begin{sideways}
      \begin{tabular}{l|ccccccccccc}
\midrule
& \scriptsize{$\left[0.0,0.2\right]$}
& \scriptsize{$\left[0.2,0.4\right]$}
& \scriptsize{$\left[0.4,0.6\right]$}
& \scriptsize{$\left[0.6,0.8\right]$}
& \scriptsize{$\left[0.8,1.0\right]$}
& \scriptsize{$\left[1.0,1.2\right]$}
& \scriptsize{$\left[1.2,1.4\right]$}
& \scriptsize{$\left[1.6,1.8\right]$}
& \scriptsize{$\left[1.8,2.0\right]$}
& \scriptsize{$\left[2.0,2.2\right]$}
& \scriptsize{$\left[2.2,2.4\right]$} \\ \midrule
\scriptsize{$\left[0.0,0.2\right]$} & 23.7 & 2.6 & 2.2 & 2.5 & 2.7 & 2.9 & 2.9 & 2.9 & 2.8 & 3.1 & 4.2 \\
\scriptsize{$\left[0.2,0.4\right]$} & 2.6 & 26.2 & 2.6 & 2.9 & 3.1 & 3.3 & 3.4 & 3.2 & 3.0 & 3.9 & 4.7 \\
\scriptsize{$\left[0.4,0.6\right]$} & 2.2 & 2.6 & 26.6 & 2.6 & 2.8 & 2.9 & 3.2 & 2.9 & 2.5 & 3.3 & 4.1 \\
\scriptsize{$\left[0.6,0.8\right]$} & 2.5 & 2.9 & 2.6 & 25.6 & 3.3 & 3.4 & 3.7 & 3.3 & 2.8 & 3.7 & 4.7 \\
\scriptsize{$\left[0.8,1.0\right]$} & 2.7 & 3.1 & 2.8 & 3.3 & 23.3 & 3.9 & 4.2 & 3.7 & 3.4 & 4.6 & 5.6 \\
\scriptsize{$\left[1.0,1.2\right]$} & 2.9 & 3.3 & 2.9 & 3.4 & 3.9 & 30.8 & 4.5 & 4.1 & 4.0 & 5.7 & 6.8 \\
\scriptsize{$\left[1.2,1.4\right]$} & 2.9 & 3.4 & 3.2 & 3.7 & 4.2 & 4.5 & 36.5 & 4.3 & 3.7 & 5.8 & 6.7 \\
\scriptsize{$\left[1.6,1.8\right]$} & 2.9 & 3.2 & 2.9 & 3.3 & 3.7 & 4.1 & 4.3 & 94.9 & 3.8 & 5.1 & 6.2 \\
\scriptsize{$\left[1.8,2.0\right]$} & 2.8 & 3.0 & 2.5 & 2.8 & 3.4 & 4.0 & 3.7 & 3.8 & 82.4 & 6.2 & 7.0 \\
\scriptsize{$\left[2.0,2.2\right]$} & 3.1 & 3.9 & 3.3 & 3.7 & 4.6 & 5.7 & 5.8 & 5.1 & 6.2 & 110.7 & 10.3 \\
\scriptsize{$\left[2.2,2.4\right]$} & 4.2 & 4.7 & 4.1 & 4.7 & 5.6 & 6.8 & 6.7 & 6.2 & 7.0 & 10.3 & 171.0 \\
\bottomrule
    \end{tabular}
    \end{sideways}
    \end{center}
  \caption[Correlation matrix for all the systematic errors.]
{\label{tab:corr_tot}Correlation matrix for all the systematic errors.  All the
values are given in units $\times 10^{-6}$ \cite{bendavid2011electron}.}
\end{table}

\section{Results}
The measured electron charge asymmetry results are summarised in
\TableRef{tab:updatedresults} with both the statistical and systematic
uncertainties included. The experimental results are compared with predictions
from various PDF models obtained with the MCFM\cite{campbellmcfm} MC tool.  The
MCFM MC tool has been interfaced with the CT10\cite{lai2010vv},
HERAPDF\cite{aaron2010combined}, NNPDF\cite{Lionetti:2011pw} and
MSTW2008\cite{martin2009parton} PDF models and  the uncertainties are obtained
by using the prescription outlined in \SectionRef{sec:asymuncert}.

The results are also presented in \FigureRef{fig:updated_asym}. The experimental
data are in agreement with the predictions from the CT10, NNPDF and HERAPDF pdf
models whereas MSTW is systematically lower that than the experimentally
observed asymmetry in the more central regions. This is also demonstrated in
\FigureRef{fig:Carino} which shows the ratio between the experimental results
and the theoretical predictions from the four PDF models.
\todo{plot comparing 1st and 2nd measurement...}

\begin{figure}[htbp]
  \begin{center}
\includegraphics*[width=0.80\textwidth]{updated_asym}
  \caption[Measured electron charge asymmetry with predictions from CTEQ10,
HERAPDF15, MSTW08NNLO and NNPDF2.2.]{\label{fig:updated_asym} Measured electron
charge asymmetry with predictions from CTEQ10, HERAPDF15, MSTW08NNLO and
NNPDF2.2.}
  \end{center}
\end{figure}

\begin{table}[htbp]
\begin{center}
\begin{tabular}{lccccc}
\toprule
$|\eta|$ range  & Data & CTEQ & HERAPDF & MSTW & NNPDF \\ \midrule
  $0.0<|\eta|<0.2$ &104$\pm$3$\pm$5 &$109^{+5}_{-5}$ &$106^{+4}_{-8}$ & $83^{+3}_{-5}$& 107$\pm$5\\
  $0.2<|\eta|<0.4$ &114$\pm$3$\pm$5 &$114^{+5}_{-5}$ &$110^{+4}_{-8}$ & $85^{+3}_{-5}$& 110$\pm$5\\
  $0.4<|\eta|<0.6$ &115$\pm$3$\pm$5 &$119^{+5}_{-5}$ &$115^{+4}_{-8}$ & $92^{+3}_{-5}$& 116$\pm$5\\
  $0.6<|\eta|<0.8$ &125$\pm$3$\pm$5 &$126^{+5}_{-5}$ &$122^{+4}_{-8}$ & $98^{+3}_{-5}$& 123$\pm$5\\
  $0.8<|\eta|<1.0$ &130$\pm$3$\pm$5 &$138^{+5}_{-6}$ &$132^{+4}_{-8}$ & $108^{+4}_{-5}$& 134$\pm$5\\
  $1.0<|\eta|<1.2$ &138$\pm$3$\pm$6 &$146^{+6}_{-6}$ &$140^{+5}_{-8}$ & $120^{+4}_{-5}$&145$\pm$5 \\
  $1.2<|\eta|<1.4$ &154$\pm$3$\pm$6 &$164^{+6}_{-7}$ &$153^{+5}_{-7}$ & $136^{+5}_{-5}$&158$\pm$5 \\
  $1.6<|\eta|<1.8$ &169$\pm$3$\pm$10 &$195^{+8}_{-9}$ &$181^{+5}_{-5}$ & $168^{+5}_{-5}$&190$\pm$4 \\
  $1.8<|\eta|<2.0$ &200$\pm$3$\pm$9 &$207^{+8}_{-10}$ &$196^{+4}_{-3}$ & $184^{+6}_{-5}$&206$\pm$4 \\
  $2.0<|\eta|<2.2$ &224$\pm$3$\pm$11 &$224^{+8}_{-11}$ &$211^{+5}_{-3}$ & $198^{+6}_{-5}$&219$\pm$4 \\
  $2.2<|\eta|<2.4$ &216$\pm$4$\pm$13 &$241^{+8}_{-12}$ &$225^{+9}_{-4}$ & $214^{+6}_{-5}$&231$\pm$5 \\
\bottomrule
\end{tabular}
\caption[Measured electron charge asymmetry with predictions from CTEQ6.6,
HERAPDF, NNPDF and MSTW PDFs.]{Measured electron charge asymmetry with
predictions from CTEQ6.6, HERAPDF, NNPDF and MSTW PDFs. Uncertainties on
measured asymmetry are statistical and systematic respectively and the
uncertainties on predictions are due to the uncertainties on the PDFs. All units
are $\times 10^{-3}$\cite{bendavid2011electron}.}
\label{tab:updatedresults}
\end{center}
\end{table}

\begin{figure}[htbp]
  \begin{center}
\includegraphics*[width=0.80\textwidth]{plotCarino}
  \caption[Ratio between the measured charge asymmetry and the predictions from
CTEQ(top-left), HERAPDF(top-right), MSTW(bottom-left), NNPDF(bottom-right). ]
{\label{fig:Carino} Ratio between the measured charge asymmetry and the
predictions from CTEQ(top-left), HERAPDF(top-right), MSTW(bottom-left),
NNPDF(bottom-right). The error bars include both the uncertainties on the
experimental results and the uncertainties on theoretical
predictions\cite{bendavid2011electron}.}
  \end{center}
\end{figure}

