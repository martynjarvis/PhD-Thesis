\documentclass[varwidth=true, border=10pt, convert={size=640x}]{standalone}
%\usepackage{blindtext}
\usepackage[latin1]{inputenc}
\usepackage{tikz}
\usetikzlibrary{shapes,arrows}
\usetikzlibrary{decorations.pathmorphing}
\usetikzlibrary{decorations.markings}
\usepackage{verbatim}

\begin{document}

\tikzset{
}

% Define block styles
\tikzstyle{simple} = [draw=none, fill=none]
\tikzstyle{process} = [circle]
\tikzstyle{line} = [draw]
\tikzstyle{boson} = [style={decorate, decoration={snake}, draw=red}]
\tikzstyle{fermion} = [style={draw=blue, postaction={decorate},decoration={markings,mark=at position .55 with {\arrow[draw=blue]{>}}}}]
\tikzstyle{gluon} = [style={decorate, draw=magenta,decoration={coil,amplitude=4pt, segment length=5pt}} ]

%\blindtext

% Define styles for the different kind of edges in a Feynman diagram

\begin{tikzpicture}[
        thick,
        node distance = 3cm,
        auto
    ]
    % first set up nodes

    \node[simple, circle] (init){};

    \node[process,above left of=init](p1process){};
    \node[simple,left of=p1process](p1start){};

    \node[process,below left of=init](p2process){};
    \node[simple,left of=p2process](p2start){};

    \node[simple,right of=init](wdecay){};
    \node[simple,above right of=wdecay](l1end){};
    \node[simple,below right of=wdecay](l2end){};

    % then paths, in the correct direction
    \path [fermion] (p1start.base)--(p1process.base);
    \path [fermion] (p1process.base)--(init.base);

    \path [fermion] (p2start.base)--(p2process.base);
    \path [fermion] (p2process.base)--(init.base);

    \path [boson] (init.base) -- (wdecay.base);
    \path [fermion] (wdecay.base) -- (l1end.base);
    \path [fermion] (wdecay.base) -- (l2end.base);


\end{tikzpicture}


\end{document}
