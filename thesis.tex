\documentclass{mythesis}
\usepackage{mythesis}

%% You can set the line spacing this way
%\setallspacing{double}
%% or a section at a time like this
%\setfrontmatterspacing{double}

%% PDF metadata
\makeatletter
\@ifpackageloaded{hyperref}{%
\hypersetup{%
pdftitle = {title},
pdfsubject = {subject},
pdfkeywords = {physics, LHC},
pdfauthor = {Martyn Jarvis}
}
}{}
\makeatother

\graphicspath{{figures/}{gen_figures/}}


%% Define the thesis title and author
\title{Some Thesis}
\author{Some Author}

%% Start the document
\begin{document}

%% Define the un-numbered front matter (cover pages, rubrik and table of contents)
\begin{frontmatter}
  %% Title
\titlepage[subtitle]%
{A thing submitted to somewhere}

%% Abstract
\begin{abstract}%[\smaller \thetitle\\ \vspace*{1cm} \smaller {\theauthor}]
  %\thispagestyle{empty}
  test test test
\end{abstract}


%% Declaration
\begin{declaration}
  This dissertation is awesome
  \vspace*{1cm}
  \begin{flushright}
    My Name
  \end{flushright}
\end{declaration}


%% Acknowledgements
\begin{acknowledgements}
  Of the many people who deserve thanks, some are particularly prominent,
  such as my supervisor\dots
\end{acknowledgements}


%% Preface
\begin{preface}
  This is a thesis...

  \noindent
  More preface here
\end{preface}

\dedication{To me...}

%% ToC
\tableofcontents

%% Strictly optional!
\frontquote%
  {quote}%
  {quoter}



\end{frontmatter}

%% Start the content body of the thesis
\begin{mainmatter}
  %% Actually, more semantic chapter filenames are better, like "chap-bgtheory.tex"
  \chapter{Introduction}
\label{intro}

Start writing here.
  %% To ignore a specific chapter while working on another,
  %% making the build faster, comment it out like this:
  %\input{chap4}
\end{mainmatter}

%% Produce the appendices
\begin{appendices}
  %\input{chapters/End.tex}
\end{appendices}

%% Produce the un-numbered back matter (e.g. colophon,
%% bibliography, tables of figures etc., index...)
% \begin{backmatter}
% \input{backmatter}
% \bibliographystyle{plain}
% \bibliography{thesis}
% \end{backmatter}

\end{document}
