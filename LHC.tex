\chapter{The LHC}
\label{chap:LHC}
\section{Large Hadron Collider}
The Large Hadron Collider (LHC) is a circular synchrotron with a circumference
of \unit{27}{\kilo\meter}.
It has been constructed in the existing tunnel 
\unit{40-170}{\meter} beneath the border of France and Switzerland
that was previously home to the LEP collider.

When operating at its design energy and luminosity it will collide beams or
protons at a centre of mass energy of \unit{14}{\TeV} and a luminosity of
\unit{$ 10^{34} $}{\rpsquare\cm\reciprocal\second} .
It is also designed to collide two \unit{5.5}{\TeV} beams of heavy ions, such
as lead nuclei.\cite{lhc}

The main motivation for the LHC is to determine the mechanism that is
responsible for electroweak symmetry breaking, of which the most favoured is the
Higgs mechanism.  The LHC is also designed to test the Standard Model at the
$\TeV$ scale at high precision and to search for new particles predicted by
theories beyond the standard model such as supersymmetric theories and theories
involving extra dimensions.  \FigureRef{fig:LHCxsec} shows various cross
sections for several physics processes as a function of the centre of mass
energy. The cross section for many physics processes of interest, such as the
Higgs cross section, are several orders of magnitude below the total inelastic
cross section and increase as a function of centre of mass energy.  The large
centre of mass energy and high luminosity of the LHC is needed to be able to
probe the small cross-sections of new physics processes of interest.

\begin{figure}[htbp]
  \centering
  \includegraphics[width=0.85\textwidth]{xsec.png}
  \caption{The production cross sections as a function of centre of mass energy
for several Standard Model and new physics processes.}
  \label{fig:LHCxsec}
\end{figure}

The LHC is part of a larger accelerator complex as shown in figure
\FigureRef{fig:LHCcomplex}. Hydrogen gas is first ionised to produce a cloud of
protons, which are then accelerated by the LINAC2 linear accelerator to
\unit{50}{\MeV}.  Before being injected in to the Proton Synchrotron (PS) the
protons are injected in to the Proton Synchrotron Booster (PSB) and accelerated
to \unit{1.4}{\GeV}. In the PS the protons are formed in to bunches and the
energy is increased to \unit{25}{\GeV}. The bunches are then accelerated in the
Super Proton Synchrotron (SPS) to \unit{450}{\GeV} and then injected in to the
LHC.

%bunches?
\begin{figure}[htbp]
  \centering
  \includegraphics[width=0.7\textwidth]{lhcdipole.png}
  \caption{Cross section of a LHC dipole and cryostat. Parts shown are the; Beam
screen (1), Cold bore (2), Cold mass at 1.9 K (3),
Radiative insulation (4), Thermal shield (55 to 75 K) (5),
Support post (6), Vacuum vessel (7), Alignment target (8). From \cite{lyn}.}
  \label{fig:lhcdipole}
\end{figure}

\begin{figure}[htbp]
  \centering
  \includegraphics[width=0.96\textwidth]{accelerators.png}
  \caption{The LHC complex.}
  \label{fig:LHCcomplex}
\end{figure}

There are four main detector experiments studying the collisions at the LHC. 
ALICE (A Large Ion Collider Experiment) is designed to study the quark gluon
plasma that will be produced in the heavy ion collisions. 
LHCb (The Large Hadron Collider beauty) experiment is designed to study B-meson
decays to measure CP violation. 
ATLAS (A Toroidal LHC Apparatus) and CMS (Compact Muon Solenoid) are general
purpose detectors that are designed 
to search for a wide range of new physics.\cite{lhc}

\subsection{Operational History}
In September 2008 the LHC was commissioned and the first beams were circulated.
Before the first collisions could be delivered an interconnection
between two of the dipole magnets failed when the magnet quenched.
This led to a large amount to helium rapidly evaporating which caused
considerable damage to the machine.
Due to this incident it was decided that the LHC should be run at a lower centre
of mass energy of \unit{7}{\TeV} until the quench protection system could be
upgraded.

The LHC was repaired by the end of 2009 and the first collisions at a record
energy of \unit{2.36}{\TeV} were delivered in November. 
From March to November 2010 the LHC operated at \unit{7}{\TeV} delivering
\unit{46.4}{\invpb} of proton-proton collisions.
In November and October 2010 the LHC produced lead ion collisions at
\unit{2.36}{\TeV}.

\begin{figure}[htbp]
  \centering
  \includegraphics[width=\textwidth]{int_lumi_cumulative_pp_1.png}
  \caption{The luminosity delivered by LHC and recorded by CMS in 2010, 2011 and 2012}
  \label{fig:LHC2010}
\end{figure}


The target for running in 2011 was to deliver \unit{1}{\invfb} of data. This was
achieved by June. The target was increased to \unit{5}{\invfb} of data by the end
of the yeah which was achieved by October.
The luminosity delivered by LHC and recorded in CMS in 2010, 2011 and 2012 is
shown in \FigureRef{fig:LHC2010}.

\section{CMS Detector}
\ac{CMS} is one of the two general purpose
detectors designed to study LHC collisions. The main design parameters for the
\ac{CMS} detectors  are listed in \TableRef{tab:cmsparam}

\begin{table}[htbp]
\begin{center}
\begin{tabular}{ l l }
\toprule
Parameter & CMS \\
\midrule
Total weight (tons)                 & $12,500$  \\
Overall diameter (m)                & $15$  \\
Overall length (m)                  & $20$  \\
Magnetic field for tracking (T)     & $4$  \\
Solid angle for energy measurements ($\Delta\phi \times \Delta\eta$)   
                                    & $2\pi \times 9.6$  \\
Solid angle for precision measurements ($\Delta\phi \times \Delta\eta$)   
                                    & $2\pi \times 5.0$  \\
Total cost (CHF)                    & $550\times 10 ^{6}$  \\
\bottomrule
\end{tabular}
\caption{Main design parameters of the CMS detectors. From \cite{parris}}
\end{center}
\label{tab:cmsparam}
\end{table}

The design goals of the CMS detector are:\cite{tdr}
\begin{itemize}
  \item Good muon identification and momentum resolution and the ability to
unambiguously assign charge to muons with $\PT < \unit{1}{\TeV}$
  \item Good charged particle momentum resolution and reconstruction in the
tracker.
  \item Good electromagnetic energy resolution. 
  \item Good \ETmiss and dijet mass resolution.
\end{itemize}

The design of CMS meets these requirements while overcoming significant
experimental challenges.  At design luminosity, $\approx 1 $ billion inelastic
events will occur in CMS every second, where as CMS is limited to storing the
data of only $\approx 100 $ events can be stored.  The detector must be able to
reduce this rate with a trigger to accept events that are interesting from a
physics perspective and reject events otherwise.

In addition to this challenge each event of interest will have on average 20
inelastic events superimposed on it. This results in around 1000 charged
particles produced every \unit{25}{\ns}, which will require the detectors to
have a high granularity with a good time resolution to ensure a low occupancy.
The large flux of particles will also produce high radiation levels which 
required radiation hard detectors and electronics.

\begin{figure}[htbp]
  \centering
  \includegraphics[width=0.98\textwidth]{cms_su.png}
  \caption{Diagram of the CMS detector, from\cite{sketchup}.}
  \label{fig:CMSnc}
\end{figure}

An overview of the detector is shown in \FigureRef{fig:CMSnc}.  Starting at the
interaction point in the centre of the detector and moving radially outwards,
CMS comprises the pixel tracker, silicon microstip tracker, lead-tungstate ECAL,
sampling brass-plate HCAL, \unit{4}{\tesla} superconducting solenoid magnet, an
outer HCAL and four muon chambers.

\subsection{Magnet}
A large superconducting solenoid provides the basis for the design of the CMS
detector, and is the main structural support for the detector components in the
barrel region.

The superconducting magnet in CMS produces a \unit{4}{\tesla} field in a bore of
\unit{6}{\meter} diameter and \unit{12.5}{\meter} length.  While operating at
full current the magnet stores \unit{2.6}{\giga\joule} of energy.  A large
magnetic field is needed to give CMS a large bending power and the ability to
precisely measure the momentum of high-energy charged particles.  The solenoid
bore is large enough that the tracking detectors and the calorimetry can fit
inside it.\cite{cms}

The magnetic flux is returned through a \unit{1.8}{\meter} thick saturated iron
yoke which is interleaved with the muon detector.

\subsection{Tracking}
The inner tracker is designed to accurately and efficiently measure the
trajectories of charged particles produced in collisions at the centre of CMS.
The tracker is also required to be able to reconstruct secondary vertices from
the decay of a long lived particle.  At the design luminosity of the LHC it
is expected that every \unit{25}{\ns} an average of 1000 particles will traverse
the inner detector therefore it is required that the tracker has a high
granularity and a fast response while remaining resilient to radiation damage. 

%\begin{figure}[htbp]
  %\centering
  %\includegraphics[width=0.8\textwidth]{strip.png}
  %\caption{Close up of the CMS inner tracker. The pixel tracker is shown in lime
%and the microstrip detector is show in blue.\cite{sketchup} }
  %\label{fig:tracker}
%\end{figure}

A close up view of the inner tracker is shown in \FigureRef{fig:tracker}.  The
tracker utilises silicon pixel detectors in the inner most layers where the
particle flux is the highest.  Outside of the pixel detector, the tracking
detector comprises several layers of silicon microstrip detectors where the
particle flux is smaller.  The total active area of silicon in the CMS tracker
is over \unit{200}{\meter\squared}.\cite{cms}

\subsubsection{Pixel Tracker}
The pixel tracker consists of three layers of hybrid silicon pixel detectors in
the barrel region and two in the endcap region. 
The barrel layers are positioned at radii of 4.4, 7.3 and \unit{10.2}{\cm} and have
a length of \unit{53}{\cm}. The two layers in the endcap are located at
$|z|=34.5$ and \unit{46.5}{\cm} with an inner radius of \unit{6}{\cm} and an
outer radius of \unit{15}{\cm}.

Each pixel has a surface area of \unit{$100\times150$}{\micron} which results in
an average particle ocupancy of $\mathcal{O}(10^{-4})$ per pixel per crossing.

\subsubsection{Strip Tracker}
The strip tracker is divided into two regions with differing particle flux. In
the region \unit{$20<r<55$}{cm} microstrip detectors with a cell size of
$\unit{10}{\cm}\times\unit{80}{\micron}$ are used with an average occupancy of
$\approx\unit{2-3}{\%}$.
Outside this the flux is low enough to allow for the use of larger pitch silicon
microstrips with a cell size of $\unit{25}{\cm}\times\unit{180}{\micron}$ with
an average occupancy of $\approx\unit{1}{\%}$.
The silicon entire silicon strip detector consists of approximately 15400
modules. \todo{check this}
The barrel strip tracker comprises two parts, the inner (TIB) and outer (TOB)
trackers. The TIB and TOB are formed of 4 and 6 layers respectively.

The endcaps are separated in to the Tracker End Cap (TEC) and the Tracker Inner
Disks (TID). The TEC is split in to nine disks and covers the region
$\unit{1200}{\mm} < |Z| < \unit{2800}{\mm}$. The TID comprises three rings
and fills the gap between the TEC and the TIB.

\subsubsection{Performance}



\todo[inline]{section on the performance of the tracking detectors}

\subsection{Electromagnetic Calorimeter}
The electromagnetic calorimeter (ECAL) is 
designed to measure the energy of
electrons and photons with a high resolution.
It has a fine lateral granularity to help with shower separation. 
A close up view of the CMS calorimetry is shown in \FigureRef{fig:calo}, where the
ECAL is shown in green.
It is a hermetic, homogeneous calorimeter comprising 61200 individual lead
tungstate ($PbWO_{4}$) scintillation crystals in the barrel region
($|\eta|<1.479$) closed by 7324 crystals in each of the two endcap parts
($1.479<|\eta|<3.0$).

%\begin{figure}[htbp]
  %\centering
  %\includegraphics[width=0.8\textwidth]{hcal.png}
  %\caption{Close up of the CMS calorimetry showing the ECAL (green) and the HCAL
%(yellow) within the solenoid magnet (white).}
  %\label{fig:calo}
%\end{figure}

Lead tungstate crystals are ideally suited for this since the scintillation
decay time is similar to the LHC bunch crossing time, 
with \unit{80}{\%} of light being produced within \unit{25}{\ns}.
They also have a short radiation lengths ($X_0=\unit{0.89}{\cm}$) 
and Moliere length (\unit{2.2}{\cm}) as well as being radiation hard
(up to \unit{10}{\mrad}).

A disadvantage to using lead tungstate is that the light output of the crystals
is relatively low and changes with temperature. This is overcome by maintaining
a stable temperature (within \unit{0.1}{\degreecelsius}) and using
photodetectors with an intrinsic gain.
Silicon avalanche photodiodes (APDs) are used to detect the scintillation light
in the barrel and vacuum phototriodes (VPTs) are used in the endcap parts.

The barrel section of the ECAL (EB) surrounds the inner tracker. It comprises
36 identical supermodules that each cover a half length of the barrel
($0<|\eta|<1.479$) and \unit{20}{\degree} in phi. Each supermodule contains
1700 crystals arranged in a $\phi - \eta$ grid with each crystal mounted in a
``semi-projective'' geometry, and alligned \unit{3}{\degree} off the
nominal interaction vertex.
The alignments in $\phi$ and $\eta$ are shown in \FigureRef{fig:crystaltilt} and
\FigureRef{fig:crystaltilt} respectivly. The non-poiting geometry 
 prevents particles escaping throught the gaps between the crystals.
Each crystal has a cross-section of
\unit{$22 \times 22$}{\mm\squared} and a length of
\unit{230}{\mm} (\unit{25.8}{X_0}).

\begin{figure}[htbp]
  \centering
  \includegraphics[width=0.8\textwidth]{crystallong}
  \caption{The crystal alignment in the longditinal view. From \cite{ecaltdr}. }
  \label{fig:crystallong}
\end{figure}

\begin{figure}[htbp]
  \centering
  \includegraphics[width=0.8\textwidth]{crystaltilt}
  \caption{The tilt of the ECAL crystals in the transverse plane. From \cite{ecaltdr}. }
  \label{fig:crystaltilt}
\end{figure}

The endcaps (EE) are formed of two ``Dees'', semi-circular aluminium plates
which support the ``supercrystals'', 5x5 arrays of crystals. The crystals are
mounted off the nominal interaction vertex by a small angle in a similar way
to the barrel. 
Unlike the barrel the crystals are arranged in an x-y grid.
Installed in front of the endcap ECAL is a preshower system which helps with
the rejection of \Ppizero \cite{cms}.

\subsubsection{Performance}

\begin{figure}[htbp]
  \centering
  \includegraphics[width=0.7\textwidth]{ecal_performance}
  \caption{Energy resolution $\frac{\sigma}{E}$ of ECAL as a function of
  \label{fig:ECAL}
electron energy $E$. From \cite{cms}.}
\end{figure}

Using a \unit{100}{\GeV} test beam, the energy resolution of the \ac{ECAL} was
found to be\cite{tdr},
\begin{align}
\left(\frac{\sigma}{E}\right)^{2} 
= \left(\frac{S}{\sqrt{E}}\right)^{2} + \left(\frac{N}{E}\right)^{2} + C^{2}\\
=
\left(\frac{\unit{2}{\%}}{\sqrt{E}}\right)^{2} +
\left(\frac{\unit{124}{\MeV}}{E}\right)^{2} + 
\left(\unit{0.26}{\%}\right)^{2}  
\end{align}
where $S$ is the stochastic, $N$ is the noise and $C$ is the constant terms. The
stochastic term is due to the statistical fluctuations in the particles produced
in the electromagnetic shower. The noise term is due to electronic noise and
pile-up. The constant term is due to errors that are independent of energy, such
as non-uniform signal generation and calibration errors \cite{cms}.

\subsection{Hadronic Calorimeter}

The Hadronic Calorimeter (HCAL), in addition to the electromagnetic calorimeter,
is designed to measure the energy of hadron jets and the missing transverse
energy (\met) which are important signatures in many physics studies at the LHC.
A close up view of the CMS calorimetry is shown in \FigureRef{calo}, where the
HCAL is shown in yellow.

The  HCAL is a brass/scintillator
sampling hadron calorimeter that covers the region up to $|\eta|<3.0$.  The
scintillation light is channelled by wavelength shifting fibres, that are
embedded in the scintillation tiles, to hybrid photodiodes that can operate in
the high axial magnetic field \cite{cms}.

The barrel hadron calorimeter (HB) covers the region ($|\eta| < 1.4$)
and the hadron endcap covers the region $1.3 < |\eta| < 3.0$.

THe barrel hadron calorimter is
positioned between the ECAL and inside of the solenoid magnet coil
($\unit{1.77}{\meter}<R<\unit{2.95}{\meter}$).
The strong constraints imposed by the dimensions of the solenoid magnet results
in the HB having insufficient amount of material to absorb the hadronic shower. 
To overcome this limitation the outer hadronic calorimeter (HO) or tail catcher,
has been added around the solenoid magnet to increase the
effective thickness of the hadron calorimetry to over 10 interaction lengths.
This provides better protection against punch through to the muon system.

\subsubsection{Performance}

\FigureRef{fig:hcalperform} shows the jet energy resolution for three parts of
the \ac{HCAL}. The granularity of the sampling in each region of the HCAL is
such that the resolution is similar in each.
\begin{figure}[htbp]
  \centering
  \includegraphics[width=0.7\textwidth]{hcal_performance}
  \label{fig:hcalperform}
  \caption{Energy resolution $\frac{\sigma}{E}$ of HCAL as a function of jet
transverse energy for barrel jets ($|\eta| < 1.4$), endcap jets ($1.4<|\eta| <
3$) and forward jets ($3<|\eta| < 5$) }
\end{figure}

\begin{figure}[htbp]
  \centering
  \begin{subfigure}{0.48\textwidth}
    \centering
    \includegraphics[width=\textwidth]{met_res}
    \caption{\ETm resolution.}
    \label{fig:met_res}
  \end{subfigure}
  \begin{subfigure}{0.48\textwidth}
    \centering
    \includegraphics[width=\textwidth]{met_mean}
    \caption{Average reconstructed \ETm.}
    \label{fig:met_mean}
  \end{subfigure}
  \caption{ The missing transverse energy performance as a function of the
            $\sqrt{E_T}$ for QCD events with pile-up.  }
\label{fig:met_performance} 
\end{figure}

The performance of the missing transverse energy (\ETm) is shown in
\FigureRef{fig:met_performance} in QCD events. The \ETm resolution is
$\sigma(\ETm) \approx 1.0 \sqrt \ET$ and the average \ETm is 
$\langle \ETm \rangle \approx 1.25 \sqrt \ET$ \cite{tdr}.

\subsection{The Forward Calorimter}
In the forward region ($|\eta| > 3$) energy measurements are made with the
forward hadronic calorimeter, situated \unit{11}{\meter} from the interaction
point. The main role of the forward calorimeter is to improve the \ETm
measurement and to tag jets in the forward direction.

The forward hadronic calorimeter is an iron/quartz-fibre calorimeter where the
Cherenkov light is detected by photodetectors.  The calorimeter needs to be
radiation hard due to the very large flux in the forward region. However it is
expected that after 10 years of operation the lightn output will be reduced by
about \unit{30}{\%} due to the level of radiation.

\subsection{Muon System}
The muon system lies outside of the CMS solenoid and the outer HCAL detectors.
It is designed to have three functions; to identify muons, measure the momentum
of muons and trigger on muons. To perform these functions the muon system
consists of several different types of detectors.

\begin{figure}[htbp]
  \centering
  \includegraphics[width=0.98\textwidth]{muon_system}
  \caption{Muon system}
  \label{fig:muon_system}
\end{figure}

\subsubsection{Drift Tubes}
In the barrel region ($|\eta| < 1.2$) aluminium drift tubes (DT) are used,
arranged in four stations interleaved in the flux return
plates. 
Each station contains 12 layers, eight to measure the coordinate in the
$r-\phi$ plane and four to measure the $z$ direction (except the fourth station
which only measures the $r-\phi$ plane). 

\subsubsection{Cathode Strip Chambers}
In the endcaps ($0.9<|\eta|<2.4$) cathode strip chambers (CSC) are used. The
CSCs are arranged in four stations in each endcap. Their faces are perpendicular
to the beam line, and are placed between the flux return plates.  The cathode
strips of each chamber run radially away from the beam line where as the anode
wires run perpendicular to the strips; both are read out which gives information
on both the $r-\phi$ plane (from the cathode) and the $\eta$ direction (from the
anode). \cite{cms}

\subsubsection{Resistive Plate Chambers}
In addition to DT chambers and CSCs a complementary trigger system is also used
consisting of resistive plate chambers (RPC) in the endcap and barrel regions.
The RPCs are able to provide a fast and independent trigger over a large range
($|\eta| < 1.6$). In the barrel region, 6 layers of RPCs are used, 2 in each of
the first 2 muon stations and 1 in each of the last 2 stations. In the endcap a
layer of RPCs in each of the first 3 stations.

An optical alignment system, that uses lasers and LEDs, measures the positions
of each muon station with respect to each other and the CMS inner tracker to
ensure an accurate and high resolution measurement of the muon
momentum.\cite{cms}

\subsubsection{Performance}
The performance of the muon system and inner tracker is shown in figure
\ref{fig:muon_performance}.

\begin{figure}[htbp]
  \centering
  \begin{subfigure}{0.48\textwidth}
    \centering
    \includegraphics[width=\textwidth]{muon_barrel}
    \caption{Barrel.}
    \label{fig:met_res}
  \end{subfigure}
  \begin{subfigure}{0.48\textwidth}
    \centering
    \includegraphics[width=\textwidth]{muon_endcap}
    \caption{Endcap.}
    \label{fig:met_mean}
  \end{subfigure}
  \caption{Muon transverse momentum resolution as a function of the
  transverse-momentum using only the muon system, only the tracker and both, for
  barrel muons ($|\eta| < 0.8$) and endcap muons ($1.2<|\eta| < 2.4$). From
  \cite{cms}.} 
  \label{fig:muon_perfomance} 
\end{figure}

\subsection{Trigger and Data Acquisition}
At design luminosity, the high bunch crossing frequency of the LHC means that
each crossing will contain an average of 20 superimposed inelastic
events every \unit{25}{\nano\second}, a rate of $10^{9}$ interactions per
second.
The size of an event is approximatly \unit{1}{MB} after zero-suppresion.
The total data output rate from CMS is \unit{$\approx 80$}{\tera \bel \per
\second}.
These figures are many orders of magnitude larger that the storage and offline
processing capability available to CMS, which corresponds to about
\unit{250}{MB\per\second} or about $\approx 10^{2}$ crossings written to the tape
storage every second. 

The vast majority of events will contain only glancing inelastic collisions and
not be of interest from a physics perspective and can be discarded.  It is the
job of the trigger to reduce the rate of events by a factor of $10^6$ by
rejecting the uninteresting events while keeping as many interesting events as
possible.

An overview of the \ac{DAQ} and trigger is shown in figure \ref{fig:CMSDAQ}.
The trigger is separated into two parts, the Level-1 (L1) trigger and the High
Level Trigger (HLT).\cite{cms}

\begin{figure}[htbp]
  \centering
  \includegraphics[width=0.98\textwidth]{CMSDAQ}
  \caption{Overview of the architecture of the CMS DAQ and trigger. From
  \label{fig:CMSDAQ}
\cite{cms}.}
\end{figure}


\subsubsection{Level-1 Trigger}

The Level-1 trigger (L1) is designed to reduce the event rate from the bunch
crossing frequency of \unit{40}{\mega\hertz} to a maximum output rate of
\unit{100}{\kilo\hertz}.
The L1 trigger is implemented with custom designed fast programmable electronics
 that takes as input coarse data from the calorimeters and muon systems and
places the high resolution data in pipelined memories. The reduced resolution
and granularity data are used to form ``trigger primitives'', such as isolated
high energy electromagnetic deposits that pass a certain \PT or \ET threshold,
which the L1 Trigger uses to base its decision. It also recieves information on
event wide variables such as the total sum of transverse energy and the missing
transverse energy.

\subsubsection{High-Level Triggers}
After a fixed time interval of \unit{3.2}{\micro\second} the high resolution
event data held in the pipeline memories is either readout or discarded
depending the decision at the Level-1 trigger.  The data are transfered to a
processor running the high-level trigger software.

The high-level trigger (HLT) is a software system which runs on a server farm with
over one thousand commercial multicore processors with access to the complete
event data allowing it to make more complex calculations. 
Objects are reconstructed in the HLT as they are needed and events are discarded
as soon as possible to avoid wasted processing time.
The decision to accept an event is based on the the requirements on the datasets
used in \ac{CMS} analyses. A typical data-set requires a high \pT
HLT-reconstructed object or a ammount of missing transverse energy in the event,
for example, the data-set relevent to the analyses that follow in later chapters
requires a single high \pT electron.

\subsubsection{Computing}

The \ac{CMS} data is available in several different file formats that contain
different levels of details,
\begin{itemize}
\item Raw data, is the output of the events that pass the \ac{HLT}. The files
contain the detector data, the \ac{L1} trigger results, the \ac{HLT} trigger
results and the higher-level objects that are created durint the \ac{HLT}
process.
\item \ac{RECO} data, is the output of the reconstruction process. The files contain
the reconstructed physics objects and reconstructed hits and clusters.
\item \ac{AOD}, is the reduced event representation and contains only the
reconstructed objects. This is the format that is used to preform physics
analyses.
\end{itemize}

CMS utilizes a distributed computing model called the ``Grid''.
The Grid consists of several clusters of computer distributed around the world
and organised in to several tiers.
Events that pass the \ac{HLT} are sent to the primary Tier-0 centre where the
are stored on tape and undergo prompt reconstruction. The reconstructed data is
distributed to Tier-1 and Tier-2 centres at national laboratories and
universites around the world. The role of the Tier-1 and Tier-2 sites are to run
the physics analyses and to reprocess the data when updated calibrations are
available. This is typeically done once or twice a year.

