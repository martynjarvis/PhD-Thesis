\chapter{Conclusion}
\label{chap:conclusion}

The electron charge asymmetry in inclusive \PW production has been measured
using an initial data sample corresponding to an integrated luminosity of
\unit{36}{\invpb} of data and updated with a larger data sample corresponding to
an integrated luminosity of \unit{840}{\invpb}.

The measurement has been performed with a template fitting method to the \ETm
spectrum of events selected using a simple cut based electron selection.
The backgrounds have been modelled using control regions obtained with an
inverted selection for the QCD contribution, as well as Monte Carlo simulations
for the better understood background contributions.
The signal has been modelled using a MC simulation corrected with a data-driven
boson recoil method.
The measurement was presented in bins of pseudorapidity. In the first
measurement 5 bins were used and in the updated measurement the additional
statistics available allowed the results to be presented in 11 bins.
The results have been compared directly to predictions based on several PDF
models. 


The measurement of the electron charge asymmetry with
\unit{36}{\invpb} of CMS data\cite{baisini2010electron} was combined with a
similar measurement in the muon channel\cite{majumder2010muon} and published
together\cite{asym36}.  The combined electron and muon asymmetry measurement is
shown in \FigureRef{fig:combined}. The electron and muon measurements are in
good agreement with each other.

\begin{figure}[htbp]
  \begin{center}
  \includegraphics*[width=0.95\textwidth]{combined}
  \caption[Comparison of the measured lepton charge asymmetry in the muon and
electron channels] { Comparison of the measured lepton charge asymmetry in the
muon and electron channels with $\Pt>\unit{25}{\GeV}$ and $\PT>\unit{30}{\GeV}$.
The bin width is shown by the filled bars\cite{asym36}.}
  \label{fig:combined}
  \end{center}
\end{figure}

The precise measurement of the asymmetry in inclusive \PW production at the LHC
provides constraints for parton distribution functions \cite{asym840}.
For example, The NNPDF collaboration \cite{Lionetti:2011pw} have studied the
impact of the LHC \PW lepton charge asymmetry data on their PDF
sets\cite{Ball:2011gg}. \FigureRef{fig:effect} shows a comparison of the up and
down quark and anti-quark distributions before and after the CMS data has been
added\cite{Ball:2011gg}.  The lepton asymmetry measurements reduce the PDF
uncertainties of the light quarks.

\begin{figure}[htbp]
  \begin{center}
  \includegraphics*[width=0.95\textwidth]{effect}
  \caption[Comparison of the light quark and anti-quark distributions from the
NNPDF2.1 NLO PDF set and the same distribution after reweighting to account for
the CMS lepton charge asymmetry measurements with \unit{36}{\invpb} of data.]
{Comparison of the light quark and anti-quark distributions from the NNPDF2.1
NLO PDF set and the same distribution after reweighting to account for the CMS
lepton charge asymmetry measurements with \unit{36}{\invpb} of data.  The parton
densities are normalised to the NNPDF2.1 central value.  From
\cite{Ball:2011gg}. } \label{fig:effect}
  \end{center}
\end{figure}



